字符串拥有的元素数量(str.size())通常小于保留空间的元素数量:str.capacity()。若向字符串中添加元素,将不会自动分配新的内存。std::max\_size()返回一个字符串最大可以包含多少个元素。对于这三个成员函数,下列关系成立:str.size() <= str.capacity() <= str.max\_size()。

下表展示了处理字符串的内存管理的成员函数。

\begin{center}
用于创建和删除字符串的函数
\end{center}


% Please add the following required packages to your document preamble:
% \usepackage{longtable}
% Note: It may be necessary to compile the document several times to get a multi-page table to line up properly
\begin{longtable}[c]{|l|l|}
\hline
\textbf{成员函数} & \textbf{描述}                                  \\ \hline
\endfirsthead
%
\endhead
%
str.empty()               & 检查str是否有元素。                           \\ \hline
str.size(), str.length()  & 字符串的元素个数。                        \\ \hline
str.capacity()            & 无需重新分配的str可以拥有的元素数。 \\ \hline
str.max\_size()           & str可以包含的最大元素数。              \\ \hline
std.resize(n)             & 将str调整为n个元素。                             \\ \hline
str.resize\_and\_overwrite(n, op) & 将str的大小调整为n个元素,并对其元素应用op操作。 \\ \hline
str.reserve(n)            & 至少保留n个元素的内存。                \\ \hline
std.shrink\_to\_fit()     & 将字符串的容量调整为其大小。      \\ \hline
\end{longtable}

请求str.shrink\_to\_fit()与std::vector的情况一样,非绑定。

\filename{大小与容量}

\begin{cpp}
// stringSizeCapacity.cpp
...
#include <string>
...

void showStringInfo(const std::string& s){
	std::cout << s << ": ";
	std::cout << s.size() << " ";
	std::cout << s.capacity() << " ";
	std::cout << s.max_size() << " ";
}

std::string str;
showStringInfo(str); // "": 0 0 4611686018427387897

str +="12345";
showStringInfo(str); // "12345": 5 5 4611686018427387897

str.resize(30);
showStringInfo(str); // "12345": 30 30 4611686018427387897

str.reserve(1000);
showStringInfo(str); // "12345": 30 1000 4611686018427387897

str.shrink_to_fit();
showStringInfo(str); // "12345": 30 30 4611686018427387897
\end{cpp}










































