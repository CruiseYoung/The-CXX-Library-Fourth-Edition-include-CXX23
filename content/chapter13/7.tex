字符串可以通过>{}>从输入流中读取,并通过<{}<向输出流写入。

全局函数getline能够从输入流中逐行读取,直到文件结束字符。

getline函数有四种重载。前两个参数是输入流is和保存行read的字符串line,还可以指定一个特殊的行分隔符。

该函数通过对输入流的引用返回。

\begin{cpp}
istream& getline (istream& is, string& line, char delim);
istream& getline (istream&& is, string& line, char delim);
istream& getline (istream& is, string& line);
istream& getline (istream&& is, string& line);
\end{cpp}

getline表示获取整行,包括空格,只有行分隔符会忽略。该函数需要头文件<string>。

\filename{使用字符串输入和输出}

\begin{cpp}
// stringInputOutput.cpp
...
#include <string>
...

std::vector<std::string> readFromFile(const char* fileName){
	std::ifstream file(fileName);
	if (!file){
		std::cerr << "Could not open the file " << fileName << ".";
		exit(EXIT_FAILURE);
	}
	std::vector<std::string> lines;
	std::string line;
	while (getline(file , line)) lines.push_back(line);
	return lines;
}

std::string fileName;
std::cout << "Your filename: ";
std::cin >> fileName;
std::vector<std::string> lines= readFromFile(fileName.c_str());
int num{0};
for (auto line: lines) std::cout << ++num << ": " << line << '\n';
\end{cpp}

该程序显示任意文件的行,包括行号。表达式std::cin >{}> fileName读取文件名。readFromFile函数使用getline读取所有文件行,并将其压入vector对象。


































