C++提供了许多成员函数来从C或C++字符串中创建字符串。底层C字符串总是用于创建C++字符串,这在C++14中有所改变,因为新的C++标准支持C++字符串字面值:std::string str\{"string"s\}。C字符串字面值"string literal"加上后缀s就变成了C++字符串字面值"string literal"s。

该表概述了用于创建和删除C++字符串的成员函数。

\begin{center}
用于创建和删除字符串的函数
\end{center}

% Please add the following required packages to your document preamble:
% \usepackage{longtable}
% Note: It may be necessary to compile the document several times to get a multi-page table to line up properly
\begin{longtable}[c]{|l|l|}
\hline
\textbf{成员函数}        & \textbf{示例}                        \\ \hline
\endfirsthead
%
\endhead
%
默认构造                          & std::string str                         \\ \hline
从C++字符串复制         & std::string str(oth)                    \\ \hline
从C++字符串移动          & std::string str(std::move(oth))         \\ \hline
使用C++字符串的范围   & std::string(oth.begin(), oth.end())     \\ \hline
使用C++字符串的子字符串 & std::string(oth, otherIndex)            \\ \hline
使用C++字符串的子字符串 & std::string(oth, otherIndex, strlen)    \\ \hline
使用C字符串                  & std::string str("c-string")             \\ \hline
使用C数组                  & std::string str("c-array", len)         \\ \hline
使用字符                  & std::string str(num, 'c')               \\ \hline
使用初始化列表         & std::string str(\{'a', 'b', 'c', 'd'\}) \\ \hline
使用子字符串                 & str = other.substring(3, 10)            \\ \hline
析构                       & str.$\sim$string()                      \\ \hline
\end{longtable}

\filename{创建字符串}

\begin{cpp}
// stringConstructor.cpp
...
#include <string>
...
std::string defaultString;
std::string other{"123456789"};
std::string str1(other); // 123456789
std::string tmp(other); // 123456789
std::string str2(std::move(tmp)); // 123456789
std::string str3(other.begin(), other.end()); // 123456789
std::string str4(other, 2); // 3456789
std::string str5(other, 2, 5); // 34567
std::string str6("123456789", 5); // 12345
std::string str7(5, '1'); // 11111
std::string str8({'1', '2', '3', '4', '5'}); // 12345
std::cout << str6.substr(); // 12345
std::cout << str6.substr(1); // 2345
std::cout << str6.substr(1, 2); // 23
\end{cpp}































