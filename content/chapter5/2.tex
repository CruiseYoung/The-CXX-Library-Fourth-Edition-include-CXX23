

\mySamllsection{概述}

有序关联容器std::map和std::multimap将它们的键与一个值关联起来,两者都在头文件<map>中定义。std::set和std::multiset需要头文件<set>。

四个有序容器都由它们的类型、分配器和比较函数参数化。根据类型的不同,容器具有分配器和比较函数的默认值。std::map和std::set的声明很好地说明了这一点。


\begin{cpp}
template < class key, class val, class Comp= less<key>,
		   class Alloc= allocator<pair<const key, val> >
class map;

template < class T, class Comp = less<T>,
		   class Alloc = allocator<T> >
class set;
\end{cpp}


两个关联容器的声明表明std::map有一个关联值。键和值用于默认的分配器:allocator<pair<const key, val>{}>。再多一点想象力,就可以从分配器派生出更多东西。std::map具有std::pair<const key, val>类型的对子。关联值val对于排序条件不重要:less<key>,观察结果对于std::multimap和std::multiset也成立。

\mySamllsection{键和值}

对于有序关联容器的键和值有特殊的规则。

键必须是

\begin{itemize}
\item 
可排序(默认<),

\item 
可复制和可移动。
\end{itemize}

值必须是

\begin{itemize}
\item 
默认可构造,

\item 
可复制和可移动。
\end{itemize}

键关联值构建一个对子p,首先与成员键p.first,值p.second。

\begin{cpp}
#include <map>
...
std::multimap<char, int> multiMap= {{'a', 10}, {'a', 20}, {'b', 30}};
for (auto p: multiMap) std::cout << "{" << p.first << "," << p.second << "} ";
				// {a,10} {a,20} {b,30}
\end{cpp}

\mySamllsection{比较标准}

顺序关联容器的默认比较条件是std::less。若要使用用户定义类型作为键,则必须重载操作符<。为数据类型重载操作符<就足够了,因为C++运行时在关系符(!)的帮助下进行比较。(!(elem1<elem2 || elem2<elem1)),两个元素相等。

可以将排序条件指定为模板参数。这个排序标准必须实现严格的弱序。

\begin{myNotic}{严格弱序}
若满足下列条件,则给出集合S上排序准则的严格弱序。

\begin{itemize}
\item 
若s来自S,则‘s < s’不成立。

\item 
S中的所有s1和s2必须成立:若s1 < s2,则s2 < s1不成立。

\item 
对于所有s1、s2和s3,其中s1 < s2和s2 < s3必须满足:s1 < s3。

\item 
对于所有s1、s2和s3,其中s1与s2不可比较,s2与s3不可比较,必须成立:s1与s3不可比较。
\end{itemize}
\end{myNotic}

与严格弱序的定义相反,对std::map使用带有严格弱序的比较条件要简单得多。

\begin{cpp}
#include <map>
...
std::map<int, std::string, std::greater<int>> int2Str{
	{5, "five"}, {1, "one"}, {4, "four"}, {3, "three"},
	{2, "two"}, {7, "seven"}, {6, "six"} };
for (auto p: int2Str) std::cout << "{" << p.first << "," << p.second << "} ";
	// {7,seven} {6,six} {5,five} {4,four} {3,three} {2,two} {1,one}
\end{cpp}

\mySamllsection{特定的搜索函数}

有序关联容器针对搜索进行了优化,所以会提供了特定的搜索函数。

\begin{center}
有序关联容器的特殊搜索函数
\end{center}

% Please add the following required packages to your document preamble:
% \usepackage{longtable}
% Note: It may be necessary to compile the document several times to get a multi-page table to line up properly
\begin{longtable}[c]{|l|l|}
\hline
\textbf{搜索函数} & \textbf{描述}                       \\ \hline
\endfirsthead
%
\endhead
%
ordAssCont.count(key)    & 返回带有键的值的个数。 \\ \hline
ordAssCont.find(key)         & \begin{tabular}[c]{@{}l@{}}返回ordAssCont中key的迭代器。若在ordAssCont中 \\ 没有键,返回orordAssCont.end()。\end{tabular} \\ \hline
ordAssCont.lower\_bound(key) & 返回ordAssCont第一个插入键迭代器的位置。                       \\ \hline
ordAssCont.upper\_bound(key) & 返回ordAssCont最后插入键迭代器的位置。                               \\ \hline
ordAssCont.equal\_range(key) & \begin{tabular}[c]{@{}l@{}}返回std::pair中的范围ordAssCont.lower\_bound(key)和 \\ ordAssCont.upper\_bound(key)。\end{tabular}          \\ \hline
\end{longtable}

\filename{在关联容器中进行搜索}

\begin{cpp}
// associativeContainerSearch.cpp
...
#include <set>
...
std::multiset<int> mySet{3, 1, 5, 3, 4, 5, 1, 4, 4, 3, 2, 2, 7, 6, 4, 3, 6};

for (auto s: mySet) std::cout << s << " ";
	// 1 1 2 2 3 3 3 3 4 4 4 4 5 5 6 6 7
	
mySet.erase(mySet.lower_bound(4), mySet.upper_bound(4));
for (auto s: mySet) std::cout << s << " ";
	// 1 1 2 2 3 3 3 3 5 5 6 6 7
std::cout << mySet.count(3) << '\n'; // 4
std::cout << *mySet.find(3) << '\n'; // 3
std::cout << *mySet.lower_bound(3) << '\n'; // 3
std::cout << *mySet.upper_bound(3) << '\n'; // 5
auto pair= mySet.equal_range(3);
std::cout << "(" << *pair.first << "," << *pair.second << ")"; // (3,5)
\end{cpp}


\mySamllsection{std::map}

\myGraphic{0.6}{content/chapter5/images/2.jpg}{}

\href{http://en.cppreference.com/w/cpp/container/map}{std::map}是最常用的关联容器,它结合了通常足够的性能和非常方便的接口,可以通过索引操作符访问它的元素。若键不存在,std:map创建一个键/值对。对于该值,使用默认构造函数。

\begin{myTip}{将std::map看作为std::vector的泛化}
通常,std::map称为关联数组,因为std::map像序列容器一样支持索引操作符。区别在于,其索引不像std::vector那样局限于数字。std::map的索引几乎可以是任意类型。

同样的观察结果也适用于std::unordered\_map。
\end{myTip}

除了索引操作符之外,std::map还支持at成员函数。检查通过at的访问。若请求键在std::map中不存在,则抛出std::out\_of\_range异常。

















































