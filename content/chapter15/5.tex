
std::regex\_search checks if the text contains a text pattern. You can use the function with and without a std::match\_results object and apply it to a C string, a C++ string, or a range.

The example below shows how to use std::regex\_search with texts of type const char*, std::string, const wchar\_t*, and std::wstring.

\filename{std::search}

\begin{cpp}
// search.cpp
...
#include <regex>
...

// regular expression holder for time
std::regex crgx("([01]?[0-9]|2[0-3]):[0-5][0-9]");

// const char*
std::cmatch cmatch;

const char* ctime{"Now it is 23:10." };
if (std::regex_search(ctime, cmatch, crgx)){
	std::cout << ctime << '\n'; // Now it is 23:10.
	std::cout << "Time: " << cmatch[0] << '\n'; // Time: 23:10
}

// std::string
std::smatch smatch;

std::string stime{"Now it is 23:25." };
if (std::regex_search(stime, smatch, crgx)){
	std::cout << stime << '\n'; // Now it is 23:25.
	std::cout << "Time: " << smatch[0] << '\n'; // Time: 23:25
}

// regular expression holder for time
std::wregex wrgx(L"([01]?[0-9]|2[0-3]):[0-5][0-9]");

// const wchar_t*
std::wcmatch wcmatch;

const wchar_t* wctime{L "Now it is 23:47." };
if (std::regex_search(wctime, wcmatch, wrgx)){
	std::wcout << wctime << '\n'; // Now it is 23:47.
	std::wcout << "Time: " << wcmatch[0] << '\n'; // Time: 23:47
}

// std::wstring
std::wsmatch wsmatch;

std::wstring wstime{L "Now it is 00:03." };
if (std::regex_search(wstime, wsmatch, wrgx)){
	std::wcout << wstime << '\n'; // Now it is 00:03.
	std::wcout << "Time: " << wsmatch[0] << '\n'; // Time: 00:03
}
\end{cpp}






































