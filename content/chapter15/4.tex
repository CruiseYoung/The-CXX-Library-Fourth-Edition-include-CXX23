
std::regex\_match确定文本是否与文本模式匹配,可以进一步分析std::match\_results类型的搜索结果。

下面的代码片段展示了std::regex\_match的三个简单应用:一个C字符串、一个C++字符串和一个只返回布尔值的范围。这三个类型分别可用于std::match\_results对象。

\filename{std::match}

\begin{cpp}
// match.cpp
...
#include <regex>
...

std::string numberRegEx(R"([-+]?([0-9]*\.[0-9]+|[0-9]+))");
std::regex rgx(numberRegEx);
const char* numChar{"2011"};

if (std::regex_match(numChar, rgx)){
	std::cout << numChar << "is a number." << '\n';
} // 2011 is a number.

const std::string numStr{"3.14159265359"};
if (std::regex_match(numStr, rgx)){
	std::cout << numStr << " is a number." << '\n';
} // 3.14159265359 is a number.

const std::vector<char> numVec{{'-', '2', '.', '7', '1', '8', '2',
								'8', '1', '8', '2', '8'}};
if (std::regex_match(numVec.begin(), numVec.end(), rgx)){
	for (auto c: numVec){ std::cout << c ;};
	std::cout << "is a number." << '\n';
} // -2.718281828 is a number
\end{cpp}


































