
std::regex\_match determines if the text matches a text pattern. You can further analyze the search result of type std::match\_results.

The code snippet below shows three simple applications of std::regex\_match: a C string, a C++ string, and a range returning only a boolean. The three variants are available for std::match\_results objects, respectively.

\filename{std::match}

\begin{cpp}
// match.cpp
...
#include <regex>
...

std::string numberRegEx(R"([-+]?([0-9]*\.[0-9]+|[0-9]+))");
std::regex rgx(numberRegEx);
const char* numChar{"2011"};

if (std::regex_match(numChar, rgx)){
	std::cout << numChar << "is a number." << '\n';
} // 2011 is a number.

const std::string numStr{"3.14159265359"};
if (std::regex_match(numStr, rgx)){
	std::cout << numStr << " is a number." << '\n';
} // 3.14159265359 is a number.

const std::vector<char> numVec{{'-', '2', '.', '7', '1', '8', '2',
								'8', '1', '8', '2', '8'}};
if (std::regex_match(numVec.begin(), numVec.end(), rgx)){
	for (auto c: numVec){ std::cout << c ;};
	std::cout << "is a number." << '\n';
} // -2.718281828 is a number
\end{cpp}


































