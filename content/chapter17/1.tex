C++支持各种格式化函数。基本编译时格式化函数std::format、std::format\_to和std::format\_to\_n。运行时格式化函数std::vformat和std::vformat\_to与函数std::make\_format\_args结合使用。最后是,std::print和std::println。

格式化库有三个基本的格式化功能。

\noindent
\\\textbf{std::format}

返回格式化字符串。

\noindent
\textbf{std::format\_to}

通过输出迭代器写入格式化字符串

\noindent
\textbf{std::format\_to\_n}

通过输出迭代器写入格式化字符串,但不超过n个字符 \\

\begin{cpp}
#include <format>
....
std::format("{1} {0}!", "world", "Hello"); // Hello world!

std::string buffer;
std::format_to(std::back_inserter(buffer),
	"Hello, C++{}!\n", "20"); // Hello, C++20!
\end{cpp}

\mySamllsection{std::vformat, std::vformat\_to 和 std::make\_format\_args}

三个格式化函数std::format、std::format\_to和std::format\_to\_n使用格式字符串创建格式化字符串。此格式字符串必须是编译时值,所以无效的格式字符串会导致编译时错误。

对于运行时格式字符串,有两个可选函数std::vformat和std::vformat\_to,必须与std::make\_format\_args结合使用。

\begin{cpp}
#include <format>
...
std::string formatString = "{1} {0}!";
std::vformat(formatString, std::make_format_args("world", "Hello")); // Hello world
\end{cpp}

\mySamllsection{std::print 和 std::println}

函数std::print和std::println写入输出控制台。println在输出中添加一个换行符。此外,这两个函数都使它能够写入输出文件流并支持\href{https://en.wikipedia.org/wiki/Unicode}{Unicode},必须包含头文件<print>。

\begin{cpp}
#include <print>
...
std::print("{1} {0}!", "world", "Hello"); // prints "Hello world!"

std::ofstream outFile("testfile.txt");
std::print(outFile, "{1} {0}!", "world", "Hello"); // writes "Hello world!" into the out\
File
\end{cpp}

\begin{myNotic}{本章剩下的部分使用std::format}
编译时格式化函数std::format、std::format\_to和std::format\_to\_n,运行时格式化函数std::vformat和std::vformat\_n,以及函数std::print和std::println对格式字符串应用相同的语法。为简单起见,本章剩下的部分将使用std::format。
\end{myNotic}
































