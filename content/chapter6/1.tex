C++将std::stack、std::queue和std::priority\_queue作为三个特殊的序列容器。大多数人都知道这些经典的数据结构。

容器的适配器

\begin{itemize}
\item 
支持现有序列容器的简化接口;

\item 
不能与标准模板库的算法一起使用,

\item 
由数据类型和容器参数化的类模板(std::vector、std::list和std::deque);

\item 
默认使用std::deque作为内部序列容器:

\begin{cpp}
template <typename T, typename Container= deque<T>>
class stack;
\end{cpp}
\end{itemize}

\mySamllsection{栈}

\myGraphic{0.6}{content/chapter6/images/2.jpg}{}

\href{http://en.cppreference.com/w/cpp/container/stack}{std::stack}遵循后进先出原则(后进先出)。栈sta需要头文件<stack>,它有三个特殊的成员函数。

使用sta.push(e),可以在堆栈的顶部插入一个新元素e,使用sta.pop()可从顶部删除它,并使用sta.top()可引用它。堆栈支持比较操作符,并且知道期大小。栈操作具有恒定的复杂性。

\filename{std::stack}

\begin{cpp}
// stack.cpp
...
#include <stack>
...
std::stack<int> myStack;

std::cout << myStack.empty() << '\n'; // true
std::cout << myStack.size() << '\n'; // 0

myStack.push(1);
myStack.push(2);
myStack.push(3);
std::cout << myStack.top() << '\n'; // 3

while (!myStack.empty()){
	std::cout << myStack.top() << " ";
	myStack.pop();
} // 3 2 1

std::cout << myStack.empty() << '\n'; // true
std::cout << myStack.size() << '\n'; // 0
\end{cpp}


\mySamllsection{队列}

\myGraphic{0.8}{content/chapter6/images/3.jpg}{}

\href{http://en.cppreference.com/w/cpp/container/queue}{std::queue}遵循FIFO原则(先进先出)。队列que需要头文件<queue>,它有四个特殊的成员函数。

使用queue.push(e),可以在队列的末尾插入元素e,并使用queue.pop()从队列中删除第一个元素。queue.back()允许引用queue的最后一个组件queue.front()到queue中的第一个元素。std::queue与std::stack具有相似的特征。因此,可以比较std::queue实例并获得它们的大小。队列操作也具有恒定的复杂性。

\filename{std::queue}

\begin{cpp}
// queue.cpp
...
#include <queue>
...
std::queue<int> myQueue;

std::cout << myQueue.empty() << '\n'; // true
std::cout << myQueue.size() << '\n'; // 0

myQueue.push(1);
myQueue.push(2);
myQueue.push(3);
std::cout << myQueue.back() << '\n'; // 3
std::cout << myQueue.front() << '\n'; // 1

while (!myQueue.empty()){
	std::cout << myQueue.back() << " ";
	std::cout << myQueue.front() << " : ";
	myQueue.pop();
} // 3 1 : 3 2 : 3 3
	
std::cout << myQueue.empty() << '\n'; // true
std::cout << myQueue.size() << '\n'; // 0
\end{cpp}


\mySamllsection{优先级队列}

\myGraphic{0.8}{content/chapter6/images/4.jpg}{}

\href{http://en.cppreference.com/w/cpp/container/priority_queue}{std::priority\_queue}是一个简化的std::queue。它需要头文件<queue>。

与std::queue的不同之处在于,它们的最大元素总是位于优先级队列的顶部。std::priority\_queue pri默认使用比较运算符是、td::less。与std::queue类似,pri.push(e)在优先级队列中插入一个新元素e。pop()删除pri的第一个元素,但其复杂度为对数。使用pri.top(),可以引用优先级队列中的第一个元素,它是最大的一个。priority\_queue知道其大小,但不支持对其实例使用比较运算符。

\filename{std::priority\_queue}

\begin{cpp}
// priorityQueue.cpp
...
#include <queue>
...
std::priority_queue<int> myPriorityQueue;

std::cout << myPriorityQueue.empty() << '\n'; // true
std::cout << myPriorityQueue.size() << '\n'; // 0

myPriorityQueue.push(3);
myPriorityQueue.push(1);
myPriorityQueue.push(2);
std::cout << myPriorityQueue.top() << '\n'; // 3

while (!myPriorityQueue.empty()){
	std::cout << myPriorityQueue.top() << " ";
	myPriorityQueue.pop();
} // 3 2 1

std::cout << myPriorityQueue.empty() << '\n'; // true
std::cout << myPriorityQueue.size() << '\n'; // 0

std::priority_queue<std::string, std::vector<std::string>,
					std::greater<std::string>> myPriorityQueue2;

myPriorityQueue2.push("Only");
myPriorityQueue2.push("for");
myPriorityQueue2.push("testing");
myPriorityQueue2.push("purpose");
myPriorityQueue2.push(".");

while (!myPriorityQueue2.empty()){
	std::cout << myPriorityQueue2.top() << " ";
	myPriorityQueue2.pop();
} // . Only for purpose testing
\end{cpp}









