The foundation of multithreading is a well-defined memory model. This memory model has to deal with the following points:

\begin{itemize}
\item 
Atomic operations: Operations that can be performed without interruption.

\item 
Partial ordering of operations: Sequence of operations that must not be reordered.

\item 
Visible effects of operations: Guarantees when operations on shared variables are visible in other threads.
\end{itemize}

The C++ memory model has a lot in common with its predecessor: the Java memory model.
Additionally, C++ permits the breaking of sequential consistency. Sequential consistency is the default behavior of atomic operations.

Sequential consistency provides two guarantees.

\begin{enumerate}
\item 
The instructions of a program are executed in source code order.

\item 
There is a global order of all operations on all threads.
\end{enumerate}















































