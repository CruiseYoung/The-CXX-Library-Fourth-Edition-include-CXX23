汇入线程的附加功能基于std::stop\_token、std::stop\_callback和std::stop\_source。下面的代码会给让你有一个大致的概念。

\filename{中断不可中断和可中断的std::jthread}

\begin{cpp}
// interruptJthread.cpp

...

#include <thread>
#include <stop_token>

using namespace::std::literals;

...

std::jthread nonInterruptable([]{ // (1)
	int counter{0};
	while (counter < 10){
		std::this_thread::sleep_for(0.2s);
		std::cerr << "nonInterruptable: " << counter << '\n';
		++counter;
	}
});

std::jthread interruptable([](std::stop_token stoken){ // (2)
	int counter{0};
	while (counter < 10){
		std::this_thread::sleep_for(0.2s);
		if (stoken.stop_requested()) return; // (3)
		std::cerr << "interruptable: " << counter << '\n';
		++counter;
	}
});

std::this_thread::sleep_for(1s);

std::cerr << "Main thread interrupts both jthreads" << std:: endl;
nonInterruptable.request_stop(); // (4)
interruptable.request_stop();

...
\end{cpp}

主程序中启动了两个线程nonInterruptable和interruptable((1)和(2))。与线程nonInterruptable相反,线程interruptable获得一个std::stop\_token,并在(3)中使用它来检查它是否被中断:stoken\_stop\_requested()。中断的情况下,Lambda函数返回,线程结束。调用interruptable.request\_stop()(4)触发线程的结束。这并不适用于之前的调用nonInterruptable.request\_stop(),后者没有效果。

\myGraphic{0.6}{content/chapter19/images/3.jpg}{}


\mySamllsection{std::stop\_token, std::stop\_source 和 std::stop\_callback}

stop\_token、std::stop\_callback或std::stop\_source使其能够异步请求执行停止或询问执行是否获得停止信号。std::stop\_token可以传递给操作,然后用于主动轮询令牌以获得停止请求或通过std::stop\_callback注册回调。std::stop\_source发送停止请求,此信号影响所有相关的std::stop\_token。std::stop\_source、std::stop\_token和std::stop\_callback这三个类共享一个相关停止状态的所有权。调用request\_stop()、stop\_requested()和stop\_possible()是原子操作。

组件std::stop\_source和std::stop\_token为停止处理提供了以下属性。

\filename{std::stop\_source的构造函数}

\begin{cpp}
stop_source();
explicit stop_source(std::nostopstate_t) noexcept;
\end{cpp}

默认构造的std::stop\_source获得一个停止源。接受std::nostopstate\_t参数的构造函数构造了一个空的std::stop\_source,没有关联的停止状态。

\begin{center}
std::stop\_source的成员函数
\end{center}

% Please add the following required packages to your document preamble:
% \usepackage{longtable}
% Note: It may be necessary to compile the document several times to get a multi-page table to line up properly
\begin{longtable}[c]{|l|l|}
\hline
\textbf{成员函数} & \textbf{描述}                                                          \\ \hline
\endfirsthead
%
\endhead
%
src.get\_token() &
\begin{tabular}[c]{@{}l@{}}若stop\_possible(),则返回一个对应停止状态的stop\_token。\\ 否则,返回一个默认构造的(空)stop\_token。\end{tabular} \\ \hline
src.stop\_possible()     & 若src可以请求停止,则为true。                                         \\ \hline
src.stop\_requested()    & 若其中一个所有者调用stop\_possible()和request\_stop(),则为true。 \\ \hline
src.request\_stop() &
\begin{tabular}[c]{@{}l@{}}若stop\_possible()和!stop\_requested(),调用一个停止请求。\\ 否则,调用无效。\end{tabular} \\ \hline
\end{longtable}


src.stop\_possible()表示具有关联的停止状态。src.stop\_requested()在具有关联的停止状态,并且之前被请求停止时返回true。src.request\_stop()成功,若具有相关的停止状态并已接收到停止请求,则返回true。

调用src.get\_token()返回停止令牌。有了令牌,就可以检查是否已经发出停止请求,或者是否可以针对其关联的停止源src发出停止请求。停止令牌符号观察停止源src。


\begin{center}
std::stop\_token stoken的成员函数
\end{center}

% Please add the following required packages to your document preamble:
% \usepackage{longtable}
% Note: It may be necessary to compile the document several times to get a multi-page table to line up properly
\begin{longtable}[c]{|l|l|}
\hline
\textbf{成员函数} & \textbf{描述}                                                                                                                   \\ \hline
\endfirsthead
%
\endhead
%
src.stop\_possible()     & 若stoken有关联的停止状态,则返回true。                                                                                   \\ \hline
src.stop\_requested()    & \begin{tabular}[c]{@{}l@{}}若在相应的std::stop\_source src上调用了request\_stop(),则返回true;\\ 否则,返回false.\end{tabular} \\ \hline
\end{longtable}

默认构造的停止令牌没有关联的停止状态。若停止请求已经发出,stoken.stop\_possible()也返回true。stoken.stop\_requested()在停止令牌具有关联的停止状态,并且已经接收到停止请求时返回true。

若std::stop\_token应该暂时禁用,可以用默认构造的token替换它。默认构造的令牌没有关联的停止状态。下面的代码片段展示了如何禁用和启用线程接受停止请求的能力。

\filename{暂时禁用停止令牌}

\begin{cpp}
std::jthread jthr([](std::stop_token stoken) {
	...
	std::stop_token interruptDisabled;
	std::swap(stoken, interruptDisabled); // (1)
	... // (2)
	std::swap(stoken, interruptDisabled);
	...
}
\end{cpp}

std::stop\_token interruptDisabled没有关联的停止状态,所以线程jthr可以接受除(1)和(2)行以外的所有行中的停止请求。













