若多个线程共享一个变量,则必须协调访问。这就是C++中互斥锁的工作。

\mySamllsection{数据竞争}

\noindent
\textbf{数据竞争}

数据竞争是指至少有两个线程同时访问共享数据,并且至少有一个线程是写线程的状态。因此程序具有未定义行为。

若有几个线程写入std::cout,可以很好地观察到线程的交错。在本例中,输出流std::cout是共享变量。

\filename{不同步的写入std::cout}

\begin{cpp}
// withoutMutex.cpp
...
#include <thread>
...

using namespace std;

struct Worker{
	Worker(string n):name(n){};
	void operator() (){
		for (int i= 1; i <= 3; ++i){
			this_thread::sleep_for(chrono::milliseconds(200));
			cout << name << ": " << "Work " << i << endl;
		}
	}
	private:
	string name;
};

thread herb= thread(Worker("Herb"));
thread andrei= thread(Worker(" Andrei"));
thread scott= thread(Worker ("     Scott"));
thread bjarne= thread(Worker("       Bjarne"));
\end{cpp}

\myGraphic{0.8}{content/chapter19/images/4.jpg}{}

std::cout的输出不协调。

\begin{myTip}{流是线程安全的}
C++11标准保证字符是自动写入的,因此不需要保护。只有当整个读操作不交错时,才需要保护流上线程的交错。这种保证适用于输入和输出流。

C++20中,已经同步了像std::osyncstream和std::wosyncstream这样的输出流。保证对一个输出流的写入是同步的。输出写入内部缓冲区,并在超出作用域时刷新。同步输出流可以有一个名称,比如synced\_out,也可以没有名称。

\filename{同步输出流}

\begin{cpp}
{
	std::osyncstream synced_out(std::cout);
	synced_out << "Hello, ";
	synced_out << "World!";
	synced_out << '\n'; // no effect
	synced_out << "and more!\n";
} // destroys the synced_output and emits the internal buffer

std::osyncstream(std::cout) << "Hello, " << "World!" << "\n";
\end{cpp}
\end{myTip}

本例中,std::cout是共享变量,应该对流具有独占访问权。

\mySamllsection{互斥锁}

互斥(互斥量)保证一次只有一个线程可以访问临界区域。它们需要头文件<mutex>。互斥锁m 通过调用m.lock()来锁定临界区,并通过m.lock()来解锁。

\filename{使用std::mutex进行同步}

\begin{cpp}
// mutex.cpp
...
#include <mutex>
#include <thread>
...

using namespace std;

std::mutex mutexCout;

struct Worker{
	Worker(string n):name(n){};
	void operator() (){
		for (int i= 1; i <= 3; ++i){
			this_thread::sleep_for(chrono::milliseconds(200));
			mutexCout.lock();
			cout << name << ": " << "Work " << i << endl;
			mutexCout.unlock();
		}
	}
	private:
	string name;
};

thread herb= thread(Worker("Herb"));
thread andrei= thread(Worker(" Andrei"));
thread scott= thread(Worker ("    Scott"));
thread bjarne= thread(Worker("      Bjarne"));
\end{cpp}

因为它使用相同的互斥锁mutexCout,所以每个线程在彼此之后协调写入std::cout。

\myGraphic{0.6}{content/chapter19/images/5.jpg}{}

C++有五种不同的互斥量。它们可以递归地锁定,暂时有或没有时间限制。

\begin{center}
C++的互斥量
\end{center}

% Please add the following required packages to your document preamble:
% \usepackage{longtable}
% Note: It may be necessary to compile the document several times to get a multi-page table to line up properly
\begin{longtable}[c]{|l|l|l|l|l|l|}
\hline
\textbf{成员函数} & \textbf{mutex} & \textbf{recursive\_mutex} & \textbf{timed\_mutex} & \textbf{recursive\_timed\_mutex} & \textbf{shared\_timed\_mutex} \\ \hline
\endfirsthead
%
\endhead
%
m.lock             & yes & yes & yes & yes & yes \\ \hline
m.unlock           & yes & yes & yes & yes & yes \\ \hline
m.try\_lock        & yes & yes & yes & yes & yes \\ \hline
m.try\_lock\_for   &     &     & yes & yes & yes \\ \hline
m.try\_lock\_until &     &     & yes & yes & yes \\ \hline
\end{longtable}


std::shared\_time\_mutex使它能够实现读写锁。成员函数m.t try\_lock\_for(relTime)需要一个相对时间段;成员函数m.t try\_lock\_until(absTime)一个绝对时间点。


\mySamllsection{死锁}

死锁是两个或多个线程被阻塞的状态,因为每个线程在释放其资源之前等待资源的释放。

若忘记调用m.k unlock(),可能很快就会出现死锁。例如,在函数getVar()中出现异常的情况下,就会发生这种情况。

\begin{cpp}
m.lock();
sharedVar= getVar();
m.unlock()
\end{cpp}

\begin{myWarning}{不要在持有锁时调用未知函数}
若函数getVar试图通过调用m.lock()来获取相同的锁,将会死锁。因为它不会成功,并且调用将永远阻塞。
\end{myWarning}

以错误的顺序锁定两个互斥锁,是导致死锁的另一个典型原因。

\filename{A deadlock}

\begin{cpp}
// deadlock.cpp
...
#include <mutex>
...

struct CriticalData{
	std::mutex mut;
};

void deadLock(CriticalData& a, CriticalData& b){
	a.mut.lock();
	std::cout << "get the first mutex\n";
	std::this_thread::sleep_for(std::chrono::milliseconds(1));
	b.mut.lock();
	std::cout << "get the second mutex\n";
	a.mut.unlock(), b.mut.unlock();
}

CriticalData c1;
CriticalData c2;

std::thread t1([&]{ deadLock(c1, c2); });
std::thread t2([&]{ deadLock(c2, c1); });
\end{cpp}

一毫秒的短时间窗口(std::this\_thread::sleep\_for(std::chrono::milliseconds(1)))足以产生高概率的死锁,因为每个线程都在等待另一个互斥锁。

\myGraphic{0.6}{content/chapter19/images/6.jpg}{}

\begin{myTip}{将互斥锁封装在锁中}
很容易忘记解锁互斥锁或以不同的顺序锁定互斥锁。为了克服互斥锁的大多数问题,可以将互斥锁封装在锁中。
\end{myTip}

\mySamllsection{锁}

应该将互斥锁封装在锁中以自动释放互斥锁。锁是RAII习惯用法的一种实现,因为锁将互斥锁的生命周期与它的生命周期绑定在一起。C++11分别为简单用例提供了std::lock\_guard,为高级用例提供了std::unique\_lock。两者都需要头文件<mutex>。C++14中,有一个std::shared\_lock,与互斥锁std::shared\_time\_mutex结合在一起使用,是实现读写锁的基础。

\mySamllsection{std::lock\_guard}

std::lock\_guard只支持简单的用例,所以只能在构造函数中绑定互斥锁,并在析构函数中释放互斥锁。因此,worker示例的同步可简化为构造函数的调用。

\filename{使用std::lock\_guard进行同步}

\begin{cpp}
// lockGuard.cpp
...
std::mutex coutMutex;

struct Worker{
	Worker(std::string n):name(n){};
	void operator() (){
		for (int i= 1; i <= 3; ++i){
			std::this_thread::sleep_for(std::chrono::milliseconds(200));
			std::lock_guard<std::mutex> myLock(coutMutex);
			std::cout << name << ": " << "Work " << i << '\n';
		}
	}
	private:
	std::string name;
};
\end{cpp}

\mySamllsection{std::unique\_lock}

使用std::unique\_lock的开销比使用std::lock\_guard的开销大。相反,std::unique\_lock可以在有互斥锁和没有互斥锁的情况下创建,显式地锁定或释放它的互斥锁,或者延迟它的互斥锁。

下表显示了std::unique\_lock类的成员函数。

\begin{center}
std::unique\_lock lk的接口
\end{center}

% Please add the following required packages to your document preamble:
% \usepackage{longtable}
% Note: It may be necessary to compile the document several times to get a multi-page table to line up properly
\begin{longtable}[c]{|l|l|}
\hline
\textbf{成员函数}                                                      & \textbf{描述}                                         \\ \hline
\endfirsthead
%
\endhead
%
lk.lock()                                                                     & 锁定关联的互斥锁。                                  \\ \hline
std::lock(lk1, lk2, ...)                                                      & 自动锁定任意数量的关联互斥锁。 \\ \hline
\begin{tabular}[c]{@{}l@{}}lk.try\_lock() and\\ lk.try\_lock\_for(relTime) and\\ lk.try\_lock\_until(absTime)\end{tabular} & 尝试锁定关联的互斥锁。 \\ \hline
lk.release()                                                                  & 释放互斥锁。互斥锁保持锁定状态。                \\ \hline
\begin{tabular}[c]{@{}l@{}}lk.swap(lk2) and\\ std::swap(lk, lk2)\end{tabular} & 交换锁。                                             \\ \hline
lk.mutex()                                                                    & 返回一个指向关联互斥对象的指针。                  \\ \hline
lk.owns\_lock()                                                               & 检查锁是否有互斥锁。                              \\ \hline
\end{longtable}

由于获取锁的顺序不同而导致的死锁,可以很通过std::atomic解决。

\filename{std::unique\_lock}

\begin{cpp}
// deadLockResolved. cpp
...
#include <mutex>
...

using namespace std;
struct CriticalData{
	mutex mut;
};

void deadLockResolved(CriticalData& a, CriticalData& b){
	unique_lock<mutex>guard1(a.mut, defer_lock);
	cout << this_thread::get_id() << ": get the first lock" << endl;
	this_thread::sleep_for(chrono::milliseconds(1));
	unique_lock<mutex>guard2(b.mut, defer_lock);
	cout << this_thread::get_id() << ": get the second lock" << endl;
	cout << this_thread::get_id() << ": atomic locking";
	lock(guard1, guard2);
}

CriticalData c1;
CriticalData c2;

thread t1([&]{ deadLockResolved(c1, c2); });
thread t2([&]{ deadLockResolved(c2, c1); });
\end{cpp}

由于std::unique\_lock的参数std::defer\_lock,a.mut和b.mut则延迟锁定。锁定在调用std::lock(guard1, guard2)时自动发生。

\myGraphic{0.6}{content/chapter19/images/7.jpg}{}

\mySamllsection{std::shared\_lock}

std::shared\_lock与std::unique\_lock具有相同的接口。另外,std::shared\_lock支持多个线程共享同一个锁定互斥量的情况。对于这个特殊的用例,必须将std::shared\_lock与std::shared\_timed\_mutex结合使用。但若多个线程在一个std::unique\_lock中使用同一个std::shared\_time\_mutex,则只有一个线程可以拥有它。

\begin{cpp}
#include <mutex>
...

std::shared_timed_mutex sharedMutex;

std::unique_lock<std::shared_timed_mutex> writerLock(sharedMutex);

std::shared_lock<std::shared_time_mutex> readerLock(sharedMutex);
std::shared_lock<std::shared_time_mutex> readerLock2(sharedMutex);
\end{cpp}

该示例展示了典型的读写锁场景。类型std::unique\_lock<std::shared\_timed\_mutex>的writerLock只能独占sharedMutex。std::share\_lock<std::shared\_time\_mutex>类型的读锁readerLock和readerLock2可以共享同一个互斥锁sharedMutex。

\mySamllsection{线程安全的初始化}

若不修改数据,则以线程安全的方式初始化就足够了。C++提供了多种方法来实现这一点:使用常量表达式,使用具有块作用域的静态变量,或者使用函数std::call\_once结合标志std::once::flag。


\mySamllsection{常数表达式}

常量表达式在编译时初始化,它们本身是线程安全的。通过在变量前使用关键字constexpr,变量就变成了常量表达式。用户定义类型的实例也可以是常量表达式,若将成员函数声明为常量表达式,则可以以线程安全的方式初始化。

\begin{cpp}
struct MyDouble{
	constexpr MyDouble(double v):val(v){};
	constexpr double getValue(){ return val; }
private:
	double val
};

constexpr MyDouble myDouble(10.5);
std::cout << myDouble.getValue(); // 10.5
\end{cpp}

\mySamllsection{块中的静态变量}

若在块中定义静态变量,C++11运行时保证以线程安全的方式初始化它。

\begin{cpp}
void blockScope(){
	static int MySharedDataInt= 2011;
}
\end{cpp}

\mySamllsection{std::call\_once 和 std::once\_flag}

std::call\_once接受两个参数:标志std::once\_flag和一个可调用对象。C++运行时借助标志std::once\_flag保证可调用对象只执行一次。

\filename{T线程安全的初始化}

\begin{cpp}
// callOnce.cpp
...
#include <mutex>
...

using namespace std;

once_flag onceFlag;
void do_once(){
	call_once(onceFlag, []{ cout << "Only once." << endl; });
}
thread t1(do_once);
thread t2(do_once);
\end{cpp}

虽然两个线程都执行了函数do\_once,但只有一个线程执行成功,而Lambda函数[]\{cout <{}< " only once "。<{}< endl;\}只执行一次。

\myGraphic{0.6}{content/chapter19/images/8.jpg}{}

可以使用相同的std::once\_flag来注册不同的可调用对象,并且只调用其中一个可调用对象。


