
要使用C++的多线程接口,需要头文件<thread>。C++11中支持std::thread,C++20中支持改进的std::jthread。

\mySamllsection{std::thread}

\mySamllsection{创建}

线程std::thread表示一个可执行单元。线程立即启动的可执行单元将其工作包作为可调用单元。可调用单元可以是函数、函数对象或Lambda函数:

\filename{创建线程}

\begin{cpp}
// threadCreate.cpp
...
#include <thread>
...
using namespace std;

void helloFunction(){
	cout << "function" << endl;
}

class HelloFunctionObject {
public:
	void operator()() const {
		cout << "function object" << endl;
	}
};

thread t1(helloFunction); // function

HelloFunctionObject helloFunctionObject;
thread t2(helloFunctionObject); // function object

thread t3([]{ cout << "lambda function"; }); // lambda function
\end{cpp}

\mySamllsection{生命周期}

线程的创建者必须关心所创建线程的生命周期,创建的线程的可执行单元以可调用线程的结尾结束。要么创建者等待创建的线程t完成(t.j join()),要么创建者将自己从创建的线程中分离出来:t.d detach()。若没有调用t.join()或t.detach(),则线程t是可汇入的。可接合线程在其析构函数中抛出异常std::terminate,程序终止。

\filename{线程的生命周期}

\begin{cpp}
// threadLifetime.cpp
...
#include <thread>
...

thread t1(helloFunction); // function

HelloFunctionObject helloFunctionObject;
thread t2(helloFunctionObject); // function object

thread t3([]{ cout << "lambda function"; }); // lambda function

t1.join();
t2.join();
t3.join();
\end{cpp}

从其创建者分离的线程通常称为守护线程,因为它在后台运行。

\begin{myWarning}{小心移动线程}
线程可以移动,但不能复制。

\begin{cpp}
#include <thread>
...
std::thread t([]{ cout << "lambda function"; });
std::thread t2;
t2 = std::move(t);

std::thread t3([]{ cout << "lambda function"; });
t2 = std::move(t3); // std::terminate
\end{cpp}

通过执行t2 = std::move(t),线程t2获得线程t的可调用对象。假设线程t2已经有一个可调用对象并且是可连接的,C++运行时将调用std::terminate。这恰好发生在t2 = std::move(t3)中,因为t2之前既没有执行t2.join(),也没有执行t2.detach()。
\end{myWarning}

\mySamllsection{参数}

std::thread是可变模板,可以通过复制或引用获得任意数量的参数。可调用对象或线程都可以获得参数,线程将它们委托给可调用对象:tPerCopy2和tPerReference2。

\begin{cpp}
#include <thread>
...

using namespace std;

void printStringCopy(string s){ cout << s; }
void printStringRef(const string& s){ cout << s; }

string s{"C++"};

thread tPerCopy([=]{ cout << s; }); // C++
thread tPerCopy2(printStringCopy, s); // C++
tPerCopy.join();
tPerCopy2.join();

thread tPerReference([&]{ cout << s; }); // C++
thread tPerReference2(printStringRef, s); // C++
tPerReference.join();
tPerReference2.join();
\end{cpp}

前两个线程通过复制获得参数,后两个线程通过引用获得参数。

\begin{myWarning}{默认情况下,线程应该通过复制获得参数}
\filename{线程的参数}
	
\begin{cpp}
// threadArguments.cpp
...
#include <thread>
...

using std::this_thread::sleep_for;
using std::this_thread::get_id;

struct Sleeper{
	Sleeper(int& i_):i{i_}{};
	void operator() (int k){
		for (unsigned int j= 0; j <= 5; ++j){
			sleep_for(std::chrono::milliseconds(100));
			i += k;
		}
		std::cout << get_id(); // undefined behaviour
	}
	private:
	int& i;
};

int valSleeper= 1000;

std::thread t(Sleeper(valSleeper), 5);
t.detach();

std::cout << valSleeper; // undefined behaviour
\end{cpp}

这个程序段有未定义行为。首先,std::cout的生命周期绑定到主线程的生命周期,创建的线程通过引用获得它的变量valSleeper。问题是创建的线程可能比它的创建者活得更长,若主线程结束,std::cout和valSleeper将失去其有效性。其次,valSleeper是一个共享变量,由主线程和子线程并发使用,这会是一场数据竞赛。
\end{myWarning}

\mySamllsection{操作}

可以在一个线程上执行许多操作。

\begin{center}
std::thread的操作
\end{center}

% Please add the following required packages to your document preamble:
% \usepackage{longtable}
% Note: It may be necessary to compile the document several times to get a multi-page table to line up properly
\begin{longtable}[c]{|l|l|}
\hline
\textbf{成员函数}                                                   & \textbf{描述}                                      \\ \hline
\endfirsthead
%
\endhead
%
t.join()                                                                   & 直到线程t完成了可执行单元。    \\ \hline
t.detach()                                                                 & 独立于创建者执行创建的线程。 \\ \hline
t.joinable()                                                               & 检查线程t是否支持join或detach调用。     \\ \hline
\begin{tabular}[c]{@{}l@{}}t.get\_id() and\\ std::this\_thread::get\_id()\end{tabular} & 返回线程的标识。 \\ \hline
std::thread::hardware\_concurrency()                                       & 表示可以并行运行的线程数。 \\ \hline
std::this\_thread::sleep\_until(absTime)                                   & 让线程进入睡眠状态,直到absTime。            \\ \hline
std::this\_thread::sleep\_for(relTime)                                     & 使线程在relTime持续时间内休眠。      \\ \hline
std::this\_thread::yield()                                                 & 让系统运行另一个线程。                  \\ \hline
\begin{tabular}[c]{@{}l@{}}t.swap(t2) and\\ std::swap(t1, t2)\end{tabular} & 交换线程。                                        \\ \hline
\end{longtable}

只能在线程t上调用一次t.join()或t.detach()。若多次调用,会得到异常std::system\_error。std::thread::hardware\_concurrency返回内核数,若运行时无法确定内核数,则返回0。sleep\_until和sleep\_for操作需要一个时间点或时间持续时间作为参数。

线程不能复制,但可以移动。交换操作在可能的情况下执行移动操作。

\filename{线程上的操作}

\begin{cpp}
// threadMember Functions.cpp
...
#include <thread>
...
using std::this_thread::get_id;

std::thread::hardware_concurrency(); // 4

std::thread t1([]{ get_id(); }); // 139783038650112
std::thread t2([]{ get_id(); }); // 139783030257408
t1.get_id(); // 139783038650112
t2.get_id(); // 139783030257408

t1.swap(t2);

t1.get_id(); // 139783030257408
t2.get_id(); // 139783038650112
get_id(); // 140159896602432
\end{cpp}

\mySamllsection{std::jthread}

std::jthread代表可汇入线程。除了C++11中的std::thread之外,std::jthread可以自动加入已启动的线程并发出中断信号。

\mySamllsection{自动汇入}

std::thread的非直观行为如下:当std::thread仍然可汇入,std::terminate在其析构函数中调用。

相反,若std::jthread仍然可汇入,则自动连接到其析构函数中。

\filename{终止仍然可汇入的std::jthread}

\begin{cpp}
// jthreadJoinable.cpp
...
#include <thread>
...

std::jthread thr{[]{ std::cout << "std::jthread" << "\n"; }}; // std::jthread

std::cout << "thr.joinable(): " << thr.joinable() << "\n"; // thr.joinable(): true
\end{cpp}

除了std::thread之外,std::jthread也是可中断的。














