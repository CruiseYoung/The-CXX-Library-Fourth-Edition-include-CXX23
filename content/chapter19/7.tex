条件变量允许通过消息同步线程,则需要头文件<condition\_variable>。一个线程充当消息的发送者,另一个线程充当消息的接收者。接收方等待发送方的通知。条件变量的典型用例是生产者-消费者工作流。

条件变量可以是消息的发送方和接收方。

\begin{center}
条件变量cv的成员函数
\end{center}

% Please add the following required packages to your document preamble:
% \usepackage{longtable}
% Note: It may be necessary to compile the document several times to get a multi-page table to line up properly
\begin{longtable}[c]{|l|l|}
\hline
\textbf{成员函数} & \textbf{描述}          \\ \hline
\endfirsthead
%
\endhead
%
cv.notify\_one()         & 通知一个等待线程。    \\ \hline
cv.notify\_all()         & 通知所有等待的线程。 \\ \hline
cv.wait(lock, ...)                 & 在持有std::unique\_lock的同时等待通知。                     \\ \hline
cv.wait\_for(lock, relTime, ...)   & 在持有std::unique\_lock的同时,在给定时间段内等待等待通知。 \\ \hline
cv.wait\_until(lock, absTime, ...) & 在持有std::unique\_lock的同时,在给定时间段内等待等待通知。        \\ \hline
\end{longtable}


发送方和接收方需要一个锁。在发送方的情况下,std::lock\_guard就足够了,只调用一次lock和unlock。对于接收方来说,std::unique\_lock是必要的,通常会锁定和解锁它的互斥锁几次。

\filename{条件变量}

\begin{cpp}
// conditionVariable.cpp
...
#include <condition_variable>
...

std::mutex mutex_;
std::condition_variable condVar;
bool dataReady= false;

void doTheWork(){
	std::cout << "Processing shared data." << '\n';
}

void waitingForWork(){
	std::cout << "Worker: Waiting for work." << '\n';
	std::unique_lock<std::mutex> lck(mutex_);
	condVar.wait(lck, []{ return dataReady; });
	doTheWork();
	std::cout << "Work done." << '\n';
}

void setDataReady(){
	std::lock_guard<std::mutex> lck(mutex_);
	dataReady=true;
	std::cout << "Sender: Data is ready." << '\n';
	condVar.notify_one();
}

std::thread t1(waitingForWork);
std::thread t2(setDataReady);
\end{cpp}

\myGraphic{0.6}{content/chapter19/images/10.jpg}{}

使用条件变量听起来很简单,但是有两个关键问题。

\begin{myWarning}{避免伪唤醒}
为了保护自身免受伪唤醒,条件变量的wait调用应该使用相关的谓词。谓词确保通知确实来自发送方。我使用Lambda函数[]\{return dataReady;\}作为谓词。发送方将dataReady设置为true。
\end{myWarning}

\begin{myWarning}{避免未唤醒}
为了保护自己未唤醒,条件变量的wait调用应该使用额外的谓词。谓词确保发送方的通知不会丢失。若发送方在接收方等待之前通知接收方,则通知丢失。接收者将永远等待。接收方现在首先检查它的谓词:[]\{return dataReady;\}。
\end{myWarning}
