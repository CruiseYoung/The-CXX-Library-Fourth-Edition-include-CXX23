锁存器和栅栏是允许某些线程阻塞直到计数器变为零的协调类型。在C++20中,有两个版本的锁存器和栅栏:std::latch和std::barrier。

\mySamllsection{std::latch}

现在,来仔细看看std::latch的接口。

\begin{center}
std::latch lat的成员函数
\end{center}

% Please add the following required packages to your document preamble:
% \usepackage{longtable}
% Note: It may be necessary to compile the document several times to get a multi-page table to line up properly
\begin{longtable}[c]{|l|l|}
\hline
\textbf{成员函数} & \textbf{描述}         \\ \hline
\endfirsthead
%
\endhead
%
lat.count\_down(upd=1)       & 不阻塞调用者的情况下,通过upd自动减少计数器。  \\ \hline
lat.try\_wait()          & 若counter == 0则返回true \\ \hline
lat.wait()                   & 若counter == 0,立即返回。否则,阻塞直到counter == 0。 \\ \hline
lat.arrive\_and\_wait(upd=1) & 相当于count\_down(upd);wait();。                                \\ \hline
\end{longtable}

upd的默认值为1。当upd大于计数器或为负值时,程序具有未定义行为。调用lat.try\_wait()不像它的名字所示的那样,会等待。

下面的程序使用两个std::latch来构建一个“老板与雇员们”的工作流。我使用synchronizedOut函数将输出同步到std::cout(第13行)。这种同步使遵循工作流变得更加容易。

\filename{std::latch}

\begin{cpp}
// bossWorkers.cpp

#include <latch>
...

std::latch workDone(2);
std::latch goHome(1); // (5)

void synchronizedOut(const std::string s) {
	std::lock_guard<std::mutex> lo(coutMutex);
	std::cout << s;
}

class Worker {
public:
	Worker(std::string n): name(n) { };
	void operator() (){
		// notify the boss when work is done
		synchronizedOut(name + ": " + "Work done!\n");
		workDone.count_down(); // (3)
		// waiting before going home
		goHome.wait();
		synchronizedOut(name + ": " + "Good bye!\n");
	}
private:
	std::string name;
};

...

std::cout << "BOSS: START WORKING! " << '\n';
Worker herb("   Herb"); // (1)
std::thread herbWork(herb);

Worker scott("    Scott"); // (2)
std::thread scottWork(scott);

workDone.wait(); // (4)

std::cout << '\n';

goHome.count_down();

std::cout << "BOSS: GO HOME!" << '\n';

herbWork.join();
scottWork.join();
\end{cpp}

工作流的概念很简单。两个雇员herb和scott(第1行和第2行)必须完成他们的工作。当它们完成它们的工作(第3行)时,其倒数std::latch workDone。boss(主线程)在第(4)行被阻塞,直到计数器变为0。当计数器为0时,老板使用第二个std::latch goHome来通知它的雇员回家。在这种情况下,初始计数器为1(第5行)。调用goHome.wait(0)阻塞,直到计数器变为0。

\myGraphic{0.6}{content/chapter19/images/12.jpg}{}

std::barrier类似于std::latch。

\mySamllsection{std::barrier}

std::latch和std::barrier有两个不同之处。首先,可以多次使用std::barrier;其次,可以为下一步(迭代)设置计数器。计数器变为零后,立即开始所谓的完成步骤。在这个完成步骤中,将调用\href{https://en.cppreference.com/w/cpp/named_req/Callable}{可调用对象}。栅栏在其构造函数中获得可调用对象。

完成步骤如下:

\begin{enumerate}
\item 
所有线程都阻塞。

\item 
任意线程解除阻塞,并执行可调用对象。

\item 
若完成步骤,所有线程都将解除阻塞。
\end{enumerate}

\begin{center}
std::barrier bar的成员函数
\end{center}

% Please add the following required packages to your document preamble:
% \usepackage{longtable}
% Note: It may be necessary to compile the document several times to get a multi-page table to line up properly
\begin{longtable}[c]{|l|l|}
\hline
\textbf{成员函数} & \textbf{描述}                           \\ \hline
\endfirsthead
%
\endhead
%
bar.arrive(upd)          & 计数器按upd自动递减。          \\ \hline
bar.wait()              & 阻塞在同步点,直到完成步骤。  \\ \hline
bar.arrive\_and\_wait()  & 相当于wait(arrive())                  \\ \hline
bar.arrive\_and\_drop() & 将当前和后续相位的计数器减1。\\ \hline
std::barrier::max        & 实现支持的最大值。 \\ \hline
\end{longtable}

bar.arrive\_and\_drop()调用本质上意味着计数器在下一阶段减1。


















