首先,不要叫他们\href{https://en.wikipedia.org/wiki/Functor}{函子}。这是范畴论中定义很好的术语。

\href{http://en.cppreference.com/w/cpp/utility/functional}{函数对象}是行为类似函数的对象。由于函数对象是对象,它们可以有属性,因为调用操作符实现了,所以也可以有状态。

\begin{cpp}
struct Square{
	void operator()(int& i){i= i*i;}
};

std::vector<int> myVec{1, 2, 3, 4, 5, 6, 7, 8, 9, 10};

std::for_each(myVec.begin(), myVec.end(), Square());

for (auto v: myVec) std::cout << v << " "; // 1 4 9 16 25 36 49 64 81 100
\end{cpp}

\begin{myTip}{实例化函数对象以供使用}
	
在算法:std::for\_each(myVec.begin(), myVec.end(), Square)中只使用函数对象的名称(Square)而不使用函数对象的实例(Square())是一个常见的错误。当然,这是一个错误。必须使用实例:std::for\_each(myVec.begin(), myVec.end(), Square())
\end{myTip}

\mySamllsection{预定义函数对象}

C++提供了一系列预定义的函数对象,需要头文件<functional>。这些预定义的函数对象有助于更改容器的默认行为。例如,有序关联容器的键在默认情况下使用预定义的函数对象std::less进行排序,但你可能想用std::greater来代替:

\begin{cpp}
std::map<int, std::string> myDefaultMap; // std::less<int>
std::map<int, std::string, std::greater<int> mySpecialMap>; // std::greater<int>
\end{cpp}

标准模板库中有用于算术、逻辑、位运算、求反和比较的函数对象。

\begin{center}
预定义功能对象
\end{center}

% Please add the following required packages to your document preamble:
% \usepackage{longtable}
% Note: It may be necessary to compile the document several times to get a multi-page table to line up properly
\begin{longtable}[c]{|l|l|}
\hline
\textbf{函数对象} &
\textbf{代表} \\ \hline
\endfirsthead
%
\endhead
%
否定 &
std::negate\textless{}T\textgreater{}() \\ \hline
Arithmetic &
\begin{tabular}[c]{@{}l@{}}std::plus\textless{}T\textgreater{}(), std::minus\textless{}T\textgreater{}()\\ std::multiplies\textless{}T\textgreater{}(), std::divides\textless{}T\textgreater{}()\\ std::modulus\textless{}T\textgreater{}()\end{tabular} \\ \hline
比较 &
\begin{tabular}[c]{@{}l@{}}std::equal\_to\textless{}T\textgreater{}(), std::not\_equal\_to\textless{}T\textgreater{}()\\ std::less\textless{}T\textgreater{}(), std::greater\textless{}T\textgreater{}()\\ std::less\_equal\textless{}T\textgreater{}(), std::greater\_equal\textless{}T\textgreater{}()\end{tabular} \\ \hline
逻辑 &
\begin{tabular}[c]{@{}l@{}}std::logical\_not\textless{}T\textgreater{}()\\ std::logincl\_and\textless{}T\textgreater{}(), std::logical\_or\textless{}T\textgreater{}()\end{tabular} \\ \hline
位域 &
\begin{tabular}[c]{@{}l@{}}std::bit\_and\textless{}T\textgreater{}(), std::bit\_or\textless{}T\textgreater{}()\\ std::bit\_xor\textless{}T\textgreater{}()\end{tabular} \\ \hline
\end{longtable}


























