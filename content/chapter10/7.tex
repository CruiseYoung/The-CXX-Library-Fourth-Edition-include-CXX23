

\begin{myTip}{什么是分区?}
一个集合的划分是一个集合在子集中的分解,使得集合的每个元素精确地在一个子集中。二元谓词定义子集,以便第一个子集的成员满足谓词。剩下的元素在第二个子集中。
\end{myTip}

C++提供了一些处理分区的函数,都需要一个一元谓词。std::partition和std::stable\_partition划分一个范围并返回分区点。使用std::partition\_point,可以获得分区的分区点。之后,可以使用std::is\_partitioned检查分区,或者使用std::partition\_copy复制分区。

检查范围是否被分区:

\begin{cpp}
bool is_partitioned(InpIt first, InpIt last, UnPre pre)
bool is_partitioned(ExePol pol, FwdIt first, FwdIt last, UnPre pre)
\end{cpp}

范围分区:

\begin{cpp}
FwdIt partition(FwdIt first, FwdIt last, UnPre pre)
FwdIt partition(ExePol pol, FwdIt first, FwdIt last, UnPre pre)
\end{cpp}

范围分区(稳定):

\begin{cpp}
BiIt stable_partition(FwdIt first, FwdIt last, UnPre pre)
BiIt stable_partition(ExePol pol, FwdIt first, FwdIt last, UnPre pre)
\end{cpp}

在两个范围内复制分区:

\begin{cpp}
pair<OutIt1, OutIt2> partition_copy(InIt first, InIt last,
					 OutIt1 result_true, OutIt2 result_false, UnPre pre)
pair<FwdIt1, FwdIt2> partition_copy(ExePol pol, FwdIt1 first, FwdIt1 last,
					 FwdIt2 result_true, FwdIt3 result_false, UnPre pre)
\end{cpp}

返回分区点:

\begin{cpp}
FwdIt partition_point(FwdIt first, FwdIt last, UnPre pre)
\end{cpp}

与std::partition相反,std::stable\_partition保证元素保持它们的相对顺序。返回的迭代器FwdIt和BiIt指向分区初始位置的第二个子集。算法std::partition\_copy的pair std::pair<OutIt, OutIt>包含子集result\_true和result\_false的结束迭代器。若范围未分区,则std::partition\_point的行为未定义。

\filename{分区算法}

\begin{cpp}
// partition.cpp
...
#include <algorithm>
...
using namespace std;

bool isOdd(int i){ return (i%2) == 1; }
vector<int> vec{1, 4, 3, 4, 5, 6, 7, 3, 4, 5, 6, 0, 4,
				8, 4, 6, 6, 5, 8, 8, 3, 9, 3, 7, 6, 4, 8};
auto parPoint= partition(vec.begin(), vec.end(), isOdd);
for (auto v: vec) cout << v << " ";
					// 1 7 3 3 5 9 7 3 3 5 5 0 4 8 4 6 6 6 8 8 4 6 4 4 6 4 8

for (auto v= vec.begin(); v != parPoint; ++v) cout << *v << " ";
					// 1 7 3 3 5 9 7 3 3 5 5
for (auto v= parPoint; v != vec.end(); ++v) cout << *v << " ";
					// 4 8 4 6 6 6 8 8 4 6 4 4 6 4 8

cout << is_partitioned(vec.begin(), vec.end(), isOdd); // true
list<int> le;
list<int> ri;
partition_copy(vec.begin(), vec.end(), back_inserter(le), back_inserter(ri),
			   [](int i) { return i < 5; });
for (auto v: le) cout << v << ""; // 1 3 3 3 3 0 4 4 4 4 4 4
for (auto v: ri) cout << v << ""; // 7 5 9 7 5 5 8 6 6 6 8 8 6 6 8
\end{cpp}



















