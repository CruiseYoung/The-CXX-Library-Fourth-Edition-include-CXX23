

\begin{myTip}{What is a partition?}

A partition of a set is a decomposition of a set in subsets so that each element of the set is precisely in one subset. A binary predicate defines the subset so that the members of the first subset fulfill the predicate. The remaining elements are in the second subset.
\end{myTip}

C++ offers a few functions for dealing with partitions. They all need a unary predicate pre. std::partition and std::stable\_partition partition a range and return the partition point. With std::partition\_point, you can get the partition point of a partition. Afterward, you can check the partition with std::is\_partitioned or copy it with std::partition\_copy.

Checks if the range is partitioned:

\begin{cpp}
bool is_partitioned(InpIt first, InpIt last, UnPre pre)
bool is_partitioned(ExePol pol, FwdIt first, FwdIt last, UnPre pre)
\end{cpp}

Partitions the range:

\begin{cpp}
FwdIt partition(FwdIt first, FwdIt last, UnPre pre)
FwdIt partition(ExePol pol, FwdIt first, FwdIt last, UnPre pre)
\end{cpp}

Partitions the range (stable):

\begin{cpp}
BiIt stable_partition(FwdIt first, FwdIt last, UnPre pre)
BiIt stable_partition(ExePol pol, FwdIt first, FwdIt last, UnPre pre)
\end{cpp}

Copies a partition in two ranges:

\begin{cpp}
pair<OutIt1, OutIt2> partition_copy(InIt first, InIt last,
					 OutIt1 result_true, OutIt2 result_false, UnPre pre)
pair<FwdIt1, FwdIt2> partition_copy(ExePol pol, FwdIt1 first, FwdIt1 last,
					 FwdIt2 result_true, FwdIt3 result_false, UnPre pre)
\end{cpp}

Returns the partition point:

\begin{cpp}
FwdIt partition_point(FwdIt first, FwdIt last, UnPre pre)
\end{cpp}

A std::stable\_partition guarantees, contrary to a std::partition that the elements preserve their relative order. The returned iterator FwdIt and BiIt point to the second subset of the partition’s initial position. The pair std::pair<OutIt, OutIt> of the algorithm std::partition\_copy contains the end iterator of the subsets result\_true and result\_false. The behavior of std::partition\_point is undefined if the range is not partitioned.

\filename{Partition algorithms}

\begin{cpp}
// partition.cpp
...
#include <algorithm>
...
using namespace std;

bool isOdd(int i){ return (i%2) == 1; }
vector<int> vec{1, 4, 3, 4, 5, 6, 7, 3, 4, 5, 6, 0, 4,
				8, 4, 6, 6, 5, 8, 8, 3, 9, 3, 7, 6, 4, 8};
auto parPoint= partition(vec.begin(), vec.end(), isOdd);
for (auto v: vec) cout << v << " ";
					// 1 7 3 3 5 9 7 3 3 5 5 0 4 8 4 6 6 6 8 8 4 6 4 4 6 4 8

for (auto v= vec.begin(); v != parPoint; ++v) cout << *v << " ";
					// 1 7 3 3 5 9 7 3 3 5 5
for (auto v= parPoint; v != vec.end(); ++v) cout << *v << " ";
					// 4 8 4 6 6 6 8 8 4 6 4 4 6 4 8

cout << is_partitioned(vec.begin(), vec.end(), isOdd); // true
list<int> le;
list<int> ri;
partition_copy(vec.begin(), vec.end(), back_inserter(le), back_inserter(ri),
			   [](int i) { return i < 5; });
for (auto v: le) cout << v << ""; // 1 3 3 3 3 0 4 4 4 4 4 4
for (auto v: ri) cout << v << ""; // 7 5 9 7 5 5 8 6 6 6 8 8 6 6 8
\end{cpp}



















