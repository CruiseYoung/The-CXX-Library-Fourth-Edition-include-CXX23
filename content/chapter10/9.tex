二分搜索算法利用了范围已经排序的事实。要搜索一个元素,使用std::binary\_search。使用std::lower\_bound,将获得不小于给定值的第一个元素的迭代器。使用std::upper\_bound,将获得第一个元素的迭代器,该元素大于给定值。std::equal\_range结合了这两种算法。

若容器有n个元素,则平均需要进行log2(n)次比较,二进制搜索要求使用与对容器排序相同的比较条件。默认的比较条件是std::less,可以调整它。排序标准必须遵守严格的弱排序,若不是,则程序有未定义行为。

无序关联容器的成员函数通常更快。

在范围内搜索元素val:

\begin{cpp}
bool binary_search(FwdIt first, FwdIt last, const T& val)
bool binary_search(FwdIt first, FwdIt last, const T& val, BiPre pre)
\end{cpp}

返回范围中第一个元素的位置,不小于val:

\begin{cpp}
FwdIt lower_bound(FwdIt first, FwdIt last, const T& val)
FwdIt lower_bound(FwdIt first, FwdIt last, const T& val, BiPre pre)
\end{cpp}

返回范围中第一个元素的位置,大于val:

\begin{cpp}
FwdIt upper_bound(FwdIt first, FwdIt last, const T& val)
FwdIt upper_bound(FwdIt first, FwdIt last, const T& val, BiPre pre)
\end{cpp}

返回元素val的std::lower\_bound和std::upper\_bound对:

\begin{cpp}
pair<FwdIt, FwdIt> equal_range(FwdIt first, FwdIt last, const T& val)
pair<FwdIt, FwdIt> equal_range(FwdIt first, FwdIt last, const T& val, BiPre pre)
\end{cpp}

最后,下面是代码段。

\filename{二进制搜索算法}

\begin{cpp}
// binarySearch.cpp
...
#include <algorithm>
...
using namespace std;

bool isLessAbs(int a, int b){
	return abs(a) < abs(b);
}
vector<int> vec{-3, 0, -3, 2, -3, 5, -3, 7, -0, 6, -3, 5,
				-6, 8, 9, 0, 8, 7, -7, 8, 9, -6, 3, -3, 2};

sort(vec.begin(), vec.end(), isLessAbs);
for (auto v: vec) cout << v << " ";
	// 0 0 0 2 2 -3 -3 -3 -3 -3 3 -3 5 5 -6 -6 6 7 -7 7 8 8 8 9 9
cout << binary_search(vec.begin(), vec.end(), -5, isLessAbs); // true
cout << binary_search(vec.begin(), vec.end(), 5, isLessAbs); // true

auto pair= equal_range(vec.begin(), vec.end(), 3, isLessAbs);
cout << distance(vec.begin(), pair.first); // 5
cout << distance(vec.begin(), pair.second)-1; // 11

for (auto threeIt= pair.first;threeIt != pair.second; ++threeIt)
	cout << *threeIt << " "; // -3 -3 -3 -3 -3 3 -3
\end{cpp}























