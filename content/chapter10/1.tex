You must keep a few rules in your head to use the algorithms.

The algorithms are defined in various headers.

<algorithm>: Contains the general algorithms.
<numeric> : Contains the numeric algorithms.

Many algorithms have the name suffix \_if and \_copy.

\_if: The algorithm can be parametrized by a predicate.
\_copy: The algorithm copies its elements in another range.

Algorithms like auto num= std::count(InpIt first, InpIt last, const T\& val) return the number of elements equal to val. num is of type iterator\_traits<InpIt>::difference\_type. You have the guarantee that num is sufficient to hold the result. Because of the automatic return type deduction with auto, the compiler will give you the right types.

\begin{myTip}{If the container uses an additional range, it has to be valid}

The algorithm std::copy\_if uses an iterator to the beginning of its destination range. This destination range has to be valid.
\end{myTip}

\begin{myNotic}{Naming conventions for the algorithms}

I use a few naming conventions for the type of arguments and the algorithms’ return type to make them easier to read.

\begin{center}
Signature of the algorithms
\end{center}

% Please add the following required packages to your document preamble:
% \usepackage{longtable}
% Note: It may be necessary to compile the document several times to get a multi-page table to line up properly
\begin{longtable}[c]{|l|l|}
\hline
\textbf{Name} & \textbf{Description}   \\ \hline
\endfirsthead
%
\endhead
%
InIt          & Input iterator         \\ \hline
FedIt         & Forward iterator       \\ \hline
BiIt          & Bidirectional iterator \\ \hline
UnFunc        & Unary callable         \\ \hline
BiFunc        & Binary callable        \\ \hline
UnPre         & Unary predicate        \\ \hline
BiPre         & Binary predicate       \\ \hline
Search  & The searcher encapsulates the search algorithm.                                    \\ \hline
ValType & From the input range, automatically deduced value type.                            \\ \hline
Num     & typename std::iterator\_traits\textless{}ForwardIt\textgreater{}::difference\_type \\ \hline
ExePol        & Execution policy       \\ \hline
\end{longtable}

\end{myNotic}

\mySamllsection{Iterators are the Glue}

Iterators define the range of the container on which the algorithms work. They describe a half-open range. In a half-open range, the begin iterator points to the beginning, and the end iterator points to one position after the range.

The iterators can be categorized based on their capabilities. See the Categories section of the Iterators chapter. The algorithms provide conditions to the iterators. Like in the case of std::rotate, most of the time, a forward iterator is sufficient. But that doesn’t hold for std::reverse. std::reverse requires a bidirectional iterator.









