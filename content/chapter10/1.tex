使用算法时,你必须牢记一些规则。

算法在不同的头文件中定义。

<algorithm>: 包含通用算法。
<numeric> : 包含数值算法。

许多算法的名称后缀都是\_if和\_copy。

\_if: 该算法可以通过谓词参数化。
\_copy: 该算法将其元素复制到另一个范围。

像auto num= std::count(InpIt first, InpIt last, const T\& val)这样的算法返回与val相等的元素数量。num类型为iterator\_traits<InpIt>::difference\_type,可以保证num足以保存结果。由于auto的自动返回类型推导,编译器将提供正确的类型。

\begin{myTip}{若容器使用了额外的范围,则其必须是有效的}

算法std::copy\_if使用迭代器到目标范围的开始。此目标范围必须有效。
\end{myTip}

\begin{myNotic}{算法的命名约定}

我对参数的类型和算法的返回类型使用了一些命名约定,以使它们更容易阅读。

\begin{center}
算法的签名
\end{center}

% Please add the following required packages to your document preamble:
% \usepackage{longtable}
% Note: It may be necessary to compile the document several times to get a multi-page table to line up properly
\begin{longtable}[c]{|l|l|}
\hline
\textbf{名称} & \textbf{描述}   \\ \hline
\endfirsthead
%
\endhead
%
InIt          & 输入迭代器         \\ \hline
FedIt         & 前向迭代器       \\ \hline
BiIt          & 双向迭代器 \\ \hline
UnFunc        & 一元可调用         \\ \hline
BiFunc        & 二元可调用        \\ \hline
UnPre         & 一元谓词        \\ \hline
BiPre         & 二元谓词      \\ \hline
Search  & \href{https://en.cppreference.com/w/cpp/algorithm/search}{searcher}封装了搜索算法。                                    \\ \hline
ValType & 从输入范围,自动推导出值类型。                            \\ \hline
Num     & \href{https://en.cppreference.com/w/cpp/iterator/iterator_traits}{typename std::iterator\_traits\textless{}ForwardIt\textgreater{}::difference\_type} \\ \hline
ExePol        & 执行策略       \\ \hline
\end{longtable}

\end{myNotic}











