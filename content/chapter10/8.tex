You can sort a range with std::sort or std::stable\_sort or sort until a position with std::partial\_sort. In addition, std::partial\_sort\_copy copies the partially sorted range. With std::nth\_element, you can assign an element the sorted position in the range. You can check with std::is\_sorted if a range is sorted. If you want to know until which position a range is sorted, use std::is\_sorted\_until.

Per default, the predefined function object std::less is used as a sorting criterion. However, you can use your sorting criterion. This has to obey the strict weak ordering.

Sorts the elements in the range:

\begin{cpp}
void sort(RaIt first, RaIt last)
void sort(ExePol pol, RaIt first, RaIt last)

void sort(RaIt first, RaIt last, BiPre pre)
void sort(ExePol pol, RaIt first, RaIt last, BiPre pre)
\end{cpp}

Sorts the elements in the range (stable):

\begin{cpp}
void stable_sort(RaIt first, RaIt last)
void stable_sort(ExePol pol, RaIt first, RaIt last)

void stable_sort(RaIt first, RaIt last, BiPre pre)
void stable_sort(ExePol pol, RaIt first, RaIt last, BiPre pre)
\end{cpp}

Sorts partially the elements in the range until middle:

\begin{cpp}
void partial_sort(RaIt first, RaIt middle, RaIt last)
void partial_sort(ExePol pol, RaIt first, RaIt middle, RaIt last)

void partial_sort(RaIt first, RaIt middle, RaIt last, BiPre pre)
void partial_sort(ExePol pol, RaIt first, RaIt middle, RaIt last, BiPre pre)
\end{cpp}

Sorts partially the elements in the range and copies them in the destination ranges result\_first and result\_last:

\begin{cpp}
RaIt partial_sort_copy(InIt first, InIt last,
					   RaIt result_first, RaIt result_last)
RaIt partial_sort_copy(ExePol pol, FwdIt first, FwdIt last,
					   RaIt result_first, RaIt result_last)

RaIt partial_sort_copy(InIt first, InIt last,
					   RaIt result_first, RaIt result_last, BiPre pre)
RaIt partial_sort_copy(ExePol pol, FwdIt first, FwdIt last,
					   RaIt result_first, RaIt result_last, BiPre pre)
\end{cpp}

Checks if a range is sorted:

\begin{cpp}
bool is_sorted(FwdIt first, FwdIt last)
bool is_sorted(ExePol pol, FwdIt first, FwdIt last)

bool is_sorted(FwdIt first, FwdIt last, BiPre pre)
bool is_sorted(ExePol pol, FwdIt first, FwdIt last, BiPre pre)
\end{cpp}

Returns the position to the first element that doesn’t satisfy the sorting criterion:

\begin{cpp}
FwdIt is_sorted_until(FwdIt first, FwdIt last)
FwdIt is_sorted_until(ExePol pol, FwdIt first, FwdIt last)

FwdIt is_sorted_until(FwdIt first, FwdIt last, BiPre pre)
FwdIt is_sorted_until(ExePol pol, FwdIt first, FwdIt last, BiPre pre)
\end{cpp}

Reorders the range so that the n-th element has the right (sorted) position:

\begin{cpp}
void nth_element(RaIt first, RaIt nth, RaIt last)
void nth_element(ExePol pol, RaIt first, RaIt nth, RaIt last)

void nth_element(RaIt first, RaIt nth, RaIt last, BiPre pre)
void nth_element(ExePol pol, RaIt first, RaIt nth, RaIt last, BiPre pre)
\end{cpp}

Here is a code snippet.

\filename{Sort algorithms}

\begin{cpp}
// sort.cpp
...
#include <algorithm>
...

std::string str{"RUdAjdDkaACsdfjwldXmnEiVSEZTiepfgOIkue"};
std::cout << std::is_sorted(str.begin(), str.end()); // false

std::partial_sort(str.begin(), str.begin()+30, str.end());
std::cout << str; // AACDEEIORSTUVXZaddddeeffgiijjkwspnmluk

auto sortUntil= std::is_sorted_until(str.begin(), str.end());
std::cout << *sortUntil; // s
for (auto charIt= str.begin(); charIt != sortUntil; ++charIt)
	std::cout << *charIt; // AACDEEIORSTUVXZaddddeeffgiijjkw

std::vector<int> vec{1, 0, 4, 3, 5};
auto vecIt= vec.begin();

while(vecIt != vec.end()){
	std::nth_element(vec.begin(), vecIt++, vec.end());
	std::cout << std::distance(vec.begin(), vecIt) << "-th ";
	for (auto v: vec) std::cout << v << "/";
}
// 1-th 01435/2-th 01435/3-th 10345/4-th 30145/5-th 10345
\end{cpp}












