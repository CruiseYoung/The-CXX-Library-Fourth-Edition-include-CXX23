可以使用std::min\_element、std::max\_element和std::minmax\_element算法来确定范围的最小、最大、最小和最大对。每个算法都可以用二进制谓词调用。此外,C++17允许在一对边界值之间夹紧一个值。

返回范围的最小元素:

\begin{cpp}
constexpr FwdIt min_element(FwdIt first, FwdIt last)
FwdIt min_element(ExePol pol, FwdIt first, FwdIt last)

constexpr FwdIt min_element(FwdIt first, FwdIt last, BinPre pre)
FwdIt min_element(ExePol pol, FwdIt first, FwdIt last, BinPre pre)
\end{cpp}

返回范围的最大元素:

\begin{cpp}
constexpr FwdIt max_element(FwdIt first, FwdIt last)
FwdIt max_element(ExePol pol, FwdIt first, FwdIt last)

constexpr FwdIt max_element(FwdIt first, FwdIt last, BinPre pre)
FwdIt max_element(ExePol pol, FwdIt first, FwdIt last, BinPre pre)
\end{cpp}

返回范围的std::min\_element和std::max\_element对:

\begin{cpp}
constexpr pair<FwdIt, FwdIt> minmax_element(FwdIt first, FwdIt last)
pair<FwdIt, FwdIt> minmax_element(ExePol pol, FwdIt first, FwdIt last)

constexpr pair<FwdIt, FwdIt> minmax_element(FwdIt first, FwdIt last, BinPre pre)
pair<FwdIt, FwdIt> minmax_element(ExePol pol, FwdIt first, FwdIt last,
								  BinPre pre)
\end{cpp}

若范围有多个最小或最大元素,则返回第一个。

\filename{最小和最大算法}

\begin{cpp}
// minMax.cpp
...
#include <algorithm>
...

int toInt(const std::string& s){
	std::stringstream buff;
	buff.str("");
	buff << s;
	int value;
	buff >> value;
	return value;
}

std::vector<std::string> myStrings{"94", "5", "39", "-4", "-49", "1001", "-77",
							       "23", "0", "84", "59", "96", "6", "-94"};
auto str= std::minmax_element(myStrings.begin(), myStrings.end());

std::cout << *str.first << ":" << *str.second; // -4:96

auto asInt= std::minmax_element(myStrings.begin(), myStrings.end(),
			[](std::string a, std::string b){ return toInt(a) < toInt(b); });
std::cout << *asInt.first << ":" << *asInt.second; // -94:1001
\end{cpp}

std::clamp在一对边界值之间夹住一个值:

\begin{cpp}
constexpr const T& clamp(const T& v, const T& lo, const T& hi);
constexpr const T& clamp(const T& v, const T& lo, const T& hi, BinPre pre);
\end{cpp}

若lo小于v且v小于hi,则返回对v的引用。否则,返回对lo或hi的引用。默认情况下,<用作比较标准,也可以提供一个二进制谓词pre。

\filename{箝位一个值}

\begin{cpp}
// clamp.cpp
...
#include <algorithm>
...

auto values = {1, 2, 3, 4, 5, 6, 7};
for (auto v: values) std::cout << v << ' '; // 1 2 3 4 5 6 7

auto lo = 3;
auto hi = 6;
for (auto v: values) std::cout << std::clamp(v, lo, hi) << ' '; // 3 3 3 4 5 6 6
\end{cpp}





















