不可修改算法是用于搜索和计数元素的算法,也可以检查范围上的属性、比较范围或在范围内搜索范围。

\mySamllsection{搜索元素}

可以用三种不同的方式搜索元素。

返回一个范围内的元素:

\begin{cpp}
InpIt find(InpIt first, InpI last, const T& val)
InpIt find(ExePol pol, FwdIt first, FwdIt last, const T& val)

InpIt find_if(InpIt first, InpIt last, UnPred pred)
InpIt find_if(ExePol pol, FwdIt first, FwdIt last, UnPred pred)

InpIt find_if_not(InpIt first, InpIt last, UnPred pre)
InpIt find_if_not(ExePol pol, FwdIt first, FwdIt last, UnPred pre)
\end{cpp}

返回一个范围的第一个元素:

\begin{cpp}
FwdIt1 find_first_of(InpIt1 first1, InpIt1 last1,
					 FwdIt2 first2, FwdIt2 last2)
FwdIt1 find_first_of(ExePol pol, FwdIt1 first1, FwdIt1 last1,
					 FwdIt2 first2, FwdIt2 last2)
					 
FwdIt1 find_first_of(InpIt1 first1, InpIt1 last1,
					 FwdIt2 first2, FwdIt2 last2, BiPre pre)
FwdIt1 find_first_of(ExePol pol, FwdIt1 first1, FwdIt1 last1,
					 FwdIt2 first2, FwdIt2 last2, BiPre pre)
\end{cpp}

返回范围内相同的相邻元素:

\begin{cpp}
FwdIt adjacent_find(FwdIt first, FwdIt last)
FwdIt adjacent_find(ExePol pol, FwdIt first, FwdIt last)

FwdIt adjacent_find(FwdIt first, FwdI last, BiPre pre)
FwdIt adjacent_find(ExePol pol, FwdIt first, FwdI last, BiPre pre)
\end{cpp}

这些算法需要输入或前向迭代器作为参数,并在成功找到元素时返回一个迭代器。若搜索不成功,则返回结束迭代器。

\filename{std::find, std::find\_if, std::find\_if\_not, std::find\_of 和 std::adjacent\_fint}

\begin{cpp}
// find.cpp
...
#include <algorithm>
...
using namespace std;

bool isVowel(char c){
	string myVowels{"aeiouäöü"};
	set<char> vowels(myVowels.begin(), myVowels.end());
	return (vowels.find(c) != vowels.end());
}

list<char> myCha{'a', 'b', 'c', 'd', 'e', 'f', 'g', 'h', 'i', 'j'};
int cha[]= {'A', 'B', 'C'};

cout << *find(myCha.begin(), myCha.end(), 'g'); // g
cout << *find_if(myCha.begin(), myCha.end(), isVowel); // a
cout << *find_if_not(myCha.begin(), myCha.end(), isVowel); // b

auto iter= find_first_of(myCha.begin(), myCha.end(), cha, cha + 3);
if (iter == myCha.end()) cout << "None of A, B or C."; // None of A, B or C.
auto iter2= find_first_of(myCha.begin(), myCha.end(), cha, cha+3,
				[](char a, char b){ return toupper(a) == toupper(b); });
				
if (iter2 != myCha.end()) cout << *iter2; // a
auto iter3= adjacent_find(myCha.begin(), myCha.end());
if (iter3 == myCha.end()) cout << "No same adjacent chars.";
										// No same adjacent chars.
										
auto iter4= adjacent_find(myCha.begin(), myCha.end(),
				[](char a, char b){ return isVowel(a) == isVowel(b); });
if (iter4 != myCha.end()) cout << *iter4; // b
\end{cpp}

\mySamllsection{元素计数}

可以使用STL对元素进行计数,无论是否使用谓词。

返回元素的个数:

\begin{cpp}
Num count(InpIt first, InpIt last, const T& val)
Num count(ExePol pol, FwdIt first, FwdIt last, const T& val)

Num count_if(InpIt first, InpIt last, UnPred pre)
Num count_if(ExePol pol, FwdIt first, FwdIt last, UnPred pre)
\end{cpp}

计数算法接受输入迭代器作为参数,并返回与val或谓词匹配的元素个数。

\filename{std::count 和 std::count\_if}

\begin{cpp}
// count.cpp
...
#include <algorithm>
...
std::string str{"abcdabAAAaefaBqeaBCQEaadsfdewAAQAaafbd"};
std::cout << std::count(str.begin(), str.end(), 'a'); // 9
std::cout << std::count_if(str.begin(), str.end(),
							[](char a){ return std::isupper(a); }); // 12
\end{cpp}


\mySamllsection{范围的检查条件}

三个函数std::all\_of、std::any\_of和std::none\_of可回答这个问题,若一个范围的所有、至少一个或没有元素满足条件。这些函数需要作为参数输入迭代器和一元谓词,并返回一个布尔值。

检查范围内的所有元素是否满足条件:

\begin{cpp}
bool all_of(InpIt first, InpIt last, UnPre pre)
bool all_of(ExePol pol, FwdIt first, FwdIt last, UnPre pre)
\end{cpp}

检查范围中是否至少有一个元素满足条件:

\begin{cpp}
bool any_of(InpIt first, InpIt last, UnPre pre)
bool any_of(ExePol pol, FwdIt first, FwdIt last, UnPre pre)
\end{cpp}

检查范围内是否没有元素满足条件:

\begin{cpp}
bool none_of(InpIt first, InpIt last, UnPre pre)
bool none_of(ExePol pol, FwdIt first, FwdIt last, UnPre pre)
\end{cpp}

如前所述,示例如下:

\filename{std::all\_of, std::any\_of 和 std::none\_of}

\begin{cpp}
// allAnyNone.cpp
...
#include <algorithm>
...
auto even= [](int i){ return i%2; };
std::vector<int> myVec{1, 2, 3, 4, 5, 6, 7, 8, 9};
std::cout << std::any_of(myVec.begin(), myVec.end(), even); // true
std::cout << std::all_of(myVec.begin(), myVec.end(), even); // false
std::cout << std::none_of(myVec.begin(), myVec.end(), even); // false
\end{cpp}

\mySamllsection{范围比较}

使用std::equal,可以比较相等的范围。使用std::lexicographical\_compare、std::lexicographical\_compare\_three\_way和std::mismatch,可以用于较小的范围。

检查两个范围是否相等:

\begin{cpp}
bool equal(InpIt first1, InpIt last1, InpIt first2)
bool equal(ExePol pol, FwdIt first1, FwdIt last1, FwdIt first2)

bool equal(InpIt first1, InpIt last1, InpIt first2, BiPre pred)
bool equal(ExePol pol, FwdIt first1, FwdIt last1, FwdIt first2, BiPre pred)

bool equal(InpIt first1, InpIt last1,
		   InpIt first2, InpIt last2)
bool equal(ExePol pol, FwdIt first1, FwdIt last1,
		   FwdIt first2, FwdIt last2)
		   
bool equal(InpIt first1, InpIt last1,
		   InpIt first2, InpIt last2, BiPre pred)
bool equal(ExePol pol, FwdIt first1, FwdIt last1,
		   FwdIt first2, FwdIt last2, BiPre pred)
\end{cpp}

检查第一个范围是否小于第二个范围:

\begin{cpp}
bool lexicographical_compare(InpIt first1, InpIt last1,
							 InpIt first2, InpIt last2)
bool lexicographical_compare(ExePol pol, FwdIt first1, FwdIt last1,
							 FwdIt first2, FwdIt last2)
							 
bool lexicographical_compare(InpIt first1, InpIt last1,
							 InpIt first2, InpIt last2, BiPre pred)
bool lexicographical_compare(ExePol pol, FwdIt first1, FwdIt last1,
							 FwdIt first2, FwdIt last2, BiPre pred)
\end{cpp}

检查第一个范围是否小于第二个范围。应用\href{https://en.cppreference.com/w/cpp/language/operator_comparison}{三路比较},可返回最适用比较类别的类型。

\begin{cpp}
bool lexicographical_compare_three_way(InpIt first1, InpIt last1,
									   InpIt first2, InpIt last2)
bool lexicographical_compare_three_way(InpIt first1, InpIt last1,
									   InpIt first2, InpIt last2,
									   BiPre pre)
\end{cpp}

查找两个范围不相等的第一个位置:

\begin{cpp}
pair<InpIt, InpIt> mismatch(InpIt first1, InpIt last1,
							InpIt first2)
pair<InpIt, InpIt> mismatch(ExePol pol, FwdIt first1, FwdIt last1,
							FwdIt first2)
							
pair<InpIt, InpIt> mismatch(InpIt first1, InpIt last1,
							InpIt first2, BiPre pred)
pair<InpIt, InpIt> mismatch(ExePol pol, FwdIt first1, FwdIt last2,
							FwdIt first2, BiPre pred)
							
pair<InpIt, InpIt> mismatch(InpIt first1, InpIt last1,
							InpIt first2, InpIt last2)
pair<InpIt, InpIt> mismatch(ExePol pol, FwdIt first1, FwdIt last1,
							FwdIt first2, FwdIt last2)
							
pair<InpIt, InpIt> mismatch(InpIt first1, InpIt last1,
							InpIt first2, InpIt last2, BiPre pred)
pair<InpIt, InpIt> mismatch(ExePol pol, FwdIt first1, FwdIt last1,
							FwdIt first2, FwdIt last2, BiPre pred)
\end{cpp}

这些算法接受输入迭代器和最终的二元谓词。std::mismatch返回一对输入迭代器。pa.first保存第一个不相等元素的输入迭代器。pa.second保存第二个范围对应的输入迭代器。若两个范围相同,则得到两个end迭代器。

\filename{std::equal, std::lexicographical\_compare 和 std::mismatch}

\begin{cpp}
// equalLexicographicalMismatch.cpp
...
#include <algorithm>
...
using namespace std;

string str1{"Only For Testing Purpose." };
string str2{"only for testing purpose." };
cout << equal(str1.begin(), str1.end(), str2.begin()); // false
cout << equal(str1.begin(), str1.end(), str2.begin(),
			  [](char c1, char c2){ return toupper(c1) == toupper(c2);} );
																// true

str1= {"Only for testing Purpose." };
str2= {"Only for testing purpose." };
auto pair= mismatch(str1.begin(), str1.end(), str2.begin());
if (pair.first != str1.end()){
	cout << distance(str1.begin(), pair.first)
		 << "at (" << *pair.first << "," << *pair.second << ")"; // 17 at (P,p)
}

auto pair2= mismatch(str1.begin(), str1.end(), str2.begin(),
					 [](char c1, char c2){ return toupper(c1) == toupper(c2); });
if (pair2.first == str1.end()){
	cout << "str1 and str2 are equal"; // str1 and str2 are equal
}
\end{cpp}

\mySamllsection{在范围中搜索范围}

std::search从开头搜索另一个范围,std::find\_end从结尾搜索std::search\_n搜索范围内连续的n个元素。

所有算法都采用前向迭代器,可以用二进制谓词参数化,若搜索不成功,则返回第一个范围的结束迭代器。

在第一个区域中搜索第二个区域并返回位置,从开头开始:

\begin{cpp}
FwdIt1 search(FwdIt1 first1, FwdIt1 last1, FwdIt2 first2, FwdIt2 last2)
FwdIt1 search(ExePol pol, FwdIt1 first1, FwdIt1 last1,
			  FwdIt2 first2, FwdIt2 last2)
			  
FwdIt1 search(FwdIt1 first1, FwdIt1 last1,
			  FwdIt2 first2, FwdIt2 last2, BiPre pre)
FwdIt1 search(ExePol pol, FwdIt1 first1, FwdIt1 last1,
			  FwdIt2 first2, FwdIt2 last2, BiPre pre)
			  
FwdIt1 search(FwdIt1 first, FwdIt last1, Search search)
\end{cpp}

在第一个范围中搜索第二个范围并返回位置,从结尾开始:

\begin{cpp}
FwdIt1 find_end(FwdIt1 first1, FwdIt1 last1, FwdIt2 first2 FwdIt2 last2)
FwdIt1 find_end(ExePol pol, FwdIt1 first1, FwdIt1 last1,
				FwdIt2 first2 FwdIt2 last2)
				
FwdIt1 find_end(FwdIt1 first1, FwdIt1 last1, FwdIt2 first2, FwdIt2 last2,
				BiPre pre)
FwdIt1 find_end(ExePol pol, FwdIt1 first1, FwdIt1 last1,
				FwdIt2 first2, FwdIt2 last2, BiPre pre)
\end{cpp}

搜索计算第一个范围内的连续值:

\begin{cpp}
FwdIt search_n(FwdIt first, FwdIt last, Size count, const T& value)
FwdIt search_n(ExePol pol, FwdIt first, FwdIt last, Size count, const T& value)

FwdIt search_n(FwdIt first, FwdIt last, Size count, const T& value, BiPre pre)
FwdIt search_n(ExePol pol, FwdIt first,
			   FwdIt last, Size count, const T& value, BiPre pre)
\end{cpp}

\begin{myWarning}{特殊算法search\_n}
	
算法FwdIt search\_n(FwdIt first, FwdIt last, Size count, const T\& value, BiPre pre)非常特殊。二元谓词BiPre将范围的值用作第一个参数,将value用作第二个参数。
\end{myWarning}

\filename{std::find, std::find\_end 和 std::search\_n}

\begin{cpp}
// search.cpp
...
#include <algorithm>
...
using std::search;

std::array<int, 10> arr1{0, 1, 2, 3, 4, 5, 6, 7, 8, 9};
std::array<int, 5> arr2{3, 4, -5, 6, 7};

auto fwdIt= search(arr1.begin(), arr1.end(), arr2.begin(), arr2.end());
if (fwdIt == arr1.end()) std::cout << "arr2 not in arr1."; // arr2 not in arr1.

auto fwdIt2= search(arr1.begin(), arr1.end(), arr2.begin(), arr2.end(),
					[](int a, int b){ return std::abs(a) == std::abs(b); });
if (fwdIt2 != arr1.end()) std::cout << "arr2 at position "
					<< std::distance(arr1.begin(), fwdIt2) << " in arr1.";
													// arr2 at position 3 in arr1.
\end{cpp}









