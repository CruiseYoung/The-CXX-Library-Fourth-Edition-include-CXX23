std::for\_each对其范围内的每个元素应用一元可调用对象,输入迭代器可提供范围。

\begin{cpp}
UnFunc std::for_each(InpIt first, InpIt second, UnFunc func)
void std::for_each(ExePol pol, FwdIt first, FwdIt second, UnFunc func)
\end{cpp}

在没有显式执行策略的情况下使用std::for\_each是一个特殊的算法,因为它返回其可调用的参数。若使用函数对象调用std::for\_each,则可以将函数调用的结果直接存储在函数对象中。

\begin{cpp}
InpIt std::for_each_n(InpIt first, Size n, UnFunc func) FwdIt std::for_each_n(ExePol pol, FwdIt first, Size n, UnFunc func)
\end{cpp}

std::for\_each\_n是C++17的新特性,并对其范围的前n个元素应用一元可调用对象,输入迭代器和大小提供了范围。

\filename{std::for\_each}

\begin{cpp}
// forEach.cpp
...
#include <algorithm>
...
template <typename T>
class ContInfo{
public:
	void operator()(T t){
		num++;
		sum+= t;
	}
	int getSum() const{ return sum; }
	int getSize() const{ return num; }
	double getMean() const{
		return static_cast<double>(sum)/static_cast<double>(num);
	}
private:
	T sum{0};
	int num{0};
};

std::vector<double> myVec{1.1, 2.2, 3.3, 4.4, 5.5, 6.6, 7.7, 8.8, 9.9};
auto vecInfo= std::for_each(myVec.begin(), myVec.end(), ContInfo<double>());

std::cout << vecInfo.getSum() << '\n'; // 49
std::cout << vecInfo.getSize() << '\n'; // 9
std::cout << vecInfo.getMean() << '\n'; // 5.5

std::array<int, 100> myArr{1, 2, 3, 4, 5, 6, 7, 8, 9, 10};
auto arrInfo= std::for_each(myArr.begin(), myArr.end(), ContInfo<int>());

std::cout << arrInfo.getSum() << '\n'; // 55
std::cout << arrInfo.getSize() << '\n'; // 100
std::cout << arrInfo.getMean() << '\n'; // 0.55
\end{cpp}

























