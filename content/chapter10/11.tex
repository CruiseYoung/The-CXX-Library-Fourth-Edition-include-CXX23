
\myGraphic{0.4}{content/chapter10/images/4.jpg}{}

\begin{myTip}{什么是堆?}
堆是一个二叉搜索树,其中父元素总是比子元素大。堆树针对元素的高效排序进行了优化。
\end{myTip}

可以用std::make\_heap创建一个堆,使用std::push\_heap向堆上推送新元素,也可以使用std::pop\_heap从堆中取出最大的元素。这两种操作都尊重堆特征。std::push\_heap移动堆上范围的最后一个元素,std::pop\_heap将堆中最大的元素移动到范围内的最后一个位置。可以使用std::is\_heap检查一个范围是否为堆,也可以使用std::is\_heap\_until来确定该范围在哪个位置是堆。std::sort\_heap可对堆进行排序。

堆算法要求范围和算法使用相同的排序标准。若不是,则程序具有未定义行为。默认情况下,使用预定义的排序条件std::less。若使用排序标准,则必须遵守严格的弱排序。若不是,则程序具有未定义行为。

使用范围创建一个堆:

\begin{cpp}
void make_heap(RaIt first, RaIt last)
void make_heap(RaIt first, RaIt last, BiPre pre)
\end{cpp}

检查范围是否为堆:

\begin{cpp}
bool is_heap(RaIt first, RaIt last)
bool is_heap(ExePol pol, RaIt first, RaIt last)

bool is_heap(RaIt first, RaIt last, BiPre pre)
bool is_heap(ExePol pol, RaIt first, RaIt last, BiPre pre)
\end{cpp}

确定在哪个位置之前范围是堆:

\begin{cpp}
RaIt is_heap_until(RaIt first, RaIt last)
RaIt is_heap_until(ExePol pol, RaIt first, RaIt last)

RaIt is_heap_until(RaIt first, RaIt last, BiPre pre)
RaIt is_heap_until(ExePol pol, RaIt first, RaIt last, BiPre pre)
\end{cpp}

对堆进行排序:

\begin{cpp}
void sort_heap(RaIt first, RaIt last)
void sort_heap(RaIt first, RaIt last, BiPre pre)
\end{cpp}

将范围的最后一个元素压入堆中,[first, last-1)必须是堆。

\begin{cpp}
void push_heap(RaIt first, RaIt last)
void push_heap(RaIt first, RaIt last, BiPre pre)
\end{cpp}

从堆中移除最大的元素,并将其放在范围的末尾:

\begin{cpp}
void pop_heap(RaIt first, RaIt last)
void pop_heap(RaIt first, RaIt last, BiPre pre)
\end{cpp}

使用std::pop\_heap,可以从堆中删除最大的元素,最大的元素是范围的最后一个元素。要从堆h中删除元素,使用h.pop\_back。

\filename{堆算法}

\begin{cpp}
// heap.cpp
...
#include <algorithm>
...

std::vector<int> vec{4, 3, 2, 1, 5, 6, 7, 9, 10};
std::make_heap(vec.begin(), vec.end());
for (auto v: vec) std::cout << v << " "; // 10 9 7 4 5 6 2 3 1
std::cout << std::is_heap(vec.begin(), vec.end()); // true

vec.push_back(100);
std::cout << std::is_heap(vec.begin(), vec.end()); // false
std::cout << *std::is_heap_until(vec.begin(), vec.end()); // 100
for (auto v: vec) std::cout << v << " "; // 10 9 7 4 5 6 2 3 1 100

std::push_heap(vec.begin(), vec.end());
std::cout << std::is_heap(vec.begin(), vec.end()); // true
for (auto v: vec) std::cout << v << " "; // 100 10 7 4 9 6 2 3 1 5

std::pop_heap(vec.begin(), vec.end());
for (auto v: vec) std::cout << v << " "; // 10 9 7 4 5 6 2 3 1 100
std::cout << *std::is_heap_until(vec.begin(), vec.end()); // 100

vec.resize(vec.size()-1);
std::cout << std::is_heap(vec.begin(), vec.end()); // true
std::cout << vec.front() << '\n'; // 10
\end{cpp}













































