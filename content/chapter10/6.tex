
C++有许多修改元素和范围的算法。

\mySamllsection{复制元素和范围}

可以使用std::copy向前复制范围,使用std::copy\_backward向后复制范围,使用std::copy\_if有条件地复制范围。若要复制n个元素,可以使用std::copy\_n。

复制范围:

\begin{cpp}
OutIt copy(InpIt first, InpIt last, OutIt result)
FwdIt2 copy(ExePol pol, FwdIt first, FwdIt last, FowdIt2 result)
\end{cpp}

复制n个元素:

\begin{cpp}
OutIt copy_n(InpIt first, Size n, OutIt result)
FwdIt2 copy_n(ExePol pol, FwdIt first, Size n, FwdIt2 result)
\end{cpp}

复制依赖谓词pre的元素。

\begin{cpp}
OutIt copy_if(InpIt first, InpIt last, OutIt result, UnPre pre)
FwdIt2 copy_if(ExePol pol, FwdIt first, FwdIt last, FwdIt2 result, UnPre pre)
\end{cpp}

向后复制范围:

\begin{cpp}
BiIt copy_backward(BiIt first, BiIt last, BiIt result)
\end{cpp}

算法需要输入迭代器,并将其元素复制到result,返回一个结束迭代器到目标范围内。

\filename{复制元素和范围}

\begin{cpp}
// copy.cpp
...
#include <algorithm>
...

std::vector<int> myVec{0, 1, 2, 3, 4, 5, 6, 7, 9};
std::vector<int> myVec2(10);

std::copy_if(myVec.begin(), myVec.end(), myVec2.begin()+3,
			 [](int a){ return a%2; });

for (auto v: myVec2) std::cout << v << " "; // 0 0 0 1 3 5 7 9 00

std::string str{"abcdefghijklmnop"};
std::string str2{"---------------------"};

std::cout << str2; // ---------------------
std::copy_backward(str.begin(), str.end(), str2.end());
std::cout << str2; // -----abcdefghijklmnop
std::cout << str; // abcdefghijklmnop

std::copy_backward(str.begin(), str.begin() + 5, str.end());
std::cout << str; // abcdefghijkabcde
\end{cpp}

\mySamllsection{替换元素和范围}

可以使用std::replace、std::replace\_if、std::replace\_copy和std::replace\_copy\_if来替换范围中的元素。算法在两个方面有所不同。首先,算法需要谓词吗?其次,算法是否复制目标范围内的元素?

若旧元素的值为old,则用newValue替换范围内的旧元素。

\begin{cpp}
void replace(FwdIt first, FwdIt last, const T& old, const T& newValue)
void replace(ExePol pol, FwdIt first, FwdIt last, const T& old,
			 const T& newValue)
\end{cpp}

若旧元素满足谓词pred,则用newValue替换该范围的旧元素:

\begin{cpp}
void replace_if(FwdIt first, FwdIt last, UnPred pred, const T& newValue)
void replace_if(ExePol pol, FwdIt first, FwdIt last, UnPred pred,
			    const T& newValue)
\end{cpp}

若旧元素的值为old,则用newValue替换范围内的旧元素。将结果复制到result:

\begin{cpp}
OutIt replace_copy(InpIt first, InpIt last, OutIt result, const T& old,
				   const T& newValue)
FwdIt2 replace_copy(ExePol pol, FwdIt first, FwdIt last,
					FwdIt2 result, const T& old, const T& newValue)
\end{cpp}

若旧元素满足谓词pred,则用newValue替换该范围的旧元素。

将结果复制到result:

\begin{cpp}
OutIt replace_copy_if(InpIt first, InpIt last, OutIt result, UnPre pred,
					  const T& newValue)
FwdIt2 replace_copy_if(ExePol pol, FwdIt first, FwdIt last,
					   FwdIt2 result, UnPre pred, const T& newValue)
\end{cpp}

\filename{替换元素和范围}

\begin{cpp}
// replace.cpp
...
#include <algorithm>
...

std::string str{"Only for testing purpose." };
std::replace(str.begin(), str.end(), ' ', '1');
std::cout << str; // Only1for1testing1purpose.

std::replace_if(str.begin(), str.end(), [](char c){ return c == '1'; }, '2');
std::cout << str; // Only2for2testing2purpose.

std::string str2;
std::replace_copy(str.begin(), str.end(), std::back_inserter(str2), '2', '3');
std::cout << str2; // Only3for3testing3purpose.

std::string str3;
std::replace_copy_if(str2.begin(), str2.end(),
std::back_inserter(str3), [](char c){ return c == '3'; }, '4');
std::cout << str3; // Only4for4testing4purpose.
\end{cpp}

\mySamllsection{删除元素和范围}

std::remove、std::remove\_if、std::remove\_copy和std::remove\_copy\_if支持两种操作。一方面,从范围中删除带有或不带有谓词的元素。另一方面,将修改的结果复制到新的范围。

从范围中删除值为val的元素:

\begin{cpp}
FwdIt remove(FwdIt first, FwdIt last, const T& val)
FwdIt remove(ExePol pol, FwdIt first, FwdIt last, const T& val)
\end{cpp}

从范围中移除元素,满足谓词pred:

\begin{cpp}
FwdIt remove_if(FwdIt first, FwdIt last, UnPred pred)
FwdIt remove_if(ExePol pol, FwdIt first, FwdIt last, UnPred pred)
\end{cpp}

从范围中移除值为val的元素。将结果复制到result:

\begin{cpp}
OutIt remove_copy(InpIt first, InpIt last, OutIt result, const T& val)
FwdIt2 remove_copy(ExePol pol, FwdIt first, FwdIt last,
					FwdIt2 result, const T& val)
\end{cpp}

从范围中删除满足谓词pred的元素,将结果复制到result。

\begin{cpp}
OutIt remove_copy_if(InpIt first, InpIt last, OutIt result, UnPre pred)
FwdIt2 remove_copy_if(ExePol pol, FwdIt first, FwdIt last,
						FwdIt2 result, UnPre pred)
\end{cpp}

算法需要源范围的输入迭代器和目标范围的输出迭代器。作为结果,会返回目标范围的结束迭代器。

\begin{myWarning}{应用擦除-删除习语}
	
移除变量不会从范围中移除元素,值会返回范围的新逻辑结束位置。所以,必须使用擦除-删除习惯用法来调整容器的大小。

\filename{删除元素和范围}

\begin{cpp}
// remove.cpp
...
#include <algorithm>
...

std::vector<int> myVec{0, 1, 2, 3, 4, 5, 6, 7, 8, 9};

auto newIt= std::remove_if(myVec.begin(), myVec.end(),
							[](int a){ return a%2; });
for (auto v: myVec) std::cout << v << " "; // 0 2 4 6 8 5 6 7 8 9

myVec.erase(newIt, myVec.end());
for (auto v: myVec) std::cout << v << " "; // 0 2 4 6 8

std::string str{"Only for Testing Purpose." };
str.erase( std::remove_if(str.begin(), str.end(),
			[](char c){ return std::isupper(c); }, str.end() ) );
std::cout << str << '\n'; // nly for esting urpose.
\end{cpp}
\end{myWarning}

\mySamllsection{填充和创建范围}

可以用std::fill和std::fill\_n,可以使用std::generate和std::generate\_n生成新元素。

用元素填充一个范围:

\begin{cpp}
void fill(FwdIt first, FwdIt last, const T& val)
void fill(ExePol pol, FwdIt first, FwdIt last, const T& val)
\end{cpp}

用n个新元素填充一个范围:

\begin{cpp}
OutIt fill_n(OutIt first, Size n, const T& val)
FwdIt fill_n(ExePol pol, FwdIt first, Size n, const T& val)
\end{cpp}

使用生成器gen生成一个范围:

\begin{cpp}
void generate(FwdIt first, FwdIt last, Generator gen)
void generate(ExePol pol, FwdIt first, FwdIt last, Generator gen)
\end{cpp}

使用生成器gen生成一个范围的n个元素:

\begin{cpp}
OutIt generate_n(OutIt first, Size n, Generator gen)
FwdIt generate_n(ExePol pol, FwdIt first, Size n, Generator gen)
\end{cpp}

算法期望值val或Generator gen作为参数,Gen必须是一个不接受参数并返回新值的函数。std::fill\_n和std::generate\_n算法的返回值是一个输出迭代器,指向最后创建的元素。

\filename{填充和创建范围}

\begin{cpp}
// fillAndCreate.cpp
...
#include <algorithm>
...

int getNext(){
	static int next{0};
	return ++next;
}

std::vector<int> vec(10);
std::fill(vec.begin(), vec.end(), 2011);
for (auto v: vec) std::cout << v << " ";
						// 2011 2011 2011 2011 2011 2011 2011 2011 2011 2011

std::generate_n(vec.begin(), 5, getNext);
for (auto v: vec) std::cout << v << " ";
						// 1 2 3 4 5 2011 2011 2011 2011 2011
\end{cpp}

\mySamllsection{移动范围}

std::move可向前移动范围;std::move\_backward可向后移动范围。

向前移动范围:

\begin{cpp}
OutIt move(InpIt first, InpIt last, OutIt result)
FwdIt2 move(ExePol pol, FwdIt first, FwdIt last, Fwd2It result)
\end{cpp}

向后移动范围:

\begin{cpp}
BiIt move_backward(BiIt first, BiIt last, BiIt result)
\end{cpp}

这两种算法都需要一个目标迭代器result,将范围移动到该结果中。对于std::move算法,这是一个输出迭代器。对于std::move\_backward算法,这是一个双向迭代器。这些算法返回一个输出或一个双向迭代器,指向目标范围内的初始位置。

\begin{myWarning}{源范围可以修改}	
std::move和std::move\_backward应用移动语义,所以源范围是有效的,但之后不一定有相同的元素。
\end{myWarning}

\filename{移动范围}

\begin{cpp}
// move.cpp
...
#include <algorithm>
...

std::vector<int> myVec{0, 1, 2, 3, 4, 5, 6, 7, 9};
std::vector<int> myVec2(myVec.size());
std::move(myVec.begin(), myVec.end(), myVec2.begin());
for (auto v: myVec2) std::cout << v << " "; // 0 1 2 3 4 5 6 7 9 0

std::string str{"abcdefghijklmnop"};
std::string str2{"---------------------"};
std::move_backward(str.begin(), str.end(), str2.end());
std::cout << str2; // -----abcdefghijklmnop
\end{cpp}

\mySamllsection{交换范围}

std::swap和std::swap\_ranges可以交换对象和范围。

互换对象:

\begin{cpp}
void swap(T& a, T& b)
\end{cpp}

交换范围:

\begin{cpp}
FwdIt swap_ranges(FwdIt1 first1, FwdIt1 last1, FwdIt first2)
FwdIt swap_ranges(ExePol pol, FwdIt1 first1, FwdIt1 last1, FwdIt first2)
\end{cpp}

返回的迭代器指向目标范围内最后交换的元素。

\begin{myWarning}{范围不能重叠}
	
\filename{交换算法}

\begin{cpp}
// swap.cpp
...
#include <algorithm>
...

std::vector<int> myVec{0, 1, 2, 3, 4, 5, 6, 7, 9};
std::vector<int> myVec2(9);
std::swap(myVec, myVec2);
for (auto v: myVec) std::cout << v << " "; // 0 0 0 0 0 0 0 0 0
for (auto v: myVec2) std::cout << v << " "; // 0 1 2 3 4 5 6 7 9

std::string str{"abcdefghijklmnop"};
std::string str2{"---------------------"};
std::swap_ranges(str.begin(), str.begin()+5, str2.begin()+5);
std::cout << str << '\n'; // -----fghijklmnop
std::cout << str2 << '\n'; // -----abcde-----------
\end{cpp}
\end{myWarning}

\mySamllsection{变换范围}

std::transform算法将一元或二进制可调用对象应用于范围,并将修改后的元素复制到目标范围。

将一元可调用函数fun应用于输入范围的元素,并将结果复制到result:

\begin{cpp}
OutIt transform(InpIt first1, InpIt last1, OutIt result, UnFun fun)
FwdIt2 transform(ExePol pol, FwdIt first1, FwdIt last1, FwdIt2 result, UnFun fun)
\end{cpp}

将二进制可调用对象fun应用于两个输入范围,并将结果复制到result:

\begin{cpp}
OutIt transform(InpIt1 first1, InpIt1 last1, InpIt2 first2, OutIt result,
			    BiFun fun)
FwdIt3 transform(ExePol pol, FwdIt1 first1, FwdIt1 last1,
				 FwdIt2 first2, FwdIt3 result, BiFun fun)
\end{cpp}

两个版本的不同之处在于,第一个版本将可调用单元应用于范围的每个元素;第二个版本并行地将可调用对象应用于两个范围对。返回的迭代器指向最后一个转换元素之后的一个位置。

\filename{变换算法}

\begin{cpp}
// transform.cpp
...
#include <algorithm>
...

std::string str{"abcdefghijklmnopqrstuvwxyz"};
std::transform(str.begin(), str.end(), str.begin(),
				[](char c){ return std::toupper(c); });
std::cout << str; // ABCDEFGHIJKLMNOPQRSTUVWXYZ

std::vector<std::string> vecStr{"Only", "for", "testing", "purpose", "." };
std::vector<std::string> vecStr2(5, "-");
std::vector<std::string> vecRes;
std::transform(vecStr.begin(), vecStr.end(),
			   vecStr2.begin(), std::back_inserter(vecRes),
			   [](std::string a, std::string b){ return std::string(b)+a+b; });
for (auto str: vecRes) std::cout << str << " ";
                             // -Only- -for- -testing- -purpose- -.-
\end{cpp}

\mySamllsection{反向范围}

std::reverse和std::reverse\_copy将其范围内元素的顺序颠倒。

反转范围内元素的顺序:

\begin{cpp}
void reverse(BiIt first, BiIt last)
void reverse(ExePol pol, BiIt first, BiIt last)
\end{cpp}

反转范围内元素的顺序,并将结果复制到result:

\begin{cpp}
OutIt reverse_copy(BiIt first, BiIt last, OutIt result)
FwdIt reverse_copy(ExePol pol, BiIt first, BiIt last, FwdIt result)
\end{cpp}

这两种算法都需要双向迭代器,返回的迭代器指向复制元素之前的输出范围结果位置。

\filename{反向范围的算法}

\begin{cpp}
// algorithmen.cpp
...
#include <algorithm>
...

std::string str{"123456789"};
std::reverse(str.begin(), str.begin()+5);
std::cout << str; // 543216789
\end{cpp}

\mySamllsection{旋转范围}

std::rotate和std::rotate\_copy旋转它们的元素。

以这样的方式旋转元素,middle成为新的第一个元素:

\begin{cpp}
FwdIt rotate(FwdIt first, FwdIt middle, FwdIt last)
FwdIt rotate(ExePol pol, FwdIt first, FwdIt middle, FwdIt last)
\end{cpp}

旋转元素,使middle成为新的第一个元素。将结果复制到result:

\begin{cpp}
OutIt rotate_copy(FwdIt first, FwdIt middle, FwdIt last, OutIt result)
FwdIt2 rotate_copy(ExePol pol, FwdIt first, FwdIt middle, FwdIt last,
				   FwdIt2 result)
\end{cpp}

两种算法都需要前向迭代器,返回的迭代器是复制范围的end迭代器。

\filename{旋转算法}

\begin{cpp}
// rotate.cpp
...
#include <algorithm>
...

std::string str{"12345"};
for (auto i= 0; i < str.size(); ++i){
	std::string tmp{str};
	std::rotate(tmp.begin(), tmp.begin()+i , tmp.end());
	std::cout << tmp << " ";
} // 12345 23451 34512 45123 51234
\end{cpp}

\mySamllsection{移位范围}

C++20函数std::shift\_left和std::shift\_right移动范围中的元素,这两种算法都返回前向迭代器。

将范围的元素向范围的开头移动n,返回结果范围的末尾位置。

\begin{cpp}
FwdIt shift_left(FwdIt first, FwdIt last, Num n)
FwdIt shift_left(ExePol pol, FwdIt first, FwdIt last, Num n)
\end{cpp}

将范围的元素向范围的末尾移动n,返回结果范围的起始位置。

\begin{cpp}
FwdIt shift_right(FwdIt first, FwdIt last, Num n)
FwdIt shift_right(ExePol pol, FwdIt first, FwdIt last, Num n)
\end{cpp}

当n == 0 || n >= last - first时,对std::shift\_left和std::shift\_right的操作无效。因为,区域的元素移动了。

\filename{移动范围中的元素}

\begin{cpp}
// shiftRange.cpp
...
#include <algorithm>
...

std::vector<int> myVec{1, 2, 3, 4, 5, 6, 7};
for (auto v: myVec) std::cout << v << " "; // 1 2 3 4 5 6 7

auto newEnd = std::shift_left(myVec.begin(), myVec.end(), 2);
myVec.erase(newEnd, myVec.end());
for (auto v: myVec) std::cout << v << " "; // 3 4 5 6 7

auto newBegin = std::shift_right(myVec.begin(), myVec.end(), 2);
myVec.erase(myVec.begin(), newBegin);
for (auto v: myVec) std::cout << v << " "; // 3 4 5
\end{cpp}

\mySamllsection{随机混洗范围}

可以使用std::random\_shuffle和std::shuffle随机混洗范围。

随机混洗范围内的元素:

\begin{cpp}
void random_shuffle(RanIt first, RanIt last)
\end{cpp}

使用随机数生成器gen随机混洗范围内的元素:

\begin{cpp}
void random_shuffle(RanIt first, RanIt last, RanNumGen&& gen)
\end{cpp}

随机混洗一个范围内的元素,使用均匀随机数生成器gen:

\begin{cpp}
void shuffle(RanIt first, RanIt last, URNG&& gen)
\end{cpp}

这些算法需要随机访问迭代器。RanNumGen\&\& gen必须是可调用的,接受参数并在其参数内返回值。URNG\&\& gen必须是一个均匀随机数生成器。

\begin{myTip}{首选std::shuffle}
使用std::shuffle代替std::random\_shuffle。std::random\_shuffle自C++14以来已弃用,并在C++17中删除,因为其在内部使用C函数rand。
\end{myTip}

\filename{随机混洗算法}

\begin{cpp}
// shuffle.cpp
...
#include <algorithm>
...

using std::chrono::system_clock;
using std::default_random_engine;
std::vector<int> vec1{0, 1, 2, 3, 4, 5, 6, 7, 8, 9};
std::vector<int> vec2(vec1);

std::random_shuffle(vec1.begin(), vec1.end());
for (auto v: vec1) std::cout << v << " "; // 4 3 7 8 0 5 2 1 6 9

unsigned seed= system_clock::now().time_since_epoch().count();
std::shuffle(vec2.begin(), vec2.end(), default_random_engine(seed));
for (auto v: vec2) std::cout << v << " "; // 4 0 2 3 9 6 5 1 8 7
\end{cpp}

Seed初始化随机数生成器。

\mySamllsection{删除重复}

使用std::unique和std::unique\_copy算法,可删除相邻的重复项。使用或不使用二元谓词都可以做到这一点。

删除相邻的重复项:

\begin{cpp}
FwdIt unique(FwdIt first, FwdIt last)
FwdIt unique(ExePol pol, FwdIt first, FwdIt last)
\end{cpp}

删除相邻的重复项,满足二元谓词:

\begin{cpp}
FwdIt unique(FwdIt first, FwdIt last, BiPred pre)
FwdIt unique(ExePol pol, FwdIt first, FwdIt last, BiPred pre)
\end{cpp}

删除相邻的重复项并将结果复制到result:

\begin{cpp}
OutIt unique_copy(InpIt first, InpIt last, OutIt result)
FwdIt2 unique_copy(ExePol pol, FwdIt first, FwdIt last, FwdIt2 result)
\end{cpp}

删除相邻的重复项,满足二进制谓词,并将结果复制到result:

\begin{cpp}
OutIt unique_copy(InpIt first, InpIt last, OutIt result, BiPred pre)
FwdIt2 unique_copy(ExePol pol, FwdIt first, FwdIt last,
				   FwdIt2 result, BiPred pre)
\end{cpp}

\begin{myWarning}{唯一算法返回新的逻辑结束迭代器}
唯一算法返回范围的逻辑结束迭代器。必须使用擦除-删除习惯用法删除元素。
\end{myWarning}

\filename{删除重复算法}

\begin{cpp}
// removeDuplicates.cpp
...
#include <algorithm>
...

std::vector<int> myVec{0, 0, 1, 1, 2, 2, 3, 4, 4, 5,
					   3, 6, 7, 8, 1, 3, 3, 8, 8, 9};
					   
myVec.erase(std::unique(myVec.begin(), myVec.end()), myVec.end());
for (auto v: myVec) std::cout << v << ""; // 0 1 2 3 4 5 3 6 7 8 1 3 8 9

std::vector<int> myVec2{1, 4, 3, 3, 3, 5, 7, 9, 2, 4, 1, 6, 8,
						0, 3, 5, 7, 8, 7, 3, 9, 2, 4, 2, 5, 7, 3};
std::vector<int> resVec;
resVec.reserve(myVec2.size());
std::unique_copy(myVec2.begin(), myVec2.end(), std::back_inserter(resVec),
				[](int a, int b){ return (a%2) == (b%2); } );
for(auto v: myVec2) std::cout << v << " ";
						// 1 4 3 3 3 5 7 9 2 4 1 6 8 0 3 5 7 8 7 3 9 2 4 2 5 7 3
for(auto v: resVec) std::cout << v << " "; // 1 4 3 2 1 6 3 8 7 2 5
\end{cpp}
























