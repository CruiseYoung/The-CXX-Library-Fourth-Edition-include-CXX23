许多类封装了文件系统的一个特定脑功能。

\begin{center}
文件系统中的各种类
\end{center}

% Please add the following required packages to your document preamble:
% \usepackage{longtable}
% Note: It may be necessary to compile the document several times to get a multi-page table to line up properly
\begin{longtable}[c]{|l|l|}
\hline
\textbf{类}                 & \textbf{描述}                                      \\ \hline
\endfirsthead
%
\endhead
%
path                           & 表示路径。                                        \\ \hline
filesystem\_error              & 定义异常对象。                              \\ \hline
directory\_entry               & 表示目录项。                             \\ \hline
directory\_iterator            & 定义目录迭代器。                             \\ \hline
recursive\_directory\_iterator & 定义递归目录迭代器。                   \\ \hline
file\_status                   & 存储有关文件的信息。                        \\ \hline
space\_info                    & 表示文件系统信息。                        \\ \hline
file\_type                     & 文件类型。                            \\ \hline
perms                          & 表示文件访问权限。                       \\ \hline
perm\_options                  & 代表权限选项功能。          \\ \hline
copy\_options                  & 表示copy和copy\_file函数的选项。 \\ \hline
directory\_options & 表示函数directory\_iterator和recursive\_directory\_iterator的选项。 \\ \hline
file\_time\_type               & 表示文件时间。                                     \\ \hline
\end{longtable}


\mySamllsection{操纵文件的权限}

文件的权限由类std::filesystem::perms表示,是一个\href{http://en.cppreference.com/w/cpp/concept/BitmaskType}{位掩码类型},因此可以通过位操作来操作。访问权限基于\href{https://en.wikipedia.org/wiki/POSIX}{POSIX}。

来自\href{en.cppreference.com/w/cpp/filesystem/perms}{en.cppreference.com}的程序展示了如何读取和操作文件的所有者、组和其他(世界)位。

\filename{文件权限}

\begin{cpp}
// perms.cpp
...
#include <filesystem>
...
namespace fs = std::filesystem;
void printPerms(fs::perms perm){
	std::cout << ((perm & fs::perms::owner_read) != fs::perms::none ? "r" : "-")
			<< ((perm & fs::perms::owner_write) != fs::perms::none ? "w" : "-")
			<< ((perm & fs::perms::owner_exec) != fs::perms::none ? "x" : "-")
			<< ((perm & fs::perms::group_read) != fs::perms::none ? "r" : "-")
			<< ((perm & fs::perms::group_write) != fs::perms::none ? "w" : "-")
			<< ((perm & fs::perms::group_exec) != fs::perms::none ? "x" : "-")
			<< ((perm & fs::perms::others_read) != fs::perms::none ? "r" : "-")
			<< ((perm & fs::perms::others_write) != fs::perms::none ? "w" : "-")
			<< ((perm & fs::perms::others_exec) != fs::perms::none ? "x" : "-")
			<< '\n';
}

...

std::ofstream("rainer.txt");

std::cout << "Initial file permissions for a file: ";
printPerms(fs::status("rainer.txt").permissions()); // (1)

fs::permissions("rainer.txt", fs::perms::add_perms | // (2)
				 fs::perms::owner_all | fs::perms::group_all);
std::cout << "Adding all bits to owner and group: ";
printPerms(fs::status("rainer.txt").permissions());

fs::permissions("rainer.txt", fs::perms::remove_perms | // (3)
	fs::perms::owner_write | fs::perms::group_write | fs::perms::others_write);
std::cout << "Removing the write bits for all: ";
printPerms(fs::status("rainer.txt").permissions());
\end{cpp}

由于调用fs::status("rainer.txt").permissions(),获得了文件rainer.txt的权限,并可以在函数printPerms(1)中显示它们。通过设置类型std::filesystem::add\_perms,可以向文件的所有者和组添加权限(2)。或者,可以设置常量std::filesystem::remove\_perms来删除权限(3)。

\myGraphic{0.6}{content/chapter18/images/3.jpg}{}





















