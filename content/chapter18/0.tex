\myGraphic{0.4}{content/chapter18/images/1.jpg}{Cippi在整理绘画的东西}

\href{http://en.cppreference.com/w/cpp/filesystem}{filesystem}是基于\href{http://www.boost.org/doc/libs/1_65_1/libs/filesystem/doc/index.htm}{boost::filesystem}的,一些组件是可选的。所以,并非在文件系统库的每个实现上都可以使用std::filessystem的所有功能。例如,FAT-32上就不支持符号链接。

该库基于三个概念:文件、文件名和路径。

\begin{itemize}
\item 
文件是一个对象,它包含可以写入或读取的数据。文件有名称和文件类型。文件类型包括目录、硬链接、符号链接和普通文件。

\begin{itemize}
\item 
目录是存放其他文件的容器。当前目录由点"."表示;两个点表示父目录".."。

\item 
硬链接将名称与现有文件关联起来。

\item 
符号链接将名称与可能存在的路径关联起来。

\item 
常规文件是一个目录条目,既不是目录,也不是硬链接,也不是符号链接。
\end{itemize}

\item 
文件名是一个表示文件的字符串,由实现定义的,允许使用哪些字符,名称可以有多长,或者名称是否区分大小写。

\item 
路径是标识文件位置的条目序列,有一个可选的根名称,如Windows上的“C:”,后面跟着一个根目录,如Unix上的“/”。附加部分可以是目录、硬链接、符号链接或常规文件。路径可以是绝对的、规范的或相对的。

\begin{itemize}
\item 
绝对路径是用来标识文件的路径。

\item 
规范路径是既不包含符号链接,也不包含相对路径的路径。"."(当前目录)或".."(父目录)。

\item 
相对路径指定相对于文件系统中某个位置的路径。路径,如“.”(当前目录),“.."(父目录)或"home/rainer"是相对路径。在Unix上,不是从根目录“/”开始的。
\end{itemize}
\end{itemize}

下面是文件系统的一个介绍性示例。

\filename{文件系统库的概述}

\begin{cpp}
// filesystem.cpp
...
#include <filesystem>

...

namespace fs = std::filesystem;

std::cout << "Current path: " << fs::current_path() << '\n'; // (1)

std::string dir= "sandbox/a/b";
fs::create_directories(dir); // (2)

std::ofstream("sandbox/file1.txt");
fs::path symPath= fs::current_path() /= "sandbox"; // (3)
symPath /= "syma";
fs::create_symlink("a", "symPath"); // (4)

std::cout << "fs::is_directory(dir): " << fs::is_directory(dir) << '\n';
std::cout << "fs::exists(symPath): " << fs::exists(symPath) << '\n';
std::cout << "fs::symlink(symPath): " << fs::is_symlink(symPath) << '\n';

for(auto& p: fs::recursive_directory_iterator("sandbox")){ // (5)
	std::cout << p << '\n';
}
fs::remove_all("sandbox");
\end{cpp}

ss::current\_path()(1)返回当前路径,可以使用std::filesystem::create\_directories创建目录层次结构(2)。/=运算符对于路径(3)是重载的,可以直接创建符号链接(4)或检查文件的属性。调用recursive\_directory\_iterator(5)允许递归遍历目录。

\myGraphic{0.6}{content/chapter18/images/2.jpg}{}































