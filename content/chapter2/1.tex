You can apply the many variations of the min, max, and minmax functions to values and initializer lists. These functions need the header <algorithm>. On the contrary, the functions std::move, std::forward, std::to\_underlying, and std::swap are defined in the header <utility>. You can apply them to arbitrary values.

\mySamllsection{std::min, std::max, and std::minmax}

The functions \href{http://en.cppreference.com/w/cpp/algorithm/min}{std::min}, \href{http://en.cppreference.com/w/cpp/algorithm/max}{std::max}, and \href{http://en.cppreference.com/w/cpp/algorithm/minmax}{std::minmax}, defined in the header <algorithm>, act on values and initializer lists and give you the requested value back as a result. In the case of std::minmax, you get a std::pair. The first element of the pair is the minimum; the second is the maximum of the values. The less operator (<) is used by default, but you can apply your comparison operator. This function needs two arguments and returns a boolean. Functions that either return true or false are called predicates.

\mySamllsection{The functions std::min, std::max, and std::minmax}

\begin{cpp}
// minMax.cpp
...
#include <algorithm>
...
using std::cout;
...
cout << std::min(2011, 2014); // 2011
cout << std::min({3, 1, 2011, 2014, -5}); // -5
cout << std::min(-10, -5, [](int a, int b)
                { return std::abs(a) < std::abs(b); }); // -5

auto pairInt= std::minmax(2011, 2014);
auto pairSeq= std::minmax({3, 1, 2011, 2014, -5});
auto pairAbs= std::minmax({3, 1, 2011, 2014, -5}, [](int a, int b)
                      { return std::abs(a) < std::abs(b); });

cout << pairInt.first << "," << pairInt.second; // 2011,2014
cout << pairSeq.first << "," << pairSeq.second; // -5,2014
cout << pairAbs.first << "," << pairAbs.second; // 1,2014
\end{cpp}

The table provides an overview of the functions std::min, std::max and std::minmax

\begin{center}
The variations of std::min, std::max, and std::minmax
\end{center}

% Please add the following required packages to your document preamble:
% \usepackage{longtable}
% Note: It may be necessary to compile the document several times to get a multi-page table to line up properly
\begin{longtable}[c]{|l|l|}
\hline
\textbf{Function}           & \textbf{Description}                                                                \\ \hline
\endfirsthead
%
\endhead
%
min(a, b)                   & Returns the smaller value of a and b                                                \\ \hline
min(a, b, comp)             & Returns the smaller value of a and b according to the predicate comp.               \\ \hline
min(initialiser list)       & Returns the smallest value of the initializer list.                                 \\ \hline
min(initialiser list, comp) & Returns the smallest value of the initializer list according to the predicate comp. \\ \hline
max(a, b)                   & Returns the greater value of a and b.                                               \\ \hline
max(a, b, comp)             & Returns the greater value of a and b according to the predicate comp.               \\ \hline
max(initialiser list)       & Returns the greatest value of the initializer list.                                 \\ \hline
max(initialiser list, comp) & Returns the grestest value of the initializer list according to the predicate comp. \\ \hline
minmax(a, b)                & Returns the smaller and greater value of a and b.                                   \\ \hline
minmax(a, b, comp)          & Returns the smaller and greater value of a and b according to the predicate comp.   \\ \hline
minmax(initialiser list)    & Returns the samllest and the greatest value of the initializer list.                \\ \hline
minmax(initialiser list, comp) & Returns the samllest and the greatest value of the initializer list according to the predicate comp. \\ \hline
\end{longtable}

\mySamllsection{std::midpoint, and std::lerp}

he function std::midpoint(a, b) calculates the midpoint between a and b. a and b can be integers, floating-point numbers, or pointers. If a and b are pointers, they must point to the same array object.
The function std::midpoint requires the header <numeric>.

The function std::lerp(a, b, t) calculates the linear interpolation of two numbers. It requires the header <cmath>. The return value is a + t(b - a).


\mySamllsection{std::midpoint, and std::lerp}

\begin{cpp}
// midpointLerp.cpp

#include <cmath>
#include <numeric>

...

std::cout << "std::midpoint(10, 20): " << std::midpoint(10, 20) << '\n';

for (auto v: {0.0, 0.1, 0.2, 0.3, 0.4, 0.5, 0.6, 0.7, 0.8, 0.9, 1.0}) {
	std::cout << "std::lerp(10, 20, " << v << "): " << std::lerp(10, 20, v) << "\n";
}
\end{cpp}


\myGraphic{0.4}{content/chapter2/images/2.jpg}{}

\mySamllsection{std::cmp\_equal, std::cmp\_not\_equal, std::cmp\_less, std::cmp\_greater, std::cmp\_less\_equal, and std::cmp\_greater\_equal}


The in the header <utility> definfined functions std::cmp\_equal, std::cmp\_not\_equal, std::cmp\_less, std::cmp\_greater, std::cmp\_less\_equal, and std::cmp\_greater\_equal provide safe comparison of integers. Safe comparison means that the comparison of a negative signed integer compares less than an unsigned integer, and that the comparison of values other than a signed or unsigned integer gives a compile-time error.


\begin{myWarning}{Integer conversion with builtin integers}
The following code snippet exemplifies the issue of signed/unsigned comparison.

\begin{cpp}
-1 < 0u; // true
std::cmp_greater( -1, 0u); // false
\end{cpp}

-1 as a signed integer is promoted to an unsigned type which causes a surprising result.

\end{myWarning}

\mySamllsection{std::move}

The function \href{http://en.cppreference.com/w/cpp/utility/move}{std::move}, defined in the header <utility>, authorizes the compiler to move the resource. In the so-called move semantics, the source object’s values are moved to the new object. Afterward, the source is in a well-defined but indeterminate state. Most of the time, that is the default state of the source. By using std::move, the compiler converts the source arg to an rvalue reference: static\_cast<std::remove\_reference<decltype(arg)>::type\&\&> (arg). If the compiler can not apply the move semantics, it falls back to the copy semantics:

\begin{cpp}
#include <utility>
...
std::vector<int> myBigVec(10000000, 2011);
std::vector<int> myVec;

myVec = myBigVec; // copy semantics
myVec = std::move(myBigVec); // move semantics
\end{cpp}


\begin{myTip}{To move is cheaper than to copy}
	
The move semantics has two advantages. Firstly, it is often a good idea to use cheap moving instead of expensive copying. So there is no extra allocation and deallocation of memory necessary. Secondly, there are objects which can not be copied. E.g., a thread or a lock.

\end{myTip}

\mySamllsection{std::forward}

The function \href{http://en.cppreference.com/w/cpp/utility/forward}{std::forward}, defined in the header <utility>, empowers you to write function templates, which can identically forward their arguments. Typical use cases for std::forward are factory functions or constructors. Factory functions create an object and should, therefore, pass their arguments unmodified. Constructors often use their arguments to initialize their base class with identical arguments. So std::forward is the perfect tool for authors of generic libraries:

\filename{Perfect forwarding}

\begin{cpp}
// forward.cpp
...
#include <utility>
...
using std::initialiser_list;

struct MyData{
	MyData(int, double, char){};
};

template <typename T, typename... Args>
T createT(Args&&... args){
	return T(std::forward<Args>(args)... );
}

...

int a= createT<int>();
int b= createT<int>(1);

std::string s= createT<std::string>("Only for testing.");
MyData myData2= createT<MyData>(1, 3.19, 'a');

typedef std::vector<int> IntVec;
IntVec intVec= createT<IntVec>(initialiser_list<int>({1, 2, 3}));
\end{cpp}

The function template createT has to take their arguments as a \href{https://isocpp.org/blog/2012/11/universal-references-in-c11-scott-meyers}{universal reference}: Args\&\&… args. A universal reference, or forwarding reference, is an rvalue reference in a type deduction context.

\begin{myTip}{std::forward enables completely generic functions in combination with variadic templates}
	
If you use std::forward together with variadic templates, you can define fully generic function templates. Your function template can accept an arbitrary number of arguments and forward them unchanged.
	
\end{myTip}

\mySamllsection{std::to\_underlying}

The function std::to\_underlying in C++23 converts an enumeration enum to is underlying type Enum. This function is a convenience function for the expression static\_cast<std::underlying\_type<Enum>::type>(enum), using the type traits function std::underlying\_type.


\mySamllsection{std::swap}

With the function \href{http://en.cppreference.com/w/cpp/algorithm/swap}{std::swap} defined in the header <utility>, you can easily swap two objects. The generic implementation in the C++ standard library internally uses the function std::move.

\filename{Move-semantic with std::swap}

\begin{cpp}
// swap.cpp
...
#include <utility>
...
template <typename T>
inline void swap(T& a, T& b) noexcept {
	T tmp(std::move(a));
	a = std::move(b);
	b = std::move(tmp);
}
\end{cpp}








































