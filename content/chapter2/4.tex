可以使用\href{http://en.cppreference.com/w/cpp/utility/tuple}{std::tuple}创建任意长度和类型的元组。类模板需要头文件<tuple>。std::tuple是std::pair的泛化形式。包含两个元素和对的元组可以相互转换。与std::pair一样,元组也有一个默认构造函数、一个复制构造函数和一个移动构造函数。

可以使用std::swap函数交换元组。std::get可以访问元组的第i个元素:std::get<i-1>(t)。通过std::get<type>(t),可以直接引用类型type的元素。

元组支持比较操作符==、!=、<、>、<=和>=。若比较两个元组,则按字典顺序比较元组的元素。比较从索引0开始。

\mySamllsection{std::make\_tuple}

辅助函数\href{http://en.cppreference.com/w/cpp/utility/tuple/make_tuple}{std::make\_tuple}提供了一种创建元组的方便方法。不必指定类型,编译器会自动推导出它们。

\filename{辅助函数std::make\_tuple}

\begin{cpp}
// tuple.cpp
...
#include <tuple>
...
using std::get;

std::tuple<std::string, int, float> tup1("first", 3, 4.17f);
auto tup2= std::make_tuple("second", 4, 1.1);

std::cout << get<0>(tup1) << ", " << get<1>(tup1) << ", "
		  << get<2>(tup1) << '\n'; // first, 3, 4.17
std::cout << get<0>(tup2) << ", " << get<1>(tup2) << ", "
		  << get<2>(tup2) << '\n'; // second, 4, 1.1
std::cout << (tup1 < tup2) << '\n'; // true

get<0>(tup2)= "Second";
std::cout << get<0>(tup2) << "," << get<1>(tup2) << ","
		  << get<2>(tup2) << '\n'; // Second, 4, 1.1
std::cout << (tup1 < tup2) << '\n'; // false

auto pair= std::make_pair(1, true);
std::tuple<int, bool> tup= pair;
\end{cpp}

\mySamllsection{std::tie和std::ignore}

\href{http://en.cppreference.com/w/cpp/utility/tuple/tie}{std::tie} 允许创建引用变量的元组。可以使用\href{http://en.cppreference.com/w/cpp/utility/tuple/ignore}{std::ignore}显式忽略元组元素。

\filename{辅助函数std::tie和std::ignore}

\begin{cpp}
// tupleTie.cpp
...
#include <tuple>
...
using namespace std;

int first= 1;
int second= 2;
int third= 3;

int fourth= 4;
cout << first << " " << second << " "
	 << third << " " << fourth << endl; // 1 2 3 4

auto tup= tie(first, second, third, fourth) // bind the tuple
		= std::make_tuple(101, 102, 103, 104); // create the tuple
												// and assign it

cout << get<0>(tup) << " " << get<1>(tup) << " " << get<2>(tup)
	 << " " << get<3>(tup) << endl; // 101 102 103 104
cout << first << " " << second << " " << third << " "
	 << fourth << endl; // 101 102 103 104

first= 201;
get<1>(tup)= 202;
cout << get<0>(tup) << " " << get<1>(tup) << " " << get<2>(tup)
	 << " " << get<3>(tup) << endl; // 201 202 103 104
cout << first << " " << second << " " << third << " "
	 << fourth << endl; // 201 202 103 104

int a, b;
tie(std::ignore, a, std::ignore, b)= tup;
cout << a << " " << b << endl; // 202 104
\end{cpp}








