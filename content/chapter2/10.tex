std::expected<T, E>提供了一种存储两个值中的任意一个的方法。std::expected的实例总是保存一个值:要么是类型T的期望值,要么是类型E的非期望值,使用该类型需要头文件<expected>。std::expected可以实现返回值或返回错误的函数,存储值直接在预期对象占用的存储空间中分配,从而不进行动态内存分配。

std::expected有一个类似的接口,比如std::optional。与std::optional相反,std:: expected会返回错误消息。

各种构造函数允许定义具有期望值的预期对象exp。exp.emplace将就地构造包含的值,可以显式地询问std::expected容器是否有值,或者以在逻辑表达式中进行检查。exp.value返回预期值,exp.value\_or返回预期值或默认值。若exp有一个意外值,调用exp.value将抛出std::bad\_expected\_access异常。

std::unexpected表示存储在std::expected中的非预期值。

\filename{std::expected}

\begin{cpp}
// expected.cpp

#include <iostream>
#include <expected>
#include <vector>
#include <string>

std::expected<int, std::string> getInt(std::string arg) {
	try {
		return std::stoi(arg);
	}
	catch (...) {
		return std::unexpected{std::string(arg + ": Error")};
	}
}

int main() {

	std::vector<std::string> strings = {"66", "foo", "-5"};
	
	for (auto s: strings) {
		auto res = getInt(s);
		if (res) {
			std::cout << res.value() << ' '; // 66 -5
		}
		else {
			std::cout << res.error() << ' '; // foo: Error
		}
	}
	
	std::cout << '\n';
	
	for (auto s: strings) {
		auto res = getInt(s);
		std::cout << res.value_or(2023) << ' '; // 66 2023 -5
	}

}
\end{cpp}

getInt函数将每个字符串转换为整数,并返回std::expected<int, std::string>。Int表示预期值,std::string表示意外值。两个基于范围的for循环(第22行和第34行)遍历std::vector<std::string>。第一个基于范围的for循环(第22行)中,显示预期值(第25行)或意外值(第28行)。第二个基于范围的for循环(第34行)中,显示预期值或默认值2023(第36行)。

std::expected支持单进操作以方便函数组合:exp.and\_then、exp.transform、exp.or\_else和exp.transform\_error。exp.and\_then返回给定函数调用的结果(若存在),或者返回一个空std::expected。exp.transform返回一个std::expected,其中包含转换后的值,或者一个空std::expected。此外,若exp.or\_else包含值或给定函数的结果,则返回std:: expected。

下面的程序基于前面的程序optionalMonadic.cpp。本质上,std::optional类型可用std:: expected替换。

\filename{对std::expected的一元操作}

\begin{cpp}
// expectedMonadic.cpp
...
#include <expected>

std::expected<int, std::string> getInt(std::string arg) {
	try {
		return std::stoi(arg);
	}
	catch (...) {
		return std::unexpected{std::string(arg + ": Errror")};
	}
}

std::vector<std::string> strings = {"66", "foo", "-5"};

for (auto s: strings) {
	auto res = getInt(s)
				.transform( [](int n) { return n + 100; })
				.transform( [](int n) { return std::to_string(n); });
	std::cout << *res << ' '; // 166 foo: Error 95
}
\end{cpp}

基于范围的for循环(第23行)遍历std::vector<std::string>。首先,getInt函数将每个字符串转换为整数(第24行),为其添加100(第25行),将其转换回字符串(第26行),最后显示字符串(第27行)。若初始转换为int失败,则返回string arg + ": Error"(第14行)并显示。







































