\href{http://en.cppreference.com/w/cpp/header/chrono}{时间库}主要由时间点、时间段和时钟三个部分组成。该库还提供了时间功能、日历支持、时区支持,以及输入和输出支持。

\hspace*{\fill}

\noindent
\textbf{时间点}

时间点由起始点、所谓的历元和附加的时间段来定义。

\noindent
\textbf{时间段}

时间段时间是两个时间点之间的差值,时钟周期数定义了这个值。

\noindent
\textbf{时钟}
 
时钟由起始点(历元)和刻度组成,这样就可以计算出当前的时间点。

\noindent
\textbf{日期}

从午夜开始的时间分为小时:分钟:秒。

\noindent
\textbf{日历}

表示各种日历日期,如年、月、工作日或一周的第n天。

\noindent
\textbf{时区}

表示特定于某个地理区域的时间。

\begin{myNotic}{时间库是多线程的关键组件}
时间库是C++新多线程功能的关键组件,可以将当前线程通过std::this\_thread::sleep\_for(std::chrono::milliseconds(15)设置为15毫秒的睡眠状态,或者尝试获取2分钟的锁:lock。try\_lock\_until(now + std::chrono::minutes(2))。
\end{myNotic}


\mySamllsection{时间点}

时间段是由一段时间组成,定义为某个时间单位的时钟周期数。时间点由时钟和持续时间组成。这个持续时间可以是正的,也可以是负的。

\begin{cpp}
template <class Clock, class Duration= typename Clock::duration> class time_point;
\end{cpp}

时钟std::chrono::steady\_clock、std::chrono::high\_resolution\_clock和std::chrono::system没有定义历元。但是在主流平台上,std::chrono::system的epoch通常定义为1.1.1970。可以计算自1970年1月1日以来的时间,其分辨率为纳秒、秒和分钟。

\filename{自纪元以来的时间}

\begin{cpp}
// epoch.cpp
...
#include <chrono>
...
auto timeNow= std::chrono::system_clock::now();
auto duration= timeNow.time_since_epoch();
std::cout << duration.count() << "ns" // 1413019260846652ns

typedef std::chrono::duration<double> MySecondTick;
MySecondTick mySecond(duration);
std::cout << mySecond.count() << "s"; // 1413019260.846652s

const int minute= 60;
typedef std::chrono::duration<double, <minute>> MyMinuteTick;
MyMinuteTick myMinute(duration);
std::cout << myMinute.count() << "m"; // 23550324.920572m
\end{cpp}

由于函数std::chrono::clock\_cast,可以在不同的时钟之间转换时间点。

\begin{myTip}{使用时间库进行简单的性能测试}
	
\filename{性能测试}

\begin{cpp}
// performanceMeasurement.cpp
...
#include <chrono>
...
std::vector<int> myBigVec(10000000, 2011);
std::vector<int> myEmptyVec1;

auto begin= std::chrono::high_resolution_clock::now();
myEmptyVec1 = myBigVec;
auto end= std::chrono::high_resolution_clock::now() - begin;

auto timeInSeconds = std::chrono::duration<double>(end).count();
std::cout << timeInSeconds << '\n'; // 0.0150688800
\end{cpp}
	
\end{myTip}


\mySamllsection{时间段}

时间段是两个时间点之间的差值,时间段以时钟周期数来衡量。

\begin{cpp}
template <class Rep, class Period = ratio<1>> class duration;
\end{cpp}

若Rep是浮点数,则时间持续时间支持小数。最重要的时间段在chrono库中定义:

\begin{cpp}
typedef duration<signed int, nano> nanoseconds;
typedef duration<signed int, micro> microseconds;
typedef duration<signed int, milli> milliseconds;
typedef duration<signed int> seconds;
typedef duration<signed int, ratio< 60>> minutes;
typedef duration<signed int, ratio<3600>> hours;
\end{cpp}

时间段可以有多长?C++标准保证预定义的时间长度可以存储+/-292年。可以很容易地定义时间长度,就像一个德国学校的小时:typedef std::chrono::duration<double, std::ratio<2700>{}> MyLessonTick。以自然数表示的时间持续时间必须显式地转换为以浮点数表示的时间持续时间,该值将截断:

\filename{时间段}

\begin{cpp}
// duration.cpp
...
#include <chrono>
#include <ratio>

using std::chrono;

typedef duration<long long, std::ratio<1>> MySecondTick;
MySecondTick aSecond(1);

milliseconds milli(aSecond);
std::cout << milli.count() << " milli"; // 1000 milli

seconds seconds(aSecond);
std::cout << seconds.count() << " sec"; // 1 sec

minutes minutes(duration_cast<minutes>(aSecond));
std::cout << minutes.count() << " min"; // 0 min

typedef duration<double, std::ratio<2700>> MyLessonTick;
MyLessonTick myLesson(aSecond);
std::cout << myLesson.count() << " less"; // 0.00037037 less
\end{cpp}

\begin{myNotic}{std::ratio}
	
std::ratio支持在编译时使用有理数进行算术运算。有理数有两个模板参数:分母和分母。C++11预先定义了许多有理数。
	
\begin{cpp}
typedef ratio<1, 1000000000000000000> atto;
typedef ratio<1, 1000000000000000> femto;
typedef ratio<1, 1000000000000> pico;
typedef ratio<1, 1000000000> nano;
typedef ratio<1, 1000000> micro;
typedef ratio<1, 1000> milli;
typedef ratio<1, 100> centi;
typedef ratio<1, 10> deci;
typedef ratio< 10, 1> deca;
typedef ratio< 100, 1> hecto;
typedef ratio< 1000, 1> kilo;
typedef ratio< 1000000, 1> mega;
typedef ratio< 1000000000, 1> giga;
typedef ratio< 1000000000000, 1> tera;
typedef ratio< 1000000000000000, 1> peta;
typedef ratio< 1000000000000000000, 1> exa;
\end{cpp}
	
\end{myNotic}

C++14为最常用的时间段提供了内置字面符。

\begin{center}
用于时间持续时间的内置字面符
\end{center}

% Please add the following required packages to your document preamble:
% \usepackage{longtable}
% Note: It may be necessary to compile the document several times to get a multi-page table to line up properly
\begin{longtable}[c]{|l|l|l|}
\hline
\textbf{类型}             & \textbf{后缀} & \textbf{举例} \\ \hline
\endfirsthead
%
\endhead
%
std::chrono::hours        & h               & 5h               \\ \hline
std::chrono::minutes      & min             & 5min             \\ \hline
std::chrono::seconds      & s               & 5s               \\ \hline
std::chrono::milliseconds & ms              & 5ms              \\ \hline
std::chrono::microseconds & us              & 5us              \\ \hline
std::chrono::nanoseconds  & ns              & 5ns             \\ \hline
\end{longtable}

\mySamllsection{时钟}

时钟由时间起点和刻度组成,可以通过成员函数获得当前时间。

\hspace*{\fill}

\noindent
\textbf{std::chrono::system\_clock }

系统时间,可与外部时钟同步。

\noindent
\textbf{std::chrono::steady\_clock }

时钟,无法调整。

\noindent
\textbf{std::chrono::high\_resolution\_clock }

最精确的系统时间。

\hspace*{\fill}

std::chrono::system\_clock通常指1.1.1970,不能将std::steady\_clock向前或向后调整为与其他两个时钟相反。成员函数to\_time\_t和from\_time\_t可用于在std::chrono::system\_clock和std::time\_t对象之间进行转换。

\mySamllsection{日期}

std::chrono::time\_of\_day将从午夜开始的持续时间分成小时:分钟:秒。函数std::chrono::is\_am和std::chrono::is\_pm检查时间是在正午之前(子午线之前)还是之后(子午线之后)。

std::chrono::time\_of\_day对象tOfDay支持各种成员函数。

\begin{center}
std::chrono::time\_of\_day的成员函数
\end{center}

% Please add the following required packages to your document preamble:
% \usepackage{longtable}
% Note: It may be necessary to compile the document several times to get a multi-page table to line up properly
\begin{longtable}[c]{|l|l|}
\hline
\textbf{成员函数} & \textbf{描述}                                    \\ \hline
\endfirsthead
%
\endhead
%
tOfDay.hours()           & 返回自午夜开始的小时数。              \\ \hline
tOfDay.minutes()         & 返回自午夜开始的分钟数。            \\ \hline
tOfDay.seconds()         & 返回午夜之后的第二个分量。            \\ \hline
tOfDay.subseconds()      & 返回从午夜开始的秒数(小数)。 \\ \hline
tOfDay.to\_duration()    & 返回从午夜开始的时间。               \\ \hline
\begin{tabular}[c]{@{}l@{}}std::chrono::make12(hr)\\ std::chrono::make24(hr)\end{tabular} & 返回相当于24小时(12小时)格式时间的12小时(24小时)。 \\ \hline
\begin{tabular}[c]{@{}l@{}}std::chrono::is\_am(hr)\\ std::chrono::is\_pm(hr)\end{tabular} & 检测24小时格式时间是上午还是下午。                   \\ \hline
\end{longtable}

\mySamllsection{日历}

日历表示各种日期,例如年、月、工作日或一周的第n天。

\filename{当前时间}

\begin{cpp}
// currentTime.cpp
...
#include <chrono>
using std::chrono;
...
auto now = system_clock::now();
std::cout << "The current time is " << now << " UTC\n";

auto currentYear = year_month_day(floor<days>(now)).year();
std::cout << "The current year is " << currentYear << '\n';

auto h = floor<hours>(now) - sys_days(January/1/currentYear);
std::cout << "It has been " << h << " since New Years!\n";

std::cout << '\n';

auto birthOfChrist = year_month_weekday(sys_days(January/01/0000));
std::cout << "Weekday: " << birthOfChrist.weekday() << '\n';
\end{cpp}

程序的输出显示了与当前时间相关的信息。

\myGraphic{0.9}{content/chapter2/images/4.jpg}{}

下表给出了日历类型的概述。

\begin{center}
各种日历类型
\end{center}

% Please add the following required packages to your document preamble:
% \usepackage{longtable}
% Note: It may be necessary to compile the document several times to get a multi-page table to line up properly
\begin{longtable}[c]{|l|l|}
\hline
\textbf{类型}   & \textbf{描述}                                     \\ \hline
\endfirsthead
%
\endhead
%
last\_spec       & 表示一个月内的最后一天或最后一个工作日。           \\ \hline
day              & 表示一个月中的一天。                            \\ \hline
month            & 表示一年中的一个月。                           \\ \hline
year             & 表示公历中的一年。            \\ \hline
weekday          & 表示公历中一周中的一天。 \\ \hline
weekday\_indexed & 表示一个月的第n个工作日                  \\ \hline
weekday\_last    & 表示一个月的最后一个工作日                  \\ \hline
month\_day       & 表示特定月份的特定日期。          \\ \hline
month\_day\_last & 表示特定月份的最后一天。            \\ \hline
month\_weekday   & 表示指定月份的第n个工作日。        \\ \hline
month\_weekday\_last       & 表示特定月份的最后一个工作日。          \\ \hline
year\_month      & 表示特定年份的特定月份。         \\ \hline
year\_month\_day & 表示特定的年、月和日。             \\ \hline
year\_month\_day\_last     & 表示特定年份和月份的最后一天      \\ \hline
year\_month\_weekday       & 表示特定的年、月和工作日      \\ \hline
year\_month\_weekday\_last & 表示特定年份和月份的最后一个工作日。 \\ \hline
\end{longtable}


\mySamllsection{时区}

时区表示特定于一个地理区域的时间,下面的代码段显示了不同时区的本地时间。

\filename{显示不同时区的本地时间}

\begin{cpp}
// timezone.cpp
...
#include <chrono>
using std::chrono;
..
auto time = floor<milliseconds>(system_clock::now());
auto localTime = zoned_time<milliseconds>(current_zone(), time);
auto berlinTime = zoned_time<milliseconds>("Europe/Berlin", time);
auto newYorkTime = std::chorno::zoned_time<milliseconds>("America/New_York", time);
auto tokyoTime = zoned_time<milliseconds>("Asia/Tokyo", time);

std::cout << time << '\n';
std::cout << localTime << '\n';
std::cout << berlinTime << '\n';
std::cout << newYorkTime << '\n';
std::cout << tokyoTime << '\n';
\end{cpp}

时区功能支持访问\href{https://www.iana.org/time-zones}{IANA时区数据库},支持不同时区的操作,并提供有关闰秒的信息。

下表给出了时区功能的概述。欲了解更多详细信息,请参阅\href{https://en.cppreference.com/w/cpp/chrono}{cpprefere.com}。

\begin{center}
时区信息
\end{center}

% Please add the following required packages to your document preamble:
% \usepackage{longtable}
% Note: It may be necessary to compile the document several times to get a multi-page table to line up properly
\begin{longtable}[c]{|l|l|}
\hline
\textbf{类型} & \textbf{描述}                     \\ \hline
\endfirsthead
%
\endhead
%
tzdb          & 描述IANA时区数据库。   \\ \hline
locate\_zone  & 根据time\_zone的名称定位时间。  \\ \hline
current\_zone & 返回当前time\_zone。          \\ \hline
time\_zone    & 表示一个时区。                  \\ \hline
sys\_info    & 返回特定时间点的时区信息。       \\ \hline
local\_info  & 表示从本地时间到UNIX时间的信息。 \\ \hline
zoned\_time   & 表示时区和时间点。 \\ \hline
leap\_second & 包含有关插入闰秒的信息。                   \\ \hline
\end{longtable}

\mySamllsection{时间的I/O}

函数std::chrono::parse解析流中的chrono对象。

\filename{解析时间点和时区}

\begin{cpp}
std::istringstream inputStream{"1999-10-31 01:30:00 -08:00 US/Pacific"};

std::chrono::local_seconds timePoint;
std::string timeZone;
inputStream >> std::chrono::parse("%F %T %Ez %Z", timePoint, timeZone);
\end{cpp}

解析功能提供了各种格式说明符来处理一天中的时间和日历日期,如年、月、周和日。\href{https://en.cppreference.com/w/cpp/chrono/parse}{cppreference.com}为格式说明符提供详细信息。





















