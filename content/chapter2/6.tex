智能指针对于C++来说是必不可少的,它们能够在C++中实现显式的内存管理。除了已弃用的std::auto\_ptr之外,C++还提供了三个智能指针,在头文件<memory>中定义。

首先,std::unique\_ptr对独占所有权的概念进行了建模。其次,std::shared\_ptr对共享所有权的概念进行了建模。std::weak\_ptr不是智能的,因为它有一个有限的接口,其作用是打破std::shared\_ptr的循环引用。它模拟了临时所有权的概念。

智能指针根据RAII习惯用法管理它们的资源,若智能指针超出作用域,资源将自动释放。

\begin{myNotic}{资源获取即初始化}

资源获取即初始化(Resource Acquisition Is Initialization,简称RAII)是C++中的一种流行技术,其中资源获取和释放与对象的生命周期绑定在一起。对于智能指针,内存在构造函数中分配,在析构函数中释放。在C++中,当对象超出作用域时自动调用析构函数。

\end{myNotic}

\begin{center}
智能指针概述
\end{center}

% Please add the following required packages to your document preamble:
% \usepackage{longtable}
% Note: It may be necessary to compile the document several times to get a multi-page table to line up properly
\begin{longtable}[c]{|l|l|l|}
\hline
\textbf{名称} &
\textbf{标准} &
\textbf{描述} \\ \hline
\endfirsthead
%
\endhead
%
\begin{tabular}[c]{@{}l@{}}std::auto\_ptr\\ (已废弃)\end{tabular} &
C++98 &
独占资源。移动资源,然后复制。 \\ \hline
std::unique\_ptr &
C++11 &
独占资源。不能复制. \\ \hline
std::shared\_ptr &
C++11 &
\begin{tabular}[c]{@{}l@{}}具有共享变量的引用计数器,自动管理引用计数器。\\ 若引用计数器为0,则删除资源。\end{tabular} \\ \hline
std::weak\_ptr &
C++11 &
\begin{tabular}[c]{@{}l@{}}帮助打破std::shared\_ptr的循环引用。\\ 不修改引用计数器。\end{tabular} \\ \hline
\end{longtable}


\mySamllsection{std::unique\_ptr}

\href{http://en.cppreference.com/w/cpp/memory/unique_ptr}{std::unique\_ptr} 专门负责其资源。若超出作用域,会自动释放资源。若没有使用复制语义,可以在标准模板库的容器和算法中使用std::unique\_ptr。若不使用特殊的删除器,std::unique\_ptr会与原始指针一样廉价和快速。

\begin{myWarning}{不要使用std::auto\_ptr}

经典的C++03有一个智能指针std::auto\_ptr,专门负责资源的生命周期。但是std::auto\_ptr有一个概念问题。若隐式或显式复制std::auto\_ptr,则可能会移动资源。没有复制语义,你有隐藏的移动语义,经常会有未定义行为。因此std::auto\_ptr在C++11中已弃用,所以应该使用std::unique\_ptr。既不能隐式地也不能显式地复制std::unique\_ptr,只能移动它:

\begin{cpp}
#include <memory>
...
std::auto_ptr<int> ap1(new int(2011));
std::auto_ptr<int> ap2 = ap1; // OK

std::unique_ptr<int> up1(new int(2011));
std::unique_ptr<int> up2 = up1; // ERROR
std::unique_ptr<int> up3 = std::move(up1); // OK
\end{cpp}

\end{myWarning}

这些是std::unique\_ptr的成员函数。

\begin{center}
std::unique\_ptr的成员函数
\end{center}

% Please add the following required packages to your document preamble:
% \usepackage{longtable}
% Note: It may be necessary to compile the document several times to get a multi-page table to line up properly
\begin{longtable}[c]{|l|l|}
\hline
名称         & 描述                                       \\ \hline
\endfirsthead
%
\endhead
%
get          & 返回指向资源的指针                 \\ \hline
get\_deleter & 返回delete函数。                      \\ \hline
release      & 返回指向资源的指针并释放它。 \\ \hline
reset        & 重置资源。                              \\ \hline
swap         & 交换资源。                                \\ \hline
\end{longtable}

在下面的代码片段中,可以看到这些成员函数的具体使用:

\filename{The std::unique\_ptr}

\begin{cpp}
// uniquePtr.cpp
...
#include <utility>
...
using namepace std;

struct MyInt{
	MyInt(int i):i_(i){}
	~MyInt(){
		cout << "Good bye from " << i_ << endl;
	}
	int i_;
};

unique_ptr<MyInt> uniquePtr1{new MyInt(1998)};
cout << uniquePtr1.get() << endl; // 0x15b5010

unique_ptr<MyInt> uniquePtr2{move(uniquePtr1)};
cout << uniquePtr1.get() << endl; // 0
cout << uniquePtr2.get() << endl; // 0x15b5010
{
	unique_ptr<MyInt> localPtr{new MyInt(2003)};
} // Good bye from 2003
uniquePtr2.reset(new MyInt(2011)); // Good bye from 1998
MyInt* myInt= uniquePtr2.release();
delete myInt; // Good by from 2011

unique_ptr<MyInt> uniquePtr3{new MyInt(2017)};
unique_ptr<MyInt> uniquePtr4{new MyInt(2022)};
cout << uniquePtr3.get() << endl; // 0x15b5030
cout << uniquePtr4.get() << endl; // 0x15b5010

swap(uniquePtr3, uniquePtr4);
cout << uniquePtr3.get() << endl; // 0x15b5010
cout << uniquePtr4.get() << endl; // 0x15b5030
\end{cpp}

std::unique\_ptr对数组有一个特化:

\filename{std::unique\_ptr array}

\begin{cpp}
// uniquePtrArray.cpp
...
#include <memory>
...
using namespace std;

class MyStruct{
	public:
	MyStruct():val(count){
		cout << (void*)this << " Hello: " << val << endl;
		MyStruct::count++;
	} ~
	MyStruct(){
		cout << (void*)this << " Good Bye: " << val << endl;
		MyStruct::count--;
	}
	private:
	int val;
	static int count;
};

int MyStruct::count= 0;
...
{
	// generates a myUniqueArray with three `MyStructs`
	unique_ptr<MyStruct[]> myUniqueArray{new MyStruct[3]};
}
// 0x1200018 Hello: 0
// 0x120001c Hello: 1
// 0x1200020 Hello: 2
// 0x1200020 GoodBye: 2
// 0x120001c GoodBye: 1
// 0x1200018 GoodBye: 0
\end{cpp}

\mySamllsection{特殊的删除器}

std::unique\_ptr可以用特殊的删除器:std::unique\_ptr<int, MyIntDeleter> up(new int(2011), MyIntDeleter()))。默认情况下,std::unique\_ptr使用相应资源的删除器。


\mySamllsection{std::make\_unique}


辅助函数\href{http://en.cppreference.com/w/cpp/memory/unique_ptr/make_unique}{std::make\_unique}与C++11标准中被遗忘的同级std::make\_shared不同。因此,std::make\_unique添加到C++14标准中。std::make\_unique能够在一个步骤中创建std::unique\_ptr: std::unique\_ptr<int> up= std::make\_unique<int>(2014)。


\mySamllsection{std::shared\_ptr}

\href{http://en.cppreference.com/w/cpp/memory/shared_ptr}{std::shared\_ptr} 共享资源的所有权。它们有两个句柄:一个用于资源,一个用于引用计数器。通过复制std::shared\_ptr,引用计数增加1。若std::shared\_ptr超出作用域,则减少1。若引用计数器的值变为0,并且不再有std::shared\_ptr引用该资源,则C++运行时自动释放该资源。资源的释放恰好发生在最后一个std::shared\_ptr超出作用域的时候。C++运行时保证引用计数器的调用是一个原子操作。由于这种管理开销,std::shared\_ptr比原始指针或std::unique\_ptr需要更多的时间和内存。

下表是std::shared\_ptr的成员函数。

\begin{center}
std::shared\_ptr的成员函数
\end{center}

% Please add the following required packages to your document preamble:
% \usepackage{longtable}
% Note: It may be necessary to compile the document several times to get a multi-page table to line up properly
\begin{longtable}[c]{|l|l|}
\hline
\textbf{名称} & \textbf{描述}                                             \\ \hline
\endfirsthead
%
\endhead
%
get           & 返回指向资源的指针                                \\ \hline
get\_deleter  & 返回delete函数。                                      \\ \hline
reset         & 重置资源                                              \\ \hline
swap          & 交换资源                                               \\ \hline
unique        & 检查std::shared\_ptr是否是资源的独占所有者 \\ \hline
use\_count    & 返回引用计数器的值。                      \\ \hline
\end{longtable}

\mySamllsection{std::make\_shared}

辅助函数\href{http://en.cppreference.com/w/cpp/memory/shared_ptr/make_shared}{std::make\_shared}创建资源,并在std::shared\_ptr中返回它。应该选择std::make\_shared,而非直接创建std::shared\_ptr,因为std::make\_shared更快。

下面的代码示例显示了std::shared\_ptr的典型用例。

\filename{std::shared\_ptr}

\begin{cpp}
// sharedPtr.cpp
...
#include <memory>
...
class MyInt{
public:
	MyInt(int v):val(v){
		std::cout << "Hello: " << val << '\n';
	} ~
	MyInt(){
		std::cout << "Good Bye: " << val << '\n';
	}
private:
	int val;
};

auto sharPtr= std::make_shared<MyInt>(1998); // Hello: 1998
std::cout << sharPtr.use_count() << '\n'; // 1

{
	std::shared_ptr<MyInt> locSharPtr(sharPtr);
	std::cout << locSharPtr.use_count() << '\n'; // 2
}
std::cout << sharPtr.use_count() << '\n'; // 1

std::shared_ptr<MyInt> globSharPtr= sharPtr;
std::cout << sharPtr.use_count() << '\n'; // 2

globSharPtr.reset();
std::cout << sharPtr.use_count() << '\n'; // 1
sharPtr= std::shared_ptr<MyInt>(new MyInt(2011)); // Hello:2011
												  // Good Bye: 1998
...
// Good Bye: 2011
\end{cpp}

本例中,可调用对象是一个函数对象,可以很容易地计算创建了多少个类的实例。结果在静态变量count中。

\mySamllsection{std::shared\_ptr from this}

必须从std::enable\_shared\_from\_this派生类public,可以使用类\href{http://en.cppreference.com/w/cpp/memory/enable_shared_from_this}{std::enable\_shared\_from\_this}对象进行创建,该类返回自身的std::shared\_ptr。类支持成员函数shared\_from\_this返回std::shared_ptr到this:

\filename{std::shared\_ptr from this}

\begin{cpp}
// enableShared.cpp
...
#include <memory>
...
class ShareMe: public std::enable_shared_from_this<ShareMe>{
	std::shared_ptr<ShareMe> getShared(){
		return shared_from_this();
	}
};

std::shared_ptr<ShareMe> shareMe(new ShareMe);
std::shared_ptr<ShareMe> shareMe1= shareMe->getShared();

std::cout << (void*)shareMe.get() << '\n'; // 0x152d010
std::cout << (void*)shareMe1.get() << '\n'; // 0x152d010
std::cout << shareMe.use_count() << '\n'; // 2
\end{cpp}

从代码示例中可以看到,get成员函数引用同一个对象。

\mySamllsection{std::weak\_ptr}

\href{http://en.cppreference.com/w/cpp/memory/weak_ptr}{std::weak\_ptr} 不是一个智能指针。std::weak\_ptr不支持对资源的透明访问,因为它只从std::shared\_ptr中借用资源。std::weak\_ptr并不改变引用计数器:

\filename{std::weak\_ptr}

\begin{cpp}
// weakPtr.cpp
...
#include <memory>
...
auto sharedPtr= std::make_shared<int>(2011);
std::weak_ptr<int> weakPtr(sharedPtr);

std::cout << weakPtr.use_count() << '\n'; // 1
std::cout << sharedPtr.use_count() << '\n'; // 1

std::cout << weakPtr.expired() << '\n'; // false
if( std::shared_ptr<int> sharedPtr1= weakPtr.lock() ) {
	std::cout << *sharedPtr << '\n'; // 2011
}
else{
	std::cout << "Don't get it!" << '\n';
}

weakPtr.reset();

if( std::shared_ptr<int> sharedPtr1= weakPtr.lock() ) {
	std::cout << *sharedPtr << '\n';
}
else{
	std::cout << "Don't get it!" << '\n'; // Don't get it!
}
\end{cpp}

该表概述了std::weak\_ptr的成员函数。

\begin{center}
std::weak\_ptr的成员函数。
\end{center}

% Please add the following required packages to your document preamble:
% \usepackage{longtable}
% Note: It may be necessary to compile the document several times to get a multi-page table to line up properly
\begin{longtable}[c]{|l|l|}
\hline
\textbf{名称} & \textbf{描述}                        \\ \hline
\endfirsthead
%
\endhead
%
expired       & 检查资源是否删除。         \\ \hline
lock          & 在资源上创建一个std::shared\_ptr  \\ \hline
reset         & 重置资源                         \\ \hline
swap          & 交换资源                          \\ \hline
use\_count    & 返回引用计数器的值。\\ \hline
\end{longtable}

std::weak\_ptr存在有一个原因,打破了std::shared\_ptr的循环引用。

\mySamllsection{循环引用}

若std::shared\_ptr相互引用,则得到它们的循环引用。因此,资源计数器永远不会变为0,资源也不会自动释放。若在循环中嵌入一个std::weak\_ptr,就可以打破这个循环,因为std::weak\_ptr不修改引用计数器。

代码示例的结果是,daughter自动释放,但son和mother都没有被释放。mother通过std::shared\_ptr指代son,通过std::weak\_ptr指代daughter。图中看到代码的结构会帮助读者对此关系进行理解。

\myGraphic{0.7}{content/chapter2/images/3.jpg}{循环引用}

\filename{循环引用}

\begin{cpp}
// cyclicReference.cpp
...
#include <memory>
...
using namespace std;

struct Son, Daughter;

struct Mother{
	~Mother(){cout << "Mother gone" << endl;}
	void setSon(const shared_ptr<Son> s ){mySon= s;}
	void setDaughter(const shared_ptr<Daughter> d){myDaughter= d;}
	shared_ptr<const Son> mySon;
	weak_ptr<const Daughter> myDaughter;
};

struct Son{
	Son(shared_ptr<Mother> m):myMother(m){}
	~Son(){cout << "Son gone" << endl;}
	shared_ptr<const Mother> myMother;
};

struct Daughter{
	Daughter(shared_ptr<Mother> m):myMother(m){}
	~Daughter(){cout << "Daughter gone" << endl;}
	shared_ptr<const Mother> myMother;
};

{
	shared_ptr<Mother> mother= shared_ptr<Mother>(new Mother);
	shared_ptr<Son> son= shared_ptr<Son>(new Son(mother) );
	shared_ptr<Daughter> daugh= shared_ptr<Daughter>(new Daughter(mother));
	mother->setSon(son);
	mother->setDaughter(daugh);
}
// Daughter gone
\end{cpp}


