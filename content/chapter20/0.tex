\myGraphic{0.6}{content/chapter20/images/1.jpg}{Cippi在浇花}

C++20没有具体的协程,只有一个实现协同程序的框架。C++23中,在范围库中有了第一个使用std::generator的具体协程。本章只展示一个具有挑战性的协程框架的大致概念。

协程是可以在保持状态的情况下暂停和恢复执行的函数。为了实现这一点,协程由三部分组成:promise对象、协程句柄和协程框架。

\begin{itemize}
\item 
promise对象在协程中被操作,并通过promise对象返回结果。

\item 
协程句柄是一个非拥有的句柄,用于从外部恢复或销毁协程帧。

\item 
协程帧是一个内部的、典型的堆分配状态,由前面提到的promise对象、协程的复制参数、挂起点的表示、声明周期在当前挂起点之前结束的局部变量,以及生命周期超过当前挂起点的局部变量组成。
\end{itemize}

优化协程的分配有两个必要条件:

使用关键字co\_return而不是return、co\_yield或co\_await的函数隐式地成为协程。

新的关键字用两个新概念扩展了C++函数的执行。

\noindent
\\\textbf{co\_yield 表达式}

允许编写生成器函数。生成器函数每次都返回一个新值。生成器函数是一个数据流,可以从中选择值。数据流可以无限。

\filename{生成器}

\begin{cpp}
Generator<int> getNext(int start = 0, int step = 1) noexcept {
	auto value = start;
	for (int i = 0;; ++i){
		co_yield value;
		value += step;
	}
}
\end{cpp}

\noindent
\\\textbf{co\_await 表达式}

挂起并恢复表达式的执行。若在函数func中使用co\_await表达式,则若函数的结果不可用,则调用auto getResult = func()不会阻塞。而不是消耗资源的阻塞,是资源友好的等待。表达式必须是一个所谓的可等待表达式,并且必须支持以下三个函数await\_ready、await\_suspend和await\_resume。

\filename{一个协程}

\begin{cpp}
Acceptor acceptor{443};

while (true){
	Socket socket= co_await acceptor.accept();
	auto request= co_await socket.read();
	auto response= handleRequest(request);
	co_await socket.write(response);
}
\end{cpp}

co\_await expr中的可等待表达式expr必须实现await\_ready、await\_suspend和await\_resume函数。






















