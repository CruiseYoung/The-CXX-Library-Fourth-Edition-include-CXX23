根据功能可以对迭代器进行分类。迭代器的类别取决于所使用的容器类型。C++有前向、双向、随机访问和连续迭代器。使用前向迭代器,可以使用双向迭代器在两个方向上向前迭代容器。随机访问迭代器允许它直接访问任意元素。特别是,随机迭代器支持迭代器算术和排序比较(例如:<)。连续迭代器是一种随机访问迭代器,并要求容器的元素连续地存储在内存中。

下表是容器及其迭代器类别的表示。双向迭代器包括正向迭代器的功能。随机访问迭代器包括前向和双向迭代器的功能。It和It2是迭代器,n是自然数。

\begin{center}
容器迭代器的类别
\end{center}

% Please add the following required packages to your document preamble:
% \usepackage{longtable}
% Note: It may be necessary to compile the document several times to get a multi-page table to line up properly
\begin{longtable}[c]{|l|l|l|}
\hline
\textbf{迭代器类别} &
\textbf{属性} &
\textbf{容器} \\ \hline
\endfirsthead
%
\endhead
%
前向迭代器 &
\begin{tabular}[c]{@{}l@{}}++It, It++, *It\\ It == It2, It != It2\end{tabular} &
\begin{tabular}[c]{@{}l@{}}std::unordered\_set\\ std::unordered\_map\\ std::unordered\_multiset\\ std::unordered\_multimap\\ std::forward\_list\end{tabular} \\ \hline
双向迭代器 &
--It, It-- &
\begin{tabular}[c]{@{}l@{}}std::set\\ std::map\\ std::multiset\\ std::multimap\\ std::list\end{tabular} \\ \hline
随机访问迭代器 &
\begin{tabular}[c]{@{}l@{}}It{[}i{]}\\ It += n, It -= n\\ It+n, It-n\\ n+It\\ It-It2\end{tabular} &
std::deque \\ \hline
连续迭代器 &
\begin{tabular}[c]{@{}l@{}}It \textless It2, It \textless{}= It2, It \textgreater It2\\ It \textgreater{}= It2\end{tabular} &
\begin{tabular}[c]{@{}l@{}}std::array\\ std::vector\\ std::string\end{tabular} \\ \hline
\end{longtable}

输入迭代器和输出迭代器是特殊的前向迭代器:只能对所指向的元素读写一次。




































