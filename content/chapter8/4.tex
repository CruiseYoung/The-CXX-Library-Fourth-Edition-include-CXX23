迭代器适配器允许在插入模式或流中使用迭代器,其需要头文件<iterator>。

\mySamllsection{插入迭代器}

使用三个插入迭代器std::front\_inserter、std::back\_inserter和std::inserter,可以分别在容器的开头、结尾或任意位置插入元素。这三个辅助函数,会将其功能映射到容器内容的底层成员函数上,元素的内存将自动提供。

下表提供了两条信息:容器的哪些成员函数是内部使用的,哪些迭代器可以使用,这取决于容器的类型。

\begin{center}
三个插入迭代器
\end{center}

% Please add the following required packages to your document preamble:
% \usepackage{longtable}
% Note: It may be necessary to compile the document several times to get a multi-page table to line up properly
\begin{longtable}[c]{|l|l|l|}
\hline
\textbf{名称}             & \textbf{内部使用的成员函数} & \textbf{容器}                                             \\ \hline
\endfirsthead
%
\endhead
%
std::front\_inserter(val) & cont.push\_front(val)                    & \begin{tabular}[c]{@{}l@{}}std::deque\\ std::list\end{tabular} \\ \hline
std::back\_inserter(val) & cont.push\_back(val) & \begin{tabular}[c]{@{}l@{}}std::vector\\ std::deque\\ std::list\\ std::string\end{tabular}                       \\ \hline
std::inserter(val, pos)  & con.insert(pos, val) & \begin{tabular}[c]{@{}l@{}}std::vector\\ std::deque\\ std::list\\ std::string\\ std::map\\ std::set\end{tabular} \\ \hline
\end{longtable}

可以将STL中的算法与三个插入迭代器结合使用。

\begin{cpp}
#include <iterator>
...
std::deque<int> deq{5, 6, 7, 10, 11, 12};
std::vector<int> vec{1, 2, 3, 4, 5, 6, 7, 8, 9, 10, 11, 12, 13, 14, 15};

std::copy(std::find(vec.begin(), vec.end(), 13),
			vec.end(), std::back_inserter(deq));

for (auto d: deq) std::cout << d << " ";
	// 5 6 7 10 11 12 13 14 15

std::copy(std::find(vec.begin(), vec.end(), 8),
std::find(vec.begin(), vec.end(), 10),
std::inserter(deq,
std::find(deq.begin(), deq.end(), 10)));d
for (auto d: deq) std::cout << d << " ";
	// 5 6 7 8 9 10 11 12 13 14 15
	
std::copy(vec.rbegin()+11, vec.rend(),
std::front_inserter(deq));
for (auto d: deq) std::cout << d << " ";
		// 1 2 3 4 5 6 7 8 9 10 11 12 13 14 15
\end{cpp}

\mySamllsection{流迭代器}


流迭代器适配器可以使用流作为数据源或数据接收器。C++提供了两个函数来创建istream迭代器和两个函数来创建ostream迭代器。创建的istream迭代器的行为类似于输入迭代器,ostream迭代器的行为类似于插入迭代器。

\begin{center}
四个流迭代器
\end{center}


% Please add the following required packages to your document preamble:
% \usepackage{longtable}
% Note: It may be necessary to compile the document several times to get a multi-page table to line up properly
\begin{longtable}[c]{|l|l|}
\hline
\textbf{功能}                                         & \textbf{描述}                     \\ \hline
\endfirsthead
%
\endhead
%
std::istream\_iterator\textless{}T\textgreater{}          & 创建end-if-stream迭代器。       \\ \hline
std::istream\_iterator\textless{}T\textgreater{}(istream) & 为istream创建istream迭代器。 \\ \hline
std::ostream\_iterator\textless{}T\textgreater{}(ostream) & 为ostream创建ostream迭代器。 \\ \hline
std::ostream\_iterator\textless{}T\textgreater{}(ostream, delim) & 用分隔符delim为ostream创建ostream迭代器。 \\ \hline
\end{longtable}

有了流迭代器适配器,就可以直接从流中读取或写入流。

下面的交互式程序段在一个无限循环中从std::cin中读取自然数,并将其压入vector myIntVec。若输入不是自然数,则会在输入流中出现错误。myIntVec中的所有数字都复制到std::cout中,中间用:分隔。

\begin{cpp}
#include <iterator>
...
std::vector<int> myIntVec;
std::istream_iterator<int> myIntStreamReader(std::cin);
std::istream_iterator<int> myEndIterator;

// Possible input
// 1
// 2
// 3
// 4
// z
while(myIntStreamReader != myEndIterator){
	myIntVec.push_back(*myIntStreamReader);
	++myIntStreamReader;
}

std::copy(myIntVec.begin(), myIntVec.end(),
			std::ostream_iterator<int>(std::cout, ":"));
			// 1:2:3:4:
\end{cpp}




























