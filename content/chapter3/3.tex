An iterator allows you to access the elements of a container. If you use a begin and end iterator, you have a range, which you can further process. For a container cont, you get with cont.begin() the begin iterator and with cont.end() the end iterator, which defines a half-open range. It is half-open because the begin iterator belongs to the range, and the end iterator refers to a position past the range. The iterator pair cont.begin() and cont.end() enables it to modify the elements of the container.

\begin{center}
Creation and deletion of a container
\end{center}

% Please add the following required packages to your document preamble:
% \usepackage{longtable}
% Note: It may be necessary to compile the document several times to get a multi-page table to line up properly
\begin{longtable}[c]{|l|l|}
\hline
\textbf{Iterator}               & \textbf{Description}                         \\ \hline
\endfirsthead
%
\endhead
%
cont.begin() and cont.end()     & Pair of iterators to iterate forward.        \\ \hline
cont.cbegin() and cont.cend()   & Pair of iterators to iterate const forward.  \\ \hline
cont.rbegin() and cont.rend()   & Pair of iterators to iterate backward.       \\ \hline
cont.crbegin() and cont.crend() & Pair of iterators to iterate const backward. \\ \hline
\end{longtable}

Now I can modify the container.

\filename{Access the elements of a container}

\begin{cpp}
// containerAccess.cpp
...
#include <vector>
...
struct MyInt{
	MyInt(int i): myInt(i){};
	int myInt;
};

std::vector<MyInt> myIntVec;
myIntVec.push_back(MyInt(5));
myIntVec.emplace_back(1);
std::cout << myIntVec.size() << '\n'; // 2

std::vector<int> intVec;
intVec.assign({1, 2, 3});
for (auto v: intVec) std::cout << v << " "; // 1 2 3

intVec.insert(intVec.begin(), 0);
for (auto v: intVec) std::cout << v << " "; // 0 1 2 3

intVec.insert(intVec.begin()+4, 4);
for (auto v: intVec) std::cout << v << " "; // 0 1 2 3 4

intVec.insert(intVec.end(), {5, 6, 7, 8, 9, 10, 11});

for (auto v: intVec) std::cout << v << " "; // 0 1 2 3 4 5 6 7 8 9 10 11

for (auto revIt= intVec.rbegin(); revIt != intVec.rend(); ++revIt)
	std::cout << *revIt << " "; // 11 10 9 8 7 6 5 4 3 2 1 0

intVec.pop_back();
for (auto v: intVec ) std::cout << v << " "; // 0 1 2 3 4 5 6 7 8 9 10
\end{cpp}















