For a container cont, you can check with cont.empty() if the container is empty. cont.size() returns the current number of elements, and cont.max\_size() returns the maximum number of elements cont can have. The maximum number of elements is implementation-defined.

\filename{Size of a container}

\begin{cpp}
// containerSize.cpp
...
#include <map>
#include <set>
#include <vector>
...
using namespace std;

vector<int> intVec{1, 2, 3, 4, 5, 6, 7, 8, 9};
map<string, int> str2Int = {{"bart", 12345},
	{"jenne", 34929}, {"huber", 840284}};
	
set<double> douSet{3.14, 2.5};

cout << intVec.empty() << endl; // false
cout << str2Int.empty() << endl; // false
cout << douSet.empty() << endl; // false

cout << intVec.size() << endl; // 9
cout << str2Int.size() << endl; // 3
cout << douSet.size() << endl; // 2

cout << intVec.max_size() << endl; // 4611686018427387903
cout << str2Int.max_size() << endl; // 384307168202282325
cout << douSet.max_size() << endl; // 461168601842738790
\end{cpp}


\begin{myTip}{Use cont.empty() instead of cont.size()}
For a container cont, use the member function cont.empty() instead of (cont.size() == 0) to determine if the container is empty. First, cont.empty() is in general faster than (const.size() == 0); second, std::forward\_list has no member function size().	
\end{myTip}











