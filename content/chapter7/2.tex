std::mdspan是一个连续对象序列的非拥有多维视图。通常,这种多维视图称为多维数组。对象的连续序列可以是C数组、带大小的指针、std::array、std::vector或std::string。

维度的数量和每个维度的大小决定了多维数组的形状。维度的数量称为秩,每个维度的大小扩展。std::mdspan的大小,是所有不为0的维度的乘积,所以可以使用多维索引操作符[]访问std::mdspan的元素。

std::mdspan的每个维度可以有一个静态区段或一个动态区段。静态范围意味着其长度在编译时指定;动态区段意味着其长度是在运行时指定的。

\filename{std::mdspan的定义}

\begin{cpp}
template<
	class T,
	class Extents,
	class LayoutPolicy = std::layout_right,
	class AccessorPolicy = std::default_accessor<T>
> class mdspan;
\end{cpp}

\begin{itemize}
\item 
T: 连续的对象序列

\item 
Extents: 指定维度的数量作为它们的大小;每个维度可以有一个静态范围或一个动态范围

\item 
LayoutPolicy: 指定访问底层内存的布局策略

\item 
AccessorPolicy: 指定如何引用基础元素
\end{itemize}

由于C++17中的\href{https://en.cppreference.com/w/cpp/language/class_template_argument_deduction}{类模板参数推演,class template argument deduction (CTAG)},编译器通常可以自动推断模板参数。

\filename{两个二维数组}

\begin{cpp}
// mdspan.cpp

#include <mdspan>
#include <iostream>
#include <vector>

int main() {

	std::vector myVec{1, 2, 3, 4, 5, 6, 7, 8};
	
	std::mdspan m{myVec.data(), 2, 4};
	std::cout << "m.rank(): " << m.rank() << '\n';
	
	for (std::size_t i = 0; i < m.extent(0); ++i) {
		for (std::size_t j = 0; j < m.extent(1); ++j) {
			std::cout << m[i, j] << ' ';
		}
		std::cout << '\n';
	}
	
	std::cout << '\n';
	
	std::mdspan m2{myVec.data(), 4, 2};
	std::cout << "m2.rank(): " << m2.rank() << '\n';
	for (std::size_t i = 0; i < m2.extent(0); ++i) {
		for (std::size_t j = 0; j < m2.extent(1); ++j) {
			std::cout << m2[i, j] << ' ';
		}
		std::cout << '\n';
	}

}
\end{cpp}

这个例子中,我应用了三次类模板参数推导。第9行将其用于std::vector,第11行和第23行将其用于std::mdspan。第一个二维数组m的形状为(2,4),第二个二维数组m2的形状为(4,2)。第12行和第24行显示了两个std::mdspan的秩。由于每个维度的范围(第14行和第15行)以及第16行中的索引操作符,可以直接遍历多维数组。

\myGraphic{0.3}{content/chapter7/images/4.jpg}{两个二维数组}

若多维数组应该具有静态范围,则必须指定模板参数。

\filename{显式指定std::mdspan的模板参数}

\begin{cpp}
// staticDynamicExtent.cpp

#include <mdspan>
...

std::mdspan<int, std::extents<std::size_t, 2, 4>> m{myVec.data()}; // (1)
std::cout << "m.rank(): " << m.rank() << '\n';

for (std::size_t i = 0; i < m.extent(0); ++i) {
	for (std::size_t j = 0; j < m.extent(1); ++j) {
		std::cout << m[i, j] << ' ';
	}
	std::cout << '\n';
}

std::mdspan<int, std::extents<std::size_t, std::dynamic_extent, std::dynamic_extent>>
				m2{myVec.data(), 4, 2}; // (2)
std::cout << "m2.rank(): " << m2.rank() << '\n';

for (std::size_t i = 0; i < m2.extent(0); ++i) {
	for (std::size_t j = 0; j < m2.extent(1); ++j) {
		std::cout << m2[i, j] << ' ';
	}
	std::cout << '\n';
}
\end{cpp}

staticDynamicExtent.cpp基于前面的程序mdspan.cpp,并产生相同的输出。不同之处在于,std::mdspan m(1)具有静态区段。为了完整起见,std::mdspan m2(2)具有动态范围。因此,m的形状是用模板参数指定的,而m2的形状是用函数参数指定的。

std::mdspan允许您指定访问底层内存的布局策略。默认情况下,使用std::layout\_right(C, C++或\href{https://en.wikipedia.org/wiki/Python_(programming_language)}{Python}风格),但也可以指定std::layout\_eft (\href{https://en.wikipedia.org/wiki/Fortran}{Fortran}或\href{https://en.wikipedia.org/wiki/MATLAB}{MATLAB}风格)。下面的图举例说明了访问std::mdspan的元素的顺序。

\myGraphic{0.6}{content/chapter7/images/5.jpg}{std::layout\_right 和 std::layout\_left}

使用布局策略std::layout\_right和std::layout\_left遍历两个std::mdspan,展示了其差异。

\filename{使用std::mdspan:std::layout\_right和std::layout\_left}

\begin{cpp}
// mdspanLayout.cpp
...
#include <mdspan>

std::vector myVec{1, 2, 3, 4, 5, 6, 7, 8};

std::mdspan<int, // (1)
	std::extents<std::size_t, std::dynamic_extent, std::dynamic_extent>,
	std::layout_right> m2{myVec.data(), 4, 2};

std::cout << "m.rank(): " << m.rank() << '\n';

for (std::size_t i = 0; i < m.extent(0); ++i) {
	for (std::size_t j = 0; j < m.extent(1); ++j) {
		std::cout << m[i, j] << ' ';
	}
	std::cout << '\n';
}

std::cout << '\n';

std::mdspan<int,
	std::extents<std::size_t, std::dynamic_extent, std::dynamic_extent>,
	std::layout_left> m2{myVec.data(), 4, 2}; // (2)
std::cout << "m2.rank(): " << m2.rank() << '\n';

for (std::size_t i = 0; i < m2.extent(0); ++i) {
	for (std::size_t j = 0; j < m2.extent(1); ++j) {
		std::cout << m2[i, j] << ' ';
	}
	std::cout << '\n';
}
\end{cpp}

std::mdspan m使用std::layout\_right(1),另一个std::mdspan std::layout\_left(1)。通过类模板参数推导,std::mdspan(1)的构造函数调用不需要显式的模板参数,相当于以下表达式:std::mdspan m2\{myVec.data(), 4, 2\}。

程序的输出显示了两种不同的布局策略。

\myGraphic{0.3}{content/chapter7/images/6.jpg}{std::mdspan和std::layout\_left}

下表给出了std::mdspan接口的概述。

\begin{center}
std::mdspan md的成员函数
\end{center}

% Please add the following required packages to your document preamble:
% \usepackage{longtable}
% Note: It may be necessary to compile the document several times to get a multi-page table to line up properly
\begin{longtable}[c]{|l|l|}
\hline
\textbf{函数} & \textbf{描述}                                       \\ \hline
\endfirsthead
%
\endhead
%
md{[}ind{]}       & 访问第ind个元素。                                \\ \hline
md.size           & 返回多维数组的大小。            \\ \hline
md.rank           & 返回多维数组的维度。      \\ \hline
md.extents(i)     & 返回第i维的大小。                   \\ \hline
md.data\_handle   & 返回一个指向连续元素序列的指针。 \\ \hline
\end{longtable}


































