std::span表示引用连续对象序列的对象,std::span(有时也称为视图)从来不是所有者。这个连续的对象序列可以是普通的C数组、带大小的指针、std::array、std::vector或std::string,还可以访问其子序列。

\begin{myTip}{std::span不会衰变}
	
当调用以C数组为参数的函数时,会发生衰变。该函数通过指向数组第一个元素的指针获取数组。C数组到指针的转换很容易出错,因为C数组的所有长度信息都会丢失。

相反,std::span知道其长度。

\begin{cpp}
// copySpanArray.cpp
...
#include <span>

template <typename T>
void copy_n(const T* p, T* q, int n){}

template <typename T>
void copy(std::span<const T> src, std::span<T> des){}

int arr1[] = {1, 2, 3};
int arr2[] = {3, 4, 5};

copy_n(arr1, arr2, 3); // (1)
copy<int>(arr1, arr2); // (2)
\end{cpp}

与C数组(1)相比,通过std::span(2)接受C数组的函数不需要显式的长度参数。
\end{myTip}

std::span可以具有静态区段或动态区段,带有动态extent的std::span的实现包含指向其第一个元素和长度的指针。默认情况下,std::span有一个动态范围:

\filename{std::span的定义}

\begin{cpp}
template <typename T, std::size_t Extent = std::dynamic_extent>
class span;
\end{cpp}

下表给出了std::span的成员函数。

\begin{center}
std::span sp的成员函数
\end{center}

% Please add the following required packages to your document preamble:
% \usepackage{longtable}
% Note: It may be necessary to compile the document several times to get a multi-page table to line up properly
\begin{longtable}[c]{|l|l|}
\hline
\textbf{函数} & \textbf{描述}                                \\ \hline
\endfirsthead
%
\endhead
%
sp.front()        & 访问第一个元素。                           \\ \hline
sp.back()         & 访问最后一个元素。                            \\ \hline
sp{[}i{]}         & 访问第i个元素。                            \\ \hline
sp.data()         & 返回一个指向序列开头的指针。 \\ \hline
sp.size\_bytes()  & 以字节为单位返回序列的大小。          \\ \hline
sp.empty()        & 如果序列为空则返回true。              \\ \hline
\begin{tabular}[c]{@{}l@{}}sp.first\textless{}count\textgreater{}()\\ sp.first(count)\end{tabular} &
返回由序列的第一个count元素组成的子span。 \\ \hline
\begin{tabular}[c]{@{}l@{}}sp.last\textless{}count\textgreater{}()\\ sp.last(count)\end{tabular} &
返回由序列的最后count个元素组成的子span。 \\ \hline
\begin{tabular}[c]{@{}l@{}}sp.subspan\textless{}fist, count\textgreater{}()\\ sp.subspan(frist, count)\end{tabular} &
返回由count元素组成的子span。 \\ \hline
\end{longtable}

\filename{std::span接受不同的参数}

\begin{cpp}
// printSpan.cpp
...
#include <span>

void printMe(std::span<int> container) {
	std::cout << "container.size(): " << container.size() << '\n';
	for(auto e : container) std::cout << e << ' ';
	std::cout << "\n\n";
}
std::cout << '\n';

int arr[]{1, 2, 3, 4};
printMe(arr); // (1)

std::vector vec{1, 2, 3, 4, 5};
printMe(vec); // (2)

std::array arr2{1, 2, 3, 4, 5, 6};
printMe(arr2); // (3)
\end{cpp}

std::span可以用C-array、std::vector(1)或std::array(2)初始化。


\myGraphic{0.6}{content/chapter7/images/2.jpg}{自动减少std::span的大小}

\begin{myTip}{std::string\_view优于std::span}
std::string是一个连续的字符序列,可以用std::string初始化std::span。不过,应该选择std::string\_view,而不是std::span,std::string\_view表示字符序列的视图,而不是std::span那样的对象序列。std::string\_view的接口类似于字符串,但是std::span的接口是通用的。
\end{myTip}

可以修改整个span或仅修改子span。当修改一个span时,同时也修改了其引用的对象。

\filename{修改std::span引用的对象}

\begin{cpp}
// spanTransform.cpp
...
#include <span>

std::vector vec{1, 2, 3, 4, 5, 6, 7, 8, 9, 10};
printMe(vec);

std::span span1(vec); // (1)
std::span span2{span1.subspan(1, span1.size() - 2)}; // (2)
std::transform(span2.begin(), span2.end(), // (3)
				span2.begin(),
				[](int i){ return i * i; });
				
printMe(vec);
printMe(span1);
\end{cpp}

我在前面的示例quadSpan.cpp中定义了printMe函数。span1引用std::vector vec(1)。相反,span2只引用底层vec元素,不包括第一个和最后一个元素(2),将每个元素映射到它的平方只指向这些元素(3)。

\myGraphic{0.6}{content/chapter7/images/3.jpg}{修改std::span引用的对象}


