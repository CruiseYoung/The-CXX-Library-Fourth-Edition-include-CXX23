流是一个无限的数据流,可以在其上推或拉数据。字符串流和文件流允许字符串和文件直接与流交互。

\mySamllsection{字符串流}

字符串流需要头文件<sstream>。不连接到输入或输出流,并将其数据存储在字符串中。

无论是使用字符串流作为输入或输出,还是使用字符类型char或wchar\_t,都有各种字符串流类:

\noindent
\\\textbf{std::istringstream 和 std::wistringstream}

用于char和wchar\_t类型输入数据的字符串流。

\noindent
\textbf{std::ostringstream 和 std::wostringstream}

用于char和wchar\_t类型输出数据的字符串流。

\noindent
\textbf{std::stringstream 和 std::wstringstream}

用于char和wchar\_t类型的输入或输出数据的字符串流。\\

对字符串流的操作有:

\begin{itemize}
\item 
字符串流中写入数据:

\begin{cpp}
std::stringstream os;
os << "New String";
os.str("Another new String");
\end{cpp}

\item 
从字符串流中读取数据:

\begin{cpp}
std::stringstream os;
std::string str;
os >> str;
str= os.str();
\end{cpp}

\item 
清除字符串流:

\begin{cpp}
std::stringstream os;
os.str("");
\end{cpp}
\end{itemize}

字符串流通常用于字符串和数值之间的类型安全转换:

\filename{字符串流}

\begin{cpp}
// stringStreams.cpp
...
#include <sstream>
...

template <typename T>
T StringTo ( const std::string& source ){
	std::istringstream iss(source);
	T ret;
	iss >> ret;
	return ret;
}

template <typename T>
std::string ToString(const T& n){
	std::ostringstream tmp ;
	tmp << n;
	return tmp.str();
}

std::cout << "5= " << StringTo<int>("5"); // 5
std::cout << "5 + 6= " << StringTo<int>("5") + 6; // 11
std::cout << ToString(StringTo<int>("5") + 6 ); // "11"
std::cout << "5e10: " << std::fixed << StringTo<double>("5e10"); // 50000000000
\end{cpp}

\mySamllsection{文件流}

文件流能够处理文件。文件流自动管理其文件的生命周期,需要头文件<fstream>。

无论使用文件流作为输入或输出,还是使用字符类型char或wchar\_t,都有各种文件流类:

\noindent
\\\textbf{std::ifstream 和 std::wifstream.}

用于char和wchar\_t类型输入数据的文件流。

\noindent
\textbf{std::ofstream 和 std::wofstream}

用于char和wchar\_t类型输出数据的文件流。

\noindent
\textbf{std::fstream 和 std::wfstream}

用于char和wchar\_t类型的输入和输出数据的文件流。

\noindent
\textbf{std::filebuf 和 std::wfilebuf}

类型为char和wchar\_t的数据内存。\\

\begin{myWarning}{设置文件位置指针}
用于读写的文件流必须在竞赛改变后设置文件位置指针。
\end{myWarning}

标志能够设置文件流的打开模式。

\begin{center}
打开文件流的标志
\end{center}

% Please add the following required packages to your document preamble:
% \usepackage{longtable}
% Note: It may be necessary to compile the document several times to get a multi-page table to line up properly
\begin{longtable}[c]{|l|l|}
\hline
\textbf{标志}    & \textbf{描述}                                                             \\ \hline
\endfirsthead
%
\endhead
%
std::ios::in     & 打开文件流进行读取(默认为std::ifstream和std::wifstream)。 \\ \hline
std::ios::out    & 打开文件流进行写入(默认为std::ofstream和std::wofstream)。 \\ \hline
std::ios::app    & 将字符追加到文件流的末尾。                             \\ \hline
std::ios::ate & 设置文件流末尾的文件位置指针的初始位置。 \\ \hline
std::ios::trunc  & 删除原始文件。                                                       \\ \hline
std::ios::binary & 禁止文件流中转义序列的解释。          \\ \hline
\end{longtable}

将名为in的文件,复制到以文件缓冲区in.rdbuf()命名的文件中非常容易。在这个简短的示例中需要包含错误处理。

\begin{cpp}
#include <fstream>
...
std::ifstream in("inFile.txt");
std::ofstream out("outFile.txt");
out << in.rdbuf();
\end{cpp}

下面的表中,我比较了C++和C打开文件的模式。

\begin{center}
用C++和C打开一个文件
\end{center}

文件必须以模式“r”和“r+”存在。相反,文件是用“a”和“w+”创建的,该文件将被“w”覆盖。

可以显式地管理文件流的生命周期。

\begin{center}
管理文件流的生命周期
\end{center}

% Please add the following required packages to your document preamble:
% \usepackage{longtable}
% Note: It may be necessary to compile the document several times to get a multi-page table to line up properly
\begin{longtable}[c]{|l|l|l|}
\hline
\textbf{C++模式}                          & \textbf{描述}        & \textbf{C模式} \\ \hline
\endfirsthead
%
\endhead
%
std::ios::in                               & 读取文件            & "r"             \\ \hline
std::ios::out                              & 写入文件            & "w"             \\ \hline
std::ios::out|std::ios::app                & 追加到文件。       & "a"             \\ \hline
std::ios::in|std::ios::out                 & 读取并写入文件。 & "r+"            \\ \hline
std::ios::in|std::ios::out|std::ios::trunc &写入和读取文件。 & "w+"            \\ \hline
\end{longtable}

\mySamllsection{随机访问}

随机访问可以设置文件的位置指针。

构造文件流时,文件的位置指针指向文件的开头。通过文件流文件的成员函数,可以调整文件流文件的位置。

\begin{center}
导航文件流
\end{center}

% Please add the following required packages to your document preamble:
% \usepackage{longtable}
% Note: It may be necessary to compile the document several times to get a multi-page table to line up properly
\begin{longtable}[c]{|l|l|}
\hline
\textbf{成员函数} & \textbf{描述}                    \\ \hline
\endfirsthead
%
\endhead
%
file.tellg()             & 返回文件的读取位置。      \\ \hline
file.tellp()             & 返回文件的写入位置。     \\ \hline
file.seekg(pos)          & 设置文件的读取位置为pos。  \\ \hline
file.seekp(pos)          & 设置文件的写位置为pos。 \\ \hline
file.seekg(off. rpos) & 将文件的读位置设置为相对于rpos的偏移量。  \\ \hline
file.seekp(off, rpos) & 将文件的写位置设置为相对于rpos的偏移量。 \\ \hline
\end{longtable}

off必须是一个数字。rpos可以是以下三个值:

\noindent
\\\textbf{std::ios::beg}

文件开头的位置。

\noindent
\textbf{std::ios::cur}

当前位置。

\noindent
\textbf{std::ios::end}

位于文件末尾的位置。

\begin{myWarning}{尊重文件边界}
若随机访问一个文件,C++运行时不会检查文件边界。读取或写入边界之外的数据是未定义行为。
\end{myWarning}

\filename{随机访问}

\begin{cpp}
// randomAccess.cpp
...
#include <fstream>
...

void writeFile(const std::string name){
	std::ofstream outFile(name);
	if (!outFile){
		std::cerr << "Could not open file " << name << '\n';
		exit(1);
	}
	for (unsigned int i= 0; i < 10 ; ++i){
		outFile << i << " 0123456789" << '\n';
	}
}
std::string random{"random.txt"};
writeFile(random);
std::ifstream inFile(random);

if (!inFile){
	std::cerr << "Could not open file " << random << '\n';
	exit(1);
}

std::string line;

std::cout << inFile.rdbuf();
// 0 0123456789
// 1 0123456789
...
// 9 0123456789

std::cout << inFile.tellg() << '\n'; // 200

inFile.seekg(0); // inFile.seekg(0, std::ios::beg);
getline(inFile, line);
std::cout << line; // 0 0123456789

inFile.seekg(20, std::ios::cur);
getline(inFile, line);
std::cout << line; // 2 0123456789

inFile.seekg(-20, std::ios::end);
getline(inFile, line);
std::cout << line; // 9 0123456789
\end{cpp}

\mySamllsection{流的状态}

标志表示流的状态,处理这些标志的成员函数需要头文件<iostream>。

\begin{center}
流的状态
\end{center}

下面是导致流的不同状态的条件示例:


std::ios::eofbit

\begin{itemize}
\item 
读取超出最后一个有效字符的内容。
\end{itemize}

std::ios::failbit

\begin{itemize}
\item 
错误的格式化读取。

\item 
读取超出最后一个有效字符的内容。

\item 
打开文件出错。
\end{itemize}

std::ios::badbit

\begin{itemize}
\item 
无法调整流缓冲区的大小。

\item 
流缓冲区的代码转换出错。

\item 
部分流抛出了异常。
\end{itemize}

stream.fail()返回是否设置了std::ios::failbit或std::ios::badbit。

流的状态可以读取和设置。

\noindent
\\\textbf{stream.clear()}

初始化标志并将流设置为goodbit状态。

\noindent
\textbf{stream.clear(sta)}

初始化标志并将流设置为sta状态。

\noindent
\textbf{stream.rdstate()}

返回当前状态。

\noindent
\textbf{stream.setstate(fla})

设置附加标志fla。\\

流上的操作只有在流处于好位状态时才有效。若流处于badbit状态,则不能将其设置回goodbit状态。

\filename{流的状态}

\begin{cpp}
// streamState.cpp
...
#include <iostream>
...
std::cout << std::cin.fail() << '\n'; // false

int myInt;
while (std::cin >> myInt){ // <a>
	std::cout << myInt << '\n'; //
	std::cout << std::cin.fail() << '\n'; //
}

std::cin.clear();
std::cout << std::cin.fail() << '\n'; // false
\end{cpp}

字符a的输入导致流std::cin处于std::ios::failbit状态。因此不能显示a和std::cin.fail(),首先,必须初始化流std::cin。































