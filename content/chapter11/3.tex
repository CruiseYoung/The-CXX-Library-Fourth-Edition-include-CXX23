范围适配器将范围转换为视图。

\begin{cpp}
std::vector<int> numbers = {1, 2, 3, 4, 5, 6};
auto results = numbers | std::views::filter([](int n){ return n % 2 == 0; })
				       | std::views::transform([](int n){ return n * 2; });
\end{cpp}

代码片段中,numbers是范围,std::views::filter和std::views::transform是视图。

C++20中的范围库有一组丰富的视图。

\begin{center}
C++20中的视图
\end{center}

% Please add the following required packages to your document preamble:
% \usepackage{longtable}
% Note: It may be necessary to compile the document several times to get a multi-page table to line up properly
\begin{longtable}[c]{|l|l|}
\hline
\textbf{视图}                                                                                & \textbf{描述}                                     \\ \hline
\endfirsthead
%
\endhead
%
\begin{tabular}[c]{@{}l@{}}std::views::all\_t\\ std::views::all\end{tabular}                 & 获取所有元素。                                      \\ \hline
std::ranges::ref\_view                                                                       & 获取另一个范围的所有元素。                     \\ \hline
\begin{tabular}[c]{@{}l@{}}std::ranges::filter\_view\\ std::views::filter\end{tabular}       & 获取满足谓词的元素。          \\ \hline
\begin{tabular}[c]{@{}l@{}}std::ranges::transfrom\_view\\ std::views::transform\end{tabular} & 转换每个元素。                                 \\ \hline
\begin{tabular}[c]{@{}l@{}}std::ranges::take\_view\\ std::views::take\end{tabular}           & 取另一个视图的前n个元素。              \\ \hline
\begin{tabular}[c]{@{}l@{}}std::ranges::take\_while\_view\\ std::views::take\_while\end{tabular} &
只要谓词返回true,就接受另一个视图的元素。 \\ \hline
\begin{tabular}[c]{@{}l@{}}std::ranges::drop\_view\\ std::views::drop\end{tabular}           & 跳过另一个视图的前n个元素。              \\ \hline
\begin{tabular}[c]{@{}l@{}}std::ranges::drop\_while\_view\\ std::views::drop\_while\end{tabular} &
跳过另一个视图的初始元素,直到谓词返回false。 \\ \hline
\begin{tabular}[c]{@{}l@{}}std::ranges::join\_view\\ std::views::join\end{tabular}           & 加入范围视图。                                   \\ \hline
\begin{tabular}[c]{@{}l@{}}std::ranges::split\_view\\ std::views::split\end{tabular}         & 使用分隔符拆分视图。                      \\ \hline
\begin{tabular}[c]{@{}l@{}}std::ranges::common\_view\\ std::views::common\end{tabular}       & 将视图转换为std::ranges::common\_range。       \\ \hline
\begin{tabular}[c]{@{}l@{}}std::ranges::reverse\_view\\ std::views::reverse\end{tabular}     & 以反向顺序迭代。                               \\ \hline
\begin{tabular}[c]{@{}l@{}}std::ranges::basic\_istream\_view\\ std::views::istream\_view\end{tabular} &
可对相应视图使用操作符>{}>。 \\ \hline
\begin{tabular}[c]{@{}l@{}}std::ranges::elements\_view\\ std::views::elements\end{tabular}   & 在元组的第n个元素上创建视图。            \\ \hline
\begin{tabular}[c]{@{}l@{}}std::ranges::keys\_views\\ std::views::keys\end{tabular}          & 在类似对值的第一个元素上创建视图。 \\ \hline
\begin{tabular}[c]{@{}l@{}}std::ranges::values\_views\\ std::views::values\end{tabular} &
在类似对值的第二个元素上创建视图。 \\ \hline
\end{longtable}

通常,可以使用std::views::transform这样的视图,并将其替换为std::ranges::transform\_view。

有了范围库,算法可以直接应用于容器,可以组合,并且是惰性的。

C++23支持更多的视图:

\begin{center}
C++23中的视图
\end{center}

% Please add the following required packages to your document preamble:
% \usepackage{longtable}
% Note: It may be necessary to compile the document several times to get a multi-page table to line up properly
\begin{longtable}[c]{|l|l|}
\hline
\textbf{视图}                                                                                  & \textbf{描述}                                 \\ \hline
\endfirsthead
%
\endhead
%
\begin{tabular}[c]{@{}l@{}}std::ranges::zip\_view\\ std::views::zip\end{tabular}               & 创建元组的视图。                            \\ \hline
\begin{tabular}[c]{@{}l@{}}std::ranges::zip\_transform\_view\\ std::views::zip\_transform\end{tabular} &
通过转换函数创建元组的视图。 \\ \hline
\begin{tabular}[c]{@{}l@{}}std::ranges::adjacent\_view\\ std::views::adjacent\end{tabular}     & 创建相邻元素的视图。                 \\ \hline
\begin{tabular}[c]{@{}l@{}}std::ranges::adjacent\_transform\_view\\ std::views::adjacent\_transform\end{tabular} &
通过转换函数创建相邻元素的视图。 \\ \hline
\begin{tabular}[c]{@{}l@{}}std::ranges::join\_with\_view\\ std::views::join\_with\end{tabular} &
通过分隔符将现有范围连接到视图中。 \\ \hline
\begin{tabular}[c]{@{}l@{}}std::ranges::slide\_view\\ std::views::slide\end{tabular}           & 通过获取一个视图和一个数字N来创建N个元组。    \\ \hline
\begin{tabular}[c]{@{}l@{}}std::ranges::chunk\_view\\ std::views::chunk\end{tabular}           & 创建N个视图块和一个数字N。           \\ \hline
\begin{tabular}[c]{@{}l@{}}std::ranges::chunk\_by\_view\\ std::views::chunk\_by\end{tabular}   & 基于谓词创建视图块。       \\ \hline
\begin{tabular}[c]{@{}l@{}}std::ranges::as\_const\_view\\ std::views::as\_const\end{tabular}   & 将视图转换为常量范围。               \\ \hline
\begin{tabular}[c]{@{}l@{}}std::ranges::as\_rvalue\_view\\ std::views::as\_rvalue\end{tabular} & 将每个元素强制转换为右值。                   \\ \hline
\begin{tabular}[c]{@{}l@{}}std::ranges::stride\_view\\ std::views::stride\end{tabular}         & 创建另一个视图第n个元素的视图。 \\ \hline
\end{longtable}

下面的代码段中使用了C++23视图。

\filename{C++23中的新视图}

\begin{cpp}
// cpp23Ranges.cpp
...
#include <ranges>
...

std::vector vec = {1, 2, 3, 4};

for (auto i : vec | std::views::adjacent<2>) {
	std::cout << '(' << i.first << ", " << i.second << ") "; // (1, 2) (2, 3) (3, 4)
}

for (auto i : vec | std::views::adjacent_transform<2>(std::multiplies())) {
	std::cout << i << ' '; // 2 6 12
}

std::print("{}\n", vec | std::views::chunk(2)); // [[1, 2], [3, 4],
std::print("{}\n", vec | std::views::slide(2)); // [[1, 2], [2, 3], [3, 4]
for (auto i : vec | std::views::slide(2)) {
	std::cout << '[' << i[0] << ", " << i[1] << "] "; // [1, 2] [2, 3] [3, 4] [4, 5]
}

std::vector vec2 = {1, 2, 3, 0, 5, 2};
std::print("{}\n", vec2 | std::views::chunk_by(std::ranges::less_equal{}));
														// [[1, 2, 3], [0, 5], [2]]

for (auto i : vec | std::views::slide(2)) {
	std::cout << '[' << i[0] << ", " << i[1] << "] "; // [1, 2] [2, 3] [3, 4] [4, 5]
}
\end{cpp}

\mySamllsection{生成器}

C++23中的std::generator是第一个具体的协程生成器,生成器通过反复恢复暂停的协程来生成元素序列。

\filename{协程生成器std::generator}

\begin{cpp}
// generator.cpp
...
#include <generator>
#include <ranges>

...

std::generator<int> fib() {
	co_yield 0; // 1
	auto a = 0;
	auto b = 1;
	for(auto n : std::views::iota(0)) {
		auto next = a + b;
		a = b;
		b = next;
		co_yield next; // 2
	}
}

...

for (auto f : fib() | std::views::take(10)) { // 3
	std::cout << f << " "; // 0 1 1 2 3 5 8 13 21 34
}
\end{cpp}

函数fib返回一个协程,协程创建了一个无限的斐波那契数流。数字流从0(1)开始,继续到后面的斐波那契数(2)。基于范围的for循环显式地请求前10个斐波那契数(3)。





