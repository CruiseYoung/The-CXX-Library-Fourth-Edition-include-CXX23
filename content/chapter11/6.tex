\href{https://en.cppreference.com/w/cpp/ranges/iota_view}{std::views::iota}是一个范围工厂,用于通过连续递增初始值来创建元素序列。这个序列可以是有限的,也可以是无限的。有了这个函数,就可以找到以1000000开头的前20个质数。

\filename{找出20个以1000000开头的质数}

\begin{cpp}
// rangesLazy.cpp
...
#include <ranges>

bool isPrime(int i) {
	for (int j=2; j*j <= i; ++j){
		if (i % j == 0) return false;
	}
	return true;
}
											// (1)
std::cout << "Numbers from 1000000 to 1001000 (displayed each 100th): " << "\n";
for (int i: std::views::iota(1000000, 1001000)) {
	if (i % 100 == 0) std::cout << i << " ";
}
											// (2)
auto odd = [](int i){ return i % 2 == 1; };
std::cout << "Odd numbers from 1000000 to 1001000 (displayed each 100th): " << "\n";
for (int i: std::views::iota(1000000, 1001000) | std::views::filter(odd)) {
	if (i % 100 == 1) std::cout << i << " ";
}
	
											// (3)
std::cout << "Prime numbers from 1000000 to 1001000: " << '\n';
for (int i: std::views::iota(1000000, 1001000) | std::views::filter(odd)
											   | std::views::filter(isPrime)) {
	std::cout << i << " ";
}
											// (4)
std::cout << "20 prime numbers starting with 1000000: " << '\n';
for (int i: std::views::iota(1000000) | std::views::filter(odd)
									  | std::views::filter(isPrime)
									  | std::views::take(20)) {
	std::cout << i << " ";
}
\end{cpp}

以下是我的迭代策略:

\begin{enumerate}
\item 
我不知道什么时候有20个大于1000000的质数。为了安全起见,我创建了1000个数字,只显示每100个。

\item 
素数是奇数;因此,去掉了偶数。

\item 
谓词isPrime返回一个数字是否为质数。我得到75个质数,但我只想要20个。

\item 
我使用std::iota作为无限数字工厂,从1000000开始,并精确地要求20个质数。
\end{enumerate}

\myGraphic{0.8}{content/chapter11/images/4.jpg}{}












































