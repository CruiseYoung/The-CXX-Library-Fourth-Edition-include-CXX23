
\myGraphic{0.8}{content/chapter4/images/5.jpg}{}

\href{http://en.cppreference.com/w/cpp/container/list}{std::list} is a doubly linked list. std::list needs the header <list>.

Although it has a similar interface to std::vector or std::deque, std::list is quite different from both. That’s due to its structure.

std::list makes the following points unique:

\begin{itemize}
\item
It supports no random access.

\item
Access to an arbitrary element is slow because you have to iterate in the worst case through the whole list.

\item
To add or remove an element is fast if the iterator points to the right place.

\item
 If you add or remove an element, the iterator keeps valid.
\end{itemize}

Because of its unique structure, std::list has a few special member functions.


\begin{center}
Special member functions of std::list
\end{center}

% Please add the following required packages to your document preamble:
% \usepackage{longtable}
% Note: It may be necessary to compile the document several times to get a multi-page table to line up properly
\begin{longtable}[c]{|l|l|}
\hline
\textbf{Member Functions} & \textbf{Description}                                                         \\ \hline
\endfirsthead
%
\endhead
%
lis.merge(c)              & Merges the sorted list c into the sorted list lis, so that lis keeps sorted. \\ \hline
lis.merge(c, op)     & Merges the sorted list c into the sorted list lis, so that lis keeps sorted. It uses op as sorting criterion. \\ \hline
lis.remove(val)           & Removes all elements from lis with value val.                                \\ \hline
lis.remove\_if(pre)       & Removes all elements from lis, fulfilling the predicate pre.                 \\ \hline
lis.splice(pos, ...) & Splits the elements in lis before pos. The elements can be single elements, ranges, or lists.                 \\ \hline
lis.unique()              & Removes adjacent element with the same value.                                \\ \hline
lis.unique(pre)           & Removes adjacent elements, fulfilling the predicate pre.                     \\ \hline
\end{longtable}

Here are a few of the member functions in a code snippet.

\filename{std::list}

\begin{cpp}
// list.cpp
...
#include <list>
...
std::list<int> list1{15, 2, 18, 19, 4, 15, 1, 3, 18, 5,
	                  4, 7, 17, 9, 16, 8, 6, 6, 17, 1, 2};
	                  
list1.sort();
for (auto l: list1) std::cout << l << " ";
	// 1 1 2 2 3 4 4 5 6 6 7 8 9 15 15 16 17 17 18 18 19

list1.unique();
for (auto l: list1) std::cout << l << " ";
	// 1 2 3 4 5 6 7 8 9 15 16 17 18 19

std::list<int> list2{10, 11, 12, 13, 14};
list1.splice(std::find(list1.begin(), list1.end(), 15), list2);
for (auto l: list1) std::cout << l << " ";
	// 1 2 3 4 5 6 7 8 9 10 11 12 13 14 15 16 17 18 19
\end{cpp}


