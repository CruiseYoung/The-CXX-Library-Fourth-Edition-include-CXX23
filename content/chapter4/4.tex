
\myGraphic{0.8}{content/chapter4/images/5.jpg}{}

\href{http://en.cppreference.com/w/cpp/container/list}{std::list}是一个双链表。std::list需要包含头文件<list>。

尽管std::list与std::vector或std::deque有类似的接口,但std::list由于它的结构,所以两者有很大的不同。

std::list有以下几个特点:

\begin{itemize}
\item
不支持随机访问。

\item
访问元素的速度很慢,最坏的情况下,必须遍历整个列表。

\item
若迭代器指向正确的位置,添加或删除元素很快。

\item
若添加或删除元素,迭代器持续有效。
\end{itemize}

由于其独特的结构,std::list有一些特殊的成员函数。

\begin{center}
std::list的特殊成员函数
\end{center}

% Please add the following required packages to your document preamble:
% \usepackage{longtable}
% Note: It may be necessary to compile the document several times to get a multi-page table to line up properly
\begin{longtable}[c]{|l|l|}
\hline
\textbf{成员函数} & \textbf{描述}                                                         \\ \hline
\endfirsthead
%
\endhead
%
lis.merge(c)              & 将有序列表c合并到有序列表list中,list保持有序。 \\ \hline
lis.merge(c, op)     & 将有序列表c合并到有序列表list中,使list保持有序。其使用op作为排序标准。 \\ \hline
lis.remove(val)           & 从列表中删除值为val的所有元素。                                \\ \hline
lis.remove\_if(pre)       & 从列表中删除所有满足谓词pre的元素。                 \\ \hline
lis.splice(pos, ...) & 拆分pos之前list中的元素。元素可以是单个元素、范围或列表。                 \\ \hline
lis.unique()              & 移除相同值的相邻元素。                                \\ \hline
lis.unique(pre)           & 移除满足谓词pre的相邻的元素。                     \\ \hline
\end{longtable}

下面的代码段中展示如何使用其成员函数。

\filename{std::list}

\begin{cpp}
// list.cpp
...
#include <list>
...
std::list<int> list1{15, 2, 18, 19, 4, 15, 1, 3, 18, 5,
	                  4, 7, 17, 9, 16, 8, 6, 6, 17, 1, 2};
	                  
list1.sort();
for (auto l: list1) std::cout << l << " ";
	// 1 1 2 2 3 4 4 5 6 6 7 8 9 15 15 16 17 17 18 18 19

list1.unique();
for (auto l: list1) std::cout << l << " ";
	// 1 2 3 4 5 6 7 8 9 15 16 17 18 19

std::list<int> list2{10, 11, 12, 13, 14};
list1.splice(std::find(list1.begin(), list1.end(), 15), list2);
for (auto l: list1) std::cout << l << " ";
	// 1 2 3 4 5 6 7 8 9 10 11 12 13 14 15 16 17 18 19
\end{cpp}


