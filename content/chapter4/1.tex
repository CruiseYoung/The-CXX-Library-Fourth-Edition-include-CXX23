
\myGraphic{0.8}{content/chapter4/images/2.jpg}{}


The std::array combines the memory and runtime characteristic of a C array with the interface of

std::vector. \href{http://en.cppreference.com/w/cpp/container/array}{std::array} is a homogeneous container of fixed length. It needs the header <array>.This means, in particular, the std::array knows its size.

To initialize a std::array, you must follow a few special rules.

\noindent
std::array<int, 10> arr

The 10 elements are not initialized.

\noindent
std::array<int, 10> arr{}

The 10 elements are default initialized.

\noindent
std::array<int, 10> arr{1, 2, 3, 4, 5}

The remaining elements are default initialized.

std::array supports three types of index access.

\begin{cpp}
arr[n];
arr.at(n);
std::get<n>(arr);
\end{cpp}

The most often used first type form with angle brackets does not check the boundaries of the arr. This is in opposition to arr.at(n). You will eventually get a std::range-error exception. The last type shows the relationship of the std::array with the std::tuple because both are containers of fixed length.

Here is a little bit of arithmetic with std::array.

\filename{std::array}

\begin{cpp}
// array.cpp
...
#include <array>
...
std::array<int, 10> arr{1, 2, 3, 4, 5, 6, 7, 8, 9, 10};
for (auto a: arr) std::cout << a << " " ; // 1 2 3 4 5 6 7 8 9 10

double sum= std::accumulate(arr.begin(), arr.end(), 0);
std::cout << sum << '\n'; // 55

double mean= sum / arr.size();
std::cout << mean << '\n'; // 5.5
std::cout << (arr[0] == std::get<0>(arr)); // true
\end{cpp}




















