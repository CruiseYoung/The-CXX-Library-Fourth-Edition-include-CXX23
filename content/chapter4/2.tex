
\myGraphic{0.8}{content/chapter4/images/3.jpg}{}

\href{http://en.cppreference.com/w/cpp/container/vector}{std::vector}是同构容器,其长度在运行时自动调整,std::vector需要头文件<vector>。vector在内存中连续存储元素,因此std::vector支持指针算术。

\begin{cpp}
for (int i= 0; i < vec.size(); ++i){
	std::cout << vec[i] == *(vec + i) << '\n'; // true
}
\end{cpp}

\begin{myTip}{创建std::vector时,要区分圆括号和花括号}
构造std::vector时,必须注意一些特殊规则。下面的例子中,使用圆括号的构造函数创建了一个包含10个元素的std::vector,使用大括号的构造函数创建了一个仅包含10的std::vector。

\begin{cpp}
std::vector<int> vec(10);
std::vector<int> vec{10};
\end{cpp}

同样的规则也适用于std::vector<int>(10,2011)或std::vector<int>\{10,2011\}。第一种情况下,会得到一个std::vector,其中十个元素初始化为2011。在第二种情况下,会得到一个包含元素10和2011的std::vector。因为花括号代表一个初始化列表,所以会使用列表构造函数。
\end{myTip}

\mySamllsection{大小与容量}

std::vector的元素数量通常小于已为其预留空间的元素数量,std::vector的大小可以增加,而不需要分配新内存。

有一些操作可以智能地使用内存。

\begin{center}
std::vector的内存管理
\end{center}

% Please add the following required packages to your document preamble:
% \usepackage{longtable}
% Note: It may be necessary to compile the document several times to get a multi-page table to line up properly
\begin{longtable}[c]{|l|l|}
\hline
\textbf{成员函数} & \textbf{描述}                                       \\ \hline
\endfirsthead
%
\endhead
%
vec.size()                & vec的元素个数。                                 \\ \hline
vec.capacity()            & vec无需重新分配即可拥有的元素数。 \\ \hline
vec.resize(n)             & vec会将增加到n个元素。                       \\ \hline
vec.reserve(n)            & 至少保留n个元素内存。                    \\ \hline
vec.shrink\_to\_fit()     & 将vec的容量减小到其大小。                       \\ \hline
\end{longtable}


vec.shrink\_to\_fit()没有固定行为,所以运行时可以忽略它。但在主流的平台上,可观察到期望的行为。

我们来使用一下这些成员函数。

\filename{std::vector}

\begin{cpp}
// vector.cpp
...
#include <vector>
...
std::vector<int> intVec1(5, 2011);
intVec1.reserve(10);
std::cout << intVec1.size() << '\n'; // 5
std::cout << intVec1.capacity() << '\n'; // 10

intVec1.shrink_to_fit();
std::cout << intVec1.capacity() << '\n'; // 5

std::vector<int> intVec2(10);
std::cout << intVec2.size() << '\n'; // 10

std::vector<int> intVec3{10};
std::cout << intVec3.size() << '\n'; // 1

std::vector<int> intVec4{5, 2011};
std::cout << intVec4.size() << '\n'; // 2
\end{cpp}

std::vector vec有几个成员函数来访问它的元素。使用vec.front(),可获得第一个元素;使用vec.back(),可获得vec的最后一个元素。要读取或写入vec的(n+1)个元素,可以使用索引操作符vec[n]或成员函数vec.at(n)。第二个检查vec的边界,因此最终得到一个std::out\_of\_range异常。

除了索引操作符,std::vector还提供了额外的成员函数来赋值、插入、创建或删除元素。

\begin{center}
修改std::vector的元素
\end{center}

% Please add the following required packages to your document preamble:
% \usepackage{longtable}
% Note: It may be necessary to compile the document several times to get a multi-page table to line up properly
\begin{longtable}[c]{|l|l|}
\hline
\textbf{成员函数}  & \textbf{描述}                                           \\ \hline
\endfirsthead
%
\endhead
%
vec.assign(...)            & 赋值一个或多个元素、一个范围或初始化列表。 \\ \hline
vec.clear()                & 移除vec中所有的元素。                                 \\ \hline
vec.emplace(pos, args...) & 使用vec中的参数在pos之前创建一个新元素,并返回该元素的新位置。          \\ \hline
vec.emplace\_back(args...) & 在vec中创建一个带有参数的新元素args…                     \\ \hline
vec.erase(...)             & 移除一个元素或一个范围,并返回下一个位置。  \\ \hline
vec.insert(pos, ...)      & 插入一个或多个元素、一个范围或一个初始化列表,并返回元素的新位置。 \\ \hline
vec.pop\_back()            & 删除最后一个元素。                                     \\ \hline
vec.push\_back(elem)       & 在vec的末尾添加elem的副本。                         \\ \hline
\end{longtable}























