C++和标准库有着悠久的历史,其始于上个千年的20世纪80年代,结束于2023年。了解软件开发的人都知道我们的领域发展得有多快。所以40年是很长的一段时间。C++的第一个组件,如I/O流,是用与现代标准模板库(STL)不同的思维方式设计的。C++最初是一种面向对象的语言,将泛型编程与STL结合在一起,现在采用了许多函数式编程思想。可以在c++标准库中观察到,过去40年软件开发的这种演变,也是软件问题解决方式的演变。

\myGraphic{1.0}{content/chapter1/images/1.jpg}{C++的时间线}

1998年的第一个C++98标准库有三个组件。这些是前面提到的I/O流,主要用于文件处理、字符串库和标准模板库。

标准模板库促进了算法在容器上的透明应用。2005年技术报告1 (TR1)延续了这一历史。C++库ISO/IEC TR 19768的扩展并不是官方标准,但几乎所有的组件都成为C++11的一部分。例如,基于boost库(http//www.boost.org/)的正则表达式、智能指针、哈希表、随机数和时间库。

除了TR1的标准化之外,C++11还增加了一个新组件:多线程库。

C++14只是对C++11标准的一个小更新。因此,C++14只对现有的智能指针、元组、类型特征和多线程库进行了一些改进。

C++17包含了文件系统的库和两个新的数据类型std::any和std::optional。

C++20有四个突出的特性:概念、范围、协程和模块。除了这四大特性之外,C++20中还有更多的特性:三方比较操作符、格式化库以及与并发相关的数据类型——信号量、锁存器和栅栏。

C++23改进了C++20的四大功能:扩展范围功能、协程生成器std::generator和模块化的C++标准库。












