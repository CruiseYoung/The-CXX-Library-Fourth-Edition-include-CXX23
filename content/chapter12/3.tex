C++支持基本和高级数学常数。数学常数的数据类型为double,其位于命名空间std::number中,是头文件<numbers>的一部分。

\begin{center}
数学常数
\end{center}

% Please add the following required packages to your document preamble:
% \usepackage{longtable}
% Note: It may be necessary to compile the document several times to get a multi-page table to line up properly
\begin{longtable}[c]{|l|l|}
\hline
\textbf{数学常数} & \textbf{表示} \\ \hline
\endfirsthead
%
\endhead
%
std::numbers::e                & $e$                \\ \hline
std::numbers::log2e            & $log_{2}e$                 \\ \hline
std::numbers::log10e           & $log_{10}e$                  \\ \hline
std::numbers::pi               & $\pi$              \\ \hline
std::numbers::inv\_pi          & $\frac{1}{\pi}$                \\ \hline
std::numbers::inv\_sqrtpi      & $\frac{1}{\sqrt{\pi}}$      \\ \hline
std::numbers::ln2              & $ln2$                  \\ \hline
std::numbers::ln10             & $ln10$                  \\ \hline
std::numbers::sqrt2            & $\sqrt{2}$                 \\ \hline
std::numbers::sqrt3            & $\sqrt{3}$                 \\ \hline
std::numbers::inv\_sqrt3       & $\frac{1}{\sqrt{3}}$         \\ \hline
std::numbers::egamma           &  \href{https://en.wikipedia.org/wiki/Euler%E2%80%93Mascheroni_constant}{Euler-Mascheroni constant}                \\ \hline
std::numbers::phi              &      $\phi$            \\ \hline
\end{longtable}

下面的代码段展示了所有的数学常数。

\filename{数学常数}

\begin{cpp}
// mathematicalConstants.cpp
#include <numbers>
...

std::cout<< std::setprecision(10);

std::cout << "std::numbers::e: " << std::numbers::e << '\n';
std::cout << "std::numbers::log2e: " << std::numbers::log2e << '\n';
std::cout << "std::numbers::log10e: " << std::numbers::log10e << '\n';
std::cout << "std::numbers::pi: " << std::numbers::pi << '\n';
std::cout << "std::numbers::inv_pi: " << std::numbers::inv_pi << '\n';
std::cout << "std::numbers::inv_sqrtpi: " << std::numbers::inv_sqrtpi << '\n';
std::cout << "std::numbers::ln2: " << std::numbers::ln2 << '\n';
std::cout << "std::numbers::sqrt2: " << std::numbers::sqrt2 << '\n';
std::cout << "std::numbers::sqrt3: " << std::numbers::sqrt3 << '\n';
std::cout << "std::numbers::inv_sqrt3: " << std::numbers::inv_sqrt3 << '\n';
std::cout << "std::numbers::egamma: " << std::numbers::egamma << '\n';
std::cout << "std::numbers::phi: " << std::numbers::phi << '\n';
\end{cpp}

\myGraphic{0.7}{content/chapter12/images/4.jpg}{}





























