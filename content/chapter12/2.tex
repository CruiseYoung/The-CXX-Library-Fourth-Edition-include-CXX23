C++从C继承了许多数值函数,需要头文件\href{http://en.cppreference.com/w/cpp/numeric/math}{<cmath>}。下表显示了这些函数的名称。

\begin{center}
<cmath>中的数学函数
\end{center}

% Please add the following required packages to your document preamble:
% \usepackage{longtable}
% Note: It may be necessary to compile the document several times to get a multi-page table to line up properly
\begin{longtable}[c]{lllll}
pow   & sin  & tanh  & asinh  & fabs  \\
\endfirsthead
%
\endhead
%
exp   & cos  & asin  & aconsh & fmod  \\
sqrt  & tan  & acos  & atanh  & frexp \\
log   & sinh & atan  & ceil   & ldexp \\
log10 & cosh & atan2 & floor  & modf 
\end{longtable}

另外,C++从C中继承了数学函数。它们在头文件\href{http://en.cppreference.com/w/cpp/numeric/math}{<cstdlib>}中定义。

再说一遍,这些都是名字。

\begin{center}
<cstdlib>中的数学函数
\end{center}

% Please add the following required packages to your document preamble:
% \usepackage{longtable}
% Note: It may be necessary to compile the document several times to get a multi-page table to line up properly
\begin{longtable}[c]{llll}
abs  & llabs & ldiv  & srand \\
\endfirsthead
%
\endhead
%
labs & div   & lldiv & rand 
\end{longtable}

所有用于整型的函数都可用于int、long和long long;所有用于浮点数的函数都可用于float、double和long double类型。

数值函数需要使用命名空间std进行限定。

\filename{数学函数}

\begin{cpp}
// mathFunctions.cpp
...
#include <cmath>
#include <cstdlib>
...

std::cout << std::pow(2, 10); // 1024
std::cout << std::pow(2, 0.5); // 1.41421
std::cout << std::exp(1); // 2.71828
std::cout << std::ceil(5.5); // 6
std::cout << std::floor(5.5); // 5
std::cout << std::fmod(5.5, 2); // 1.5

double intPart;
auto fracPart= std::modf(5.7, &intPart);
std::cout << intPart << " + " << fracPart; // 5 + 0.7
std::div_t divresult= std::div(14, 5);
std::cout << divresult.quot << " " << divresult.rem; // 2 4

// seed
std::srand(time(nullptr));
for (int i= 0;i < 10; ++i) std::cout << (rand()%6 + 1) << " ";
											// 3 6 5 3 6 5 6 3 1 5
\end{cpp}















